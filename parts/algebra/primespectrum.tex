\documentclass{article}
\usepackage{amsmath,amssymb,amsthm,textcomp,gensymb,nccmath}
\usepackage{mathtools}
\renewcommand{\qedsymbol}{$\blacksquare$}

\setlength{\topmargin}{0.5in}
\usepackage[margin=4cm]{geometry}
\usepackage{enumerate}

\usepackage{setspace}
\onehalfspacing
\usepackage{parskip}
\setlength{\parskip}{0.5em}
\usepackage[T1]{fontenc}
\usepackage{palatino}

% useful characters/operators
\newcommand{\R}{\mathbb{R}}
\newcommand{\C}{\mathbb{C}}
\newcommand{\Z}{\mathbb{Z}}
\newcommand{\Q}{\mathbb{Q}}
\newcommand{\N}{\mathbb{N}}
\newcommand{\matP}{\mathbb{P}}
\newcommand{\matS}{\mathbb{S}}
\newcommand{\matH}{\mathbb{H}}
\newcommand{\matT}{\mathbb{T}}
\newcommand{\st}{\ s.t.\ }
\newcommand{\ie}{\ i.e.\ }
\newcommand{\eg}{\ e.g.\ }
\def \diam {\operatorname{diam}}
\def \Hom {\operatorname{Hom}}
\def \id {\operatorname{id}}
\def \tr {\operatorname{tr}}
\def \dist {\operatorname{dist}}
\def \intr {\operatorname{int}}
\def \sgn {\operatorname{sgn}}
\def \im {\operatorname{Im}}
\def \re {\operatorname{Re}}
\def \curl {\operatorname{curl}}
\def \divg {\operatorname{div}}
\def \GL {\operatorname{GL}}
\def \End {\operatorname{End}}
\def \Aut {\operatorname{Aut}}
\def \Spec {\operatorname{Spec}}
\newcommand{\pdr}[2]{\dfrac{\partial #1}{\partial #2}}
\newcommand{\dr}[2]{\dfrac{\text{d} #1}{\text{d} #2}}
\newcommand{\df}{\text{d}}
\newcommand{\inner}[2]{\left\langle #1, #2\right\rangle}

% arrows and :=, =:
\makeatletter
\providecommand*{\twoheadrightarrowfill@}{%
  \arrowfill@\relbar\relbar\twoheadrightarrow
}
\providecommand*{\twoheadleftarrowfill@}{%
  \arrowfill@\twoheadleftarrow\relbar\relbar
}
\providecommand*{\xtwoheadrightarrow}[2][]{%
  \ext@arrow 0579\twoheadrightarrowfill@{#1}{#2}%
}
\providecommand*{\xtwoheadleftarrow}[2][]{%
  \ext@arrow 5097\twoheadleftarrowfill@{#1}{#2}%
}
\makeatother

\newcommand{\defeq}{\vcentcolon=}
\newcommand{\eqdef}{=\mathrel{\mathop:}}

% integral for measure theory
\newcommand{\lowerint}{\underline{\int_{\R^d}}}
\newcommand{\upperint}{\overline{\int_{\R^d}}}
\newcommand{\lint}[1]{\underline{\int_{\R^d}} #1 (x)dx}
\newcommand{\uint}[1]{\overline{\int_{\R^d}} #1 (x)dx}
\newcommand{\sint}[1]{\simp{\int_{\R^d} #1 (x)dx}}
\newcommand{\lesint}[1]{\int_{\R^d} #1 (x)dx}

% note taking
\newcommand{\fancyem}[1]{\underline{\textsc{#1}}}

% theorem style
\newtheorem*{theorem}{Theorem}
\newtheorem*{corollary}{Corollary}
\newtheorem*{lemma}{Lemma}
\newtheorem*{conjecture}{Conjecture}
\newtheorem*{proposition}{Proposition}

\theoremstyle{definition}
\newtheorem*{definition}{Definition}
\newtheorem*{example}{Example}
\theoremstyle{remark}
\newtheorem*{remark}{Remark}

% for clearer reference
\usepackage{hyperref}
\newcommand{\corollaryautorefname}{Corollary}
\newcommand{\lemmaautorefname}{Lemma}
\newcommand{\definitionautorefname}{Definition}
\newcommand{\exampleautorefname}{Example}
\newcommand{\conjectureautorefname}{Conjecture}
\renewcommand{\subsectionautorefname}{Section}

% other styling
\usepackage{fancyvrb, fancyhdr}
\usepackage{tikz}
\usepackage{tcolorbox}

\usepackage{tikz-cd}

\begin{document}
% \renewcommand{\ref}[1]{\autoref{#1}}
\title{Math 494}
\author{Yiwei Fu}
\date{Jan 28, 2022}
\maketitle

\begin{proof}
    \begin{enumerate}
        \item Fix $P \in \Spec(R)$ and $X - V(E)$ open set containing $P$. We want to show that $\exists f \in R \st P \in X_f \subset X - V(E)$.

        $P \in X - V(E) \implies \exists f \in E \st f \notin P \implies P \in X_f$ and $f \in E,  \forall P \in F(U), X$

        \item We take the basis $\{X_{f_i}\}_{i \in I}$ be a cover of $X$. $\forall P \in \Spec(R), \exists X_{f_i} \ni P \implies f_i \notin P$.

        By lemma in class (every non $(1)$ ideal is contained in a maximal ideal) we have $(\{f_i\}_{i \in I}) = R \ni 1 \implies \exists J \subset I$ finite, $g_i \in R \st \sum_{j \in J} g_jf_j = 1$.
        
        So $(\{f_j\}_{j \in I}) = R$. This forms a finite subcover.

        \item Suppose $\{P\}$ is closed $\implies \{P\} = V(E), E \subset R \implies P$ is the only prime containing $E \implies P$ is maximal.
        
        Suppose $M$ is a maximal ideal, then $\{M\} = V(M)$, then it is closed by assumption. \qedhere 
    \end{enumerate}

\end{proof}

\begin{proof}
    \begin{enumerate}
        \item Suppose $f, g \in \phi^{-1}(P), a, b \in R, \phi(af + bg) = \phi(a)\phi(f) + \phi(b)\phi(g) \in P$. Preimage is an ideal.
        
        To show that it is prime, suppose $fg \in \phi^{-1}(p) \implies \phi(fg) = \phi(f)\phi(g) \in P$. So either $\phi(f) \in P$ or $\phi(g) \in P$.

        \item Say $Q \in Y_{\phi(f)} \iff \phi(f) \notin Q \iff f \notin \phi^{-1}(Q) = \phi^*(Q) \iff \phi^*(Q) \in X_f \iff Q \in \phi^*(X_f)$.
        
        \item $f \in (\psi \circ \phi)^*(P) \iff (\psi \circ \phi)(f) \in P \iff \psi(\phi(f)) \in P \iff \phi(f) \in \phi^*(P) \iff f \in (\phi^* \circ \psi^*)(P)$. \qedhere
    \end{enumerate}
\end{proof}


\end{document}
