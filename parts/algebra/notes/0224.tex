\documentclass{article}
\usepackage{amsmath,amssymb,amsthm,textcomp,gensymb,nccmath}
\usepackage{mathtools}
\renewcommand{\qedsymbol}{$\blacksquare$}

\setlength{\topmargin}{0.5in}
\usepackage[margin=4cm]{geometry}
\usepackage{enumerate}

\usepackage{setspace}
\onehalfspacing
\usepackage{parskip}
\setlength{\parskip}{0.5em}
\usepackage[T1]{fontenc}
\usepackage{palatino}

% useful characters/operators
\newcommand{\R}{\mathbb{R}}
\newcommand{\C}{\mathbb{C}}
\newcommand{\Z}{\mathbb{Z}}
\newcommand{\Q}{\mathbb{Q}}
\newcommand{\N}{\mathbb{N}}
\newcommand{\matP}{\mathbb{P}}
\newcommand{\matS}{\mathbb{S}}
\newcommand{\matH}{\mathbb{H}}
\newcommand{\matT}{\mathbb{T}}
\newcommand{\st}{\ s.t.\ }
\newcommand{\ie}{\ i.e.\ }
\newcommand{\eg}{\ e.g.\ }
\def \diam {\operatorname{diam}}
\def \Hom {\operatorname{Hom}}
\def \id {\operatorname{id}}
\def \tr {\operatorname{tr}}
\def \dist {\operatorname{dist}}
\def \intr {\operatorname{int}}
\def \sgn {\operatorname{sgn}}
\def \im {\operatorname{Im}}
\def \re {\operatorname{Re}}
\def \curl {\operatorname{curl}}
\def \divg {\operatorname{div}}
\def \GL {\operatorname{GL}}
\def \End {\operatorname{End}}
\def \Aut {\operatorname{Aut}}
\def \Fr {\operatorname{Frac}}
\def \cont {\operatorname{cont}}
\newcommand{\pdr}[2]{\dfrac{\partial #1}{\partial #2}}
\newcommand{\dr}[2]{\dfrac{\text{d} #1}{\text{d} #2}}
\newcommand{\df}{\text{d}}
\newcommand{\inner}[2]{\left\langle #1, #2\right\rangle}

% arrows and :=, =:
\makeatletter
\providecommand*{\twoheadrightarrowfill@}{%
  \arrowfill@\relbar\relbar\twoheadrightarrow
}
\providecommand*{\twoheadleftarrowfill@}{%
  \arrowfill@\twoheadleftarrow\relbar\relbar
}
\providecommand*{\xtwoheadrightarrow}[2][]{%
  \ext@arrow 0579\twoheadrightarrowfill@{#1}{#2}%
}
\providecommand*{\xtwoheadleftarrow}[2][]{%
  \ext@arrow 5097\twoheadleftarrowfill@{#1}{#2}%
}
\makeatother
\newcommand{\defeq}{\vcentcolon=}
\newcommand{\eqdef}{=\mathrel{\mathop:}}

% integral for measure theory
\newcommand{\lowerint}{\underline{\int_{\R^d}}}
\newcommand{\upperint}{\overline{\int_{\R^d}}}
\newcommand{\lint}[1]{\underline{\int_{\R^d}} #1 (x)dx}
\newcommand{\uint}[1]{\overline{\int_{\R^d}} #1 (x)dx}
\newcommand{\sint}[1]{\simp{\int_{\R^d} #1 (x)dx}}
\newcommand{\lesint}[1]{\int_{\R^d} #1 (x)dx}

% note taking
\newcommand{\fancyem}[1]{\underline{\textsc{#1}}}

% theorem style
\newtheorem*{theorem}{Theorem}
\newtheorem*{corollary}{Corollary}
\newtheorem*{lemma}{Lemma}
\newtheorem*{conjecture}{Conjecture}
\newtheorem*{proposition}{Proposition}

\theoremstyle{definition}
\newtheorem*{definition}{Definition}
\newtheorem*{example}{Example}
\theoremstyle{remark}
\newtheorem*{remark}{Remark}

% for clearer reference
\usepackage{hyperref}
\newcommand{\corollaryautorefname}{Corollary}
\newcommand{\lemmaautorefname}{Lemma}
\newcommand{\definitionautorefname}{Definition}
\newcommand{\exampleautorefname}{Example}
\newcommand{\conjectureautorefname}{Conjecture}
\renewcommand{\subsectionautorefname}{Section}

% other styling
\usepackage{fancyvrb, fancyhdr}
\usepackage{tikz}
\usepackage{tcolorbox}

\pagestyle{fancy}
\fancyhead[LO,L]{Feb 24, 2022}
\fancyhead[RO,R]{Yiwei Fu}
\fancyhead[C]{MATH 494}
\fancyfoot[CO,C]{\thepage}

\usepackage{tikz-cd}

\usepackage{epigraph}

% \epigraphsize{\small}% Default
\setlength\epigraphwidth{8cm}
\setlength\epigraphrule{0pt}

\usepackage{etoolbox}

\makeatletter
\patchcmd{\epigraph}{\@epitext{#1}}{\itshape\@epitext{#1}}{}{}
\makeatother


\begin{document}
% \renewcommand{\ref}[1]{\autoref{#1}}
% \title{Math 494}
% \author{Yiwei Fu}
% \date{Feb 03, 2022}
% \maketitle

\section*{Constructions with straightedge and compass}
Start with two points:

Build from those: 
\begin{itemize}
  \item points, lines, circles
  \item Given 2 points, can construct the line through them and the circle centered at one point and passing through the other
  \item Given 2 lines (or 2 circles, or 1 line 1 circle) can construct their intersection.
\end{itemize}

Say a number $\ell \in \R$ is \emph{constructible} if, with straightedge and compass, we can construct a point $C$ on the line through $A$ to $B$ such that the (signed) distance from $A$ to $C$ is $\ell$ times the (signed) distance from $A$ to $B$.

\begin{theorem}
  A number $\ell \in \R$ is constructible if and only if $\ell \in K_n$ where $\Q = K_0 \subset K_1 \subset K_2 \subset \ldots \subset K_n$ and $K_i = K_{i-1}(\sqrt{\alpha_i})$ with $\alpha_i \in K_{i-1} \cap \R_{> 0}$.
\end{theorem}

\begin{corollary}
  If $\ell \in \R$ is constructible then $[\Q(\ell): \Q] = 2^m$, $m \in \Z_{\geq 0}$.
\end{corollary}
\begin{remark}
  The converse is not true.
\end{remark}

Three classic problems:
\begin{itemize}
  \item Squaring a circle
  \item Trisect an angle
  \item Duplicate a cube
\end{itemize}

Under straightedge and compass construction,
\begin{enumerate}
  \item It is impossible to duplicate a cube that has twice the volume as the original cube, since $[\Q(\sqrt[3]{2}):\Q] = 3$.
  \item It is impossible to squaring a circle, i.e. construct a square whose area is that of a given circle, i.e. $\sqrt{\pi}$ is not constructible, since $[\Q(\sqrt{\pi}):\Q] = \infty$.
  \item It is impossible to trisect an arbitrary angle. ($\cos 60\degree = \frac{1}{2}$ is constructible but $\cos 20\degree$ is not)
  \begin{align*}
    & \cos 3\theta = 4\cos^3\theta - 3\cos\theta \\
    \implies & \frac{1}{2} = 4(\cos 20\degree)^3 - 3 \cos 20\degree \\
    \implies & \cos 20\degree \text{ is a root of } 4x^3 - 3x - \frac{1}{2}.
  \end{align*}
  Let $x = \frac{y}{2}$, then $8x^3 - 6x - 1 = y^3 - 3y - 1$ is irreducible in $\Z/2\Z[y]$. So $8x^3 - 6x - 1$ is irreducible in $\Z[x] \implies $ irreducible in $\Q[y] \implies 4x^3 - 3x - \frac{1}{2}$ is irreducible in $\Q[x] \implies [\Q(\cos 20\degree): \Q] = 3$.
\end{enumerate}

So we can construct a $17$-gon since
\[
  \cos \frac{2\pi}{17} = \frac{1}{16} + \frac{\sqrt{17}}{16} + \frac{\sqrt{34 - 2\sqrt{17}}}{16} + \frac{1}{8}\sqrt{17 + 3\sqrt{17} - \sqrt{34 - 2\sqrt{17}} - 2\sqrt{34 + 2\sqrt{17}}}.
\]

\[
\Q \to \Q(\sqrt{17}) \to \Q\left(\sqrt{34 + 2\sqrt{17}}\right) \left(\ni \sqrt{34 - 2\sqrt{17}} = \frac{8\sqrt{17}}{\sqrt{34 + 2\sqrt{17}}}\right) \to \Q\left(\cos \frac{2\pi}{17}.\right)
\]

\fancyem{Claim} Can construct a regular $n$-gon if and only if can construct $\cos \frac{2\pi}{n}$.

\begin{corollary}
  If $p$ is prime and a regular $p$-gon is constructible then $p = 2^k + 1$. ($\implies k = 2^\ell$)
\end{corollary}

$\alpha = e^{2\pi i/p} = \cos \frac{2\pi}{p} + i\sin\frac{2\pi}{p}$.

$\frac{\alpha + \frac{1}{\alpha}}{2} = \cos \frac{2\pi}{p}, \alpha^2 - (2\cos \frac{2\pi}{p})\alpha + 1 = 0 \implies \left[\Q\left(\cos \frac{2\pi}{p}, \alpha\right): \Q\left(\cos\frac{2\pi}{p}\right)\right] \leq 2$.

Since $\alpha \notin \R \ni \cos \frac{2\pi}{p}$, this is equal. But $\operatorname{minpol}_\Q(\alpha) = x^{p-1} + x^{p-2} + \ldots + 1 \implies [\Q(\alpha): \Q] = p-1$.

Then $\Q(\alpha) \xrightarrow[]{2} \Q\left(\cos \frac{2\pi}{p}\right) \xrightarrow[]{\frac{p-1}{2}} \Q \implies []$ so if $p$-gon is constructible then $p - 1 = 2^k$.

A line through 2 points with coordinates in a field $K$ has an equation over $K$. The intersection of two lines with equations over $K$ is either $\emptyset$ or a point with coordinates in $K$. The intersection of a line over $K$ with a circle in $K$ is either $\emptyset$, a point with coordinate in $K$, two point with coordinates in $K$, or two points with coordinates in $K(\sqrt{\alpha})$ for some $\alpha \in K, \alpha > 0$.

Intersecting two circles:
$x^2 + y^2 = r_1$, $(x-a)^2 + (y-b)^2 = r_2 \implies -2ax + a^2 - 2by + b^2 = r_2 - r_1$.
If $(a, b) \neq (0, 0)$ then it defines a line over $K$. 
So: intersection of two circles over $K$ is either $\emptyset$, a point over $K$, two points in $K$, or two points over $K(\sqrt{\alpha})$.

Some geometry:
\begin{enumerate}
  \item Given 2 points, we can construct perpendicular bisector of the segment between them.
  \item Given a line $\ell$ and a point $p$, can construct a line through $p$ which is perpendicular to $\ell$.
  \item Given a line $\ell$ and a point $p$, we construct a line through $p$ that is parallel to $\ell$.
  \item Given points $p, q, r$ and a line $\ell$ containing $r$, can construct a point $s$ on $\ell$ such that the segments $rs$ and $pq$ gave the same length.
  \item Can construct an angle iff can construct $\cos\theta$.
  \item If $a, b$ are constructible then so are $a + b, -a, ab$ (through similar triangles), $\frac{1}{a}$. (So constructible numbers is a field.) Also $\sqrt{a}$.
\end{enumerate}



\end{document}
