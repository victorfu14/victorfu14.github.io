\documentclass{article}
\usepackage{amsmath,amssymb,amsthm,textcomp,gensymb}
\usepackage{mathtools}
\renewcommand{\qedsymbol}{$\blacksquare$}

\setlength{\topmargin}{0.5in}
\usepackage[margin=4cm]{geometry}
\usepackage{enumerate}

\usepackage{setspace}
\onehalfspacing
\usepackage{parskip}
\setlength{\parskip}{0.5em}
\usepackage[T1]{fontenc}
\usepackage{palatino}

% useful characters/operators
\newcommand{\R}{\mathbb{R}}
\newcommand{\C}{\mathbb{C}}
\newcommand{\Z}{\mathbb{Z}}
\newcommand{\Q}{\mathbb{Q}}
\newcommand{\N}{\mathbb{N}}
\newcommand{\matP}{\mathbb{P}}
\newcommand{\matS}{\mathbb{S}}
\newcommand{\matH}{\mathbb{H}}
\newcommand{\matT}{\mathbb{T}}
\newcommand{\st}{\ s.t.\ }
\newcommand{\ie}{\ i.e.\ }
\newcommand{\eg}{\ e.g.\ }
\def \diam {\operatorname{diam}}
\def \Hom {\operatorname{Hom}}
\def \id {\operatorname{id}}
\def \tr {\operatorname{tr}}
\def \dist {\operatorname{dist}}
\def \intr {\operatorname{int}}
\def \sgn {\operatorname{sgn}}
\def \im {\operatorname{Im}}
\def \re {\operatorname{Re}}
\def \curl {\operatorname{curl}}
\def \divg {\operatorname{div}}
\def \GL {\operatorname{GL}}
\def \End {\operatorname{End}}
\def \Aut {\operatorname{Aut}}
\newcommand{\pdr}[2]{\dfrac{\partial #1}{\partial #2}}
\newcommand{\dr}[2]{\dfrac{\text{d} #1}{\text{d} #2}}
\newcommand{\df}{\text{d}}
\newcommand{\inner}[2]{\left\langle #1, #2\right\rangle}

% arrows and :=, =:
\makeatletter
\providecommand*{\twoheadrightarrowfill@}{%
  \arrowfill@\relbar\relbar\twoheadrightarrow
}
\providecommand*{\twoheadleftarrowfill@}{%
  \arrowfill@\twoheadleftarrow\relbar\relbar
}
\providecommand*{\xtwoheadrightarrow}[2][]{%
  \ext@arrow 0579\twoheadrightarrowfill@{#1}{#2}%
}
\providecommand*{\xtwoheadleftarrow}[2][]{%
  \ext@arrow 5097\twoheadleftarrowfill@{#1}{#2}%
}
\makeatother

\newcommand{\defeq}{\vcentcolon=}
\newcommand{\eqdef}{=\mathrel{\mathop:}}

% integral for measure theory
\newcommand{\lowerint}{\underline{\int_{\R^d}}}
\newcommand{\upperint}{\overline{\int_{\R^d}}}
\newcommand{\lint}[1]{\underline{\int_{\R^d}} #1 (x)dx}
\newcommand{\uint}[1]{\overline{\int_{\R^d}} #1 (x)dx}
\newcommand{\sint}[1]{\simp{\int_{\R^d} #1 (x)dx}}
\newcommand{\lesint}[1]{\int_{\R^d} #1 (x)dx}

% note taking
\newcommand{\fancyem}[1]{\underline{\textsc{#1}}}

% theorem style
\newtheorem{theorem}{Theorem}[section]
\newtheorem{corollary}{Corollary}[theorem]
\newtheorem{lemma}{Lemma}[section]
\newtheorem{conjecture}{Conjecture}[section]
\newtheorem{proposition}{Proposition}[section]

\theoremstyle{definition}
\newtheorem{definition}{Definition}[section]
\newtheorem{example}[theorem]{Example}
\theoremstyle{remark}
\newtheorem*{remark}{Remark}

% for clearer reference
\usepackage{hyperref}
\newcommand{\corollaryautorefname}{Corollary}
\newcommand{\lemmaautorefname}{Lemma}
\newcommand{\definitionautorefname}{Definition}
\newcommand{\exampleautorefname}{Example}
\newcommand{\conjectureautorefname}{Conjecture}
\renewcommand{\subsectionautorefname}{Section}

% other styling
\usepackage{fancyvrb, fancyhdr}
\usepackage{tikz}
\usepackage{tcolorbox}

\begin{document}
\renewcommand{\ref}[1]{\autoref{#1}}
\title{Math 494}
\author{Yiwei Fu}
\date{}
\maketitle

\section{}
Lase time: Defined a ring to be an abelian group under $+$ which has an associative operation $*: R \times R \to R$ with an identity element in $R$ such that the distribution law holds.

\begin{example}
$G$ is an abelian group under $+ \implies \operatorname{End}(G)$ is a ring under
\[
(\phi + \psi)(g) \defeq \phi(g) + \psi(g), \phi * \psi \defeq \phi \circ \psi.
\]
\end{example}

\begin{definition}
Suppose $R, S$ are rings. A function $f: R \to S$ is a ring homomorphism if 
\[
f(r_1 + r_2) = f(r_1) + f(r_2), f(r_1 * r_2) = f(r_1) * f(r_2), f(1_R) = f(1_S)
\]
\end{definition}
\begin{lemma}
If $f: R \to S$ is a ring homomorphism then $f(R)$ is a ring
\end{lemma}

\begin{definition}
A \underline{subring} of a ring $R$ is a subset of $R$ which is a ring under $+, *$ from $R$.
\end{definition}
\begin{definition}
Suppose $R, S$ are rings. An \underline{isomorphism} $f: R \to S$ is a bijective homomorphism.
\end{definition}
\fancyem{Note}: The set-theoretic inverse $f^{-1}: S \to R$ is then a ring homomorphism.

\begin{lemma}
$R$ is a ring $\implies r\cdot 0 = 0 \cdot r = 0,\ \forall r \in R.$
\end{lemma}
\begin{proof}
\[0 + 0 = 0 \implies r\cdot(0 + 0) = r \cdot 0. \qedhere
\]
\end{proof}

\begin{lemma}
\[(-1)\cdot r = -r = r \cdot (-1)\]
\end{lemma}
\begin{proof}
\[
1 + (-1) = 0 \implies (1 + (-1)) \cdot r = 0 \cdot r \implies r + (-1)\cdot r = 0 \qedhere
\]
\end{proof}
It is always to check since we are so used to commutative things but it is not always the case.

\begin{theorem}
Every ring is isomorphic to a subring of $\operatorname{End}(G)$ for some abelian group $G.$
\end{theorem}
\begin{proof}
Let $G$ be the additive group of $R$. For $r \in R$ define $[r]: G \to G, g \mapsto rg.$ Check:
\begin{enumerate}
\item
	$[r] \in \operatorname{End}(G).$
	\[[r](g_1 + g_2) \defeq r(g_1 + g_2) = rg_1 + rg_2 = ([r]g_1) + ([r]g_2).\]
\item
	$r \mapsto [r]$ is a ring homomorphism $R \to \End(G)$
	\[
	[rs](g) = (rs)g = r(sg) = [r](sg) = [r]([s]g)
	\]
\item
	$r \mapsto [r]$ is injective
\end{enumerate}
So $\phi: R \to \End(G), r \mapsto [r]$ is an injective ring homomorphism. Hence $\phi(R)$ is a subring of $\End(G)$ and $\phi: R \to \phi(R)$ is an isomorphism.
\end{proof}

\begin{definition}
Suppose $R$ is a ring and $r \in R.$ Say $s \in R$ is a inverse of $r$ if $rs = 1 = sr.$ If $r$ has an inverse then say $r$ is a unit in $R$. Write $R^*$ or $R^\times$ for the set of units in $R$. 
\end{definition}

\fancyem{Note:} if $s$ exists then it is unique:
\[rs = 1 = tr\implies (rs)s = (tr)s = t(rs) \implies s = t.\]
So we can denote $s$ as $r^{-1}.$

\fancyem{Note:} $R^*$ is a group under multiplication.

\begin{example}
If $G$ is a abelian group, then $(\End(G))^* = \Aut(G),$ 
\[\End(\Z \text{ as a group}) \cong \Z \text{ as a ring},\ \Aut(\Z \text{ as a group}) = \{\pm 1\}.\]
\[\End(C_m) \cong \Z/m\Z \text{ as a ring},\ \Aut(C_m) = (\Z/m\Z)^* = \{k \bmod m: \gcd(k, m) = 1\}.\]
\[\Aut(\Z \times \Z \text{ as a group}) \cong \GL_2(\Z)\]
\end{example}

More rings:
$\Z[\sqrt{2}] = \{a + b\sqrt{2}: a, b \in \Z\}.$ It turns out that all units are $\pm(1  + \sqrt{2})^n,\ n \in \Z.$

Let $R = $ ring of entire functions on $\C$ (power series which converge everywhere on $\C$). $R^* = \{\text{function in $R$ with no zeros}\} = \{e^{f(x)}: f(x) \in \R\}.$

An old but excellent result.
\begin{theorem}(Borel, 1893)
If $f_1, \ldots, f_n \in R^*$ satisfy $f_1 + \ldots + f_n = 0$ but no (non-empty) proper subset of $\{f_1, \ldots, f_n\}$ sums to 0. then $f_i / f_j \in \C^*, \forall i, j.$  
\end{theorem}




\end{document}
