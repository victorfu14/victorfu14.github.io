\documentclass{article}
\usepackage{amsmath,amssymb,amsthm,textcomp,gensymb,nccmath}
\usepackage{mathtools}
\renewcommand{\qedsymbol}{$\blacksquare$}

\setlength{\topmargin}{0.5in}
\usepackage[margin=4cm]{geometry}
\usepackage{enumerate}

\usepackage{setspace}
\onehalfspacing
\usepackage{parskip}
\setlength{\parskip}{0.5em}
\usepackage[T1]{fontenc}
\usepackage{palatino}

% useful characters/operators
\newcommand{\R}{\mathbb{R}}
\newcommand{\C}{\mathbb{C}}
\newcommand{\Z}{\mathbb{Z}}
\newcommand{\Q}{\mathbb{Q}}
\newcommand{\N}{\mathbb{N}}
\newcommand{\F}{\mathbb{F}}
\newcommand{\matP}{\mathbb{P}}
\newcommand{\matS}{\mathbb{S}}
\newcommand{\matH}{\mathbb{H}}
\newcommand{\matT}{\mathbb{T}}
\newcommand{\st}{\ s.t.\ }
\newcommand{\ie}{\ i.e.\ }
\newcommand{\eg}{\ e.g.\ }
\def \diam {\operatorname{diam}}
\def \Hom {\operatorname{Hom}}
\def \id {\operatorname{id}}
\def \tr {\operatorname{tr}}
\def \dist {\operatorname{dist}}
\def \intr {\operatorname{int}}
\def \sgn {\operatorname{sgn}}
\def \im {\operatorname{Im}}
\def \re {\operatorname{Re}}
\def \curl {\operatorname{curl}}
\def \divg {\operatorname{div}}
\def \GL {\operatorname{GL}}
\def \End {\operatorname{End}}
\def \Aut {\operatorname{Aut}}
\def \Fr {\operatorname{Frac}}
\def \cont {\operatorname{cont}}
\def \Gal {\operatorname{Gal}}
\def \minpol {\operatorname{minpol}}
\newcommand{\pdr}[2]{\dfrac{\partial #1}{\partial #2}}
\newcommand{\dr}[2]{\dfrac{\text{d} #1}{\text{d} #2}}
\newcommand{\df}{\text{d}}
\newcommand{\inner}[2]{\left\langle #1, #2\right\rangle}

% arrows and :=, =:
\makeatletter
\providecommand*{\twoheadrightarrowfill@}{%
  \arrowfill@\relbar\relbar\twoheadrightarrow
}
\providecommand*{\twoheadleftarrowfill@}{%
  \arrowfill@\twoheadleftarrow\relbar\relbar
}
\providecommand*{\xtwoheadrightarrow}[2][]{%
  \ext@arrow 0579\twoheadrightarrowfill@{#1}{#2}%
}
\providecommand*{\xtwoheadleftarrow}[2][]{%
  \ext@arrow 5097\twoheadleftarrowfill@{#1}{#2}%
}
\makeatother
\newcommand{\defeq}{\vcentcolon=}
\newcommand{\eqdef}{=\mathrel{\mathop:}}

% integral for measure theory
\newcommand{\lowerint}{\underline{\int_{\R^d}}}
\newcommand{\upperint}{\overline{\int_{\R^d}}}
\newcommand{\lint}[1]{\underline{\int_{\R^d}} #1 (x)dx}
\newcommand{\uint}[1]{\overline{\int_{\R^d}} #1 (x)dx}
\newcommand{\sint}[1]{\simp{\int_{\R^d} #1 (x)dx}}
\newcommand{\lesint}[1]{\int_{\R^d} #1 (x)dx}

% note taking
\newcommand{\fancyem}[1]{\underline{\textsc{#1}}}

% theorem style
\newtheorem*{theorem}{Theorem}
\newtheorem*{corollary}{Corollary}
\newtheorem*{lemma}{Lemma}
\newtheorem*{conjecture}{Conjecture}
\newtheorem*{proposition}{Proposition}

\theoremstyle{definition}
\newtheorem*{definition}{Definition}
\newtheorem*{example}{Example}
\theoremstyle{remark}
\newtheorem*{remark}{Remark}

% for clearer reference
\usepackage{hyperref}
\newcommand{\corollaryautorefname}{Corollary}
\newcommand{\lemmaautorefname}{Lemma}
\newcommand{\definitionautorefname}{Definition}
\newcommand{\exampleautorefname}{Example}
\newcommand{\conjectureautorefname}{Conjecture}
\renewcommand{\subsectionautorefname}{Section}

% other styling
\usepackage{fancyvrb, fancyhdr}
\usepackage{tikz}
\usepackage{tcolorbox}

\pagestyle{fancy}
\fancyhead[LO,L]{Apr 5, 2022}
\fancyhead[RO,R]{Yiwei Fu}
\fancyhead[C]{MATH 494}
\fancyfoot[CO,C]{\thepage}

\usepackage{tikz-cd}

\usepackage{epigraph}

% \epigraphsize{\small}% Default
\setlength\epigraphwidth{8cm}
\setlength\epigraphrule{0pt}

\usepackage{etoolbox}

\makeatletter
\patchcmd{\epigraph}{\@epitext{#1}}{\itshape\@epitext{#1}}{}{}
\makeatother


\begin{document}
% \renewcommand{\ref}[1]{\autoref{#1}}
% \title{Math 494}
% \author{Yiwei Fu}
% \date{Feb 03, 2022}
% \maketitle

Recall: $L/K$ is Galois if $|\Aut_K(L)| = [L : K]$. $L/K$ is separable if $|\Hom_K(L, M)| = [L: K]$ for some field $M \supseteq K$.

$L/K$ is Galois $\implies L/K$ is separable.

$L/K$ is Galois $\iff L = $ splitting field over $K$ of some separable $f(X) \in K[X] \iff L/K$ is separable and $\Hom_K(L, M) = \Aut_K(L), \forall M \supseteq L$.

If $L/K$ is separable, let $N$ be the Galois closure of $L/K$.

Define $G \defeq \Gal(N/K), H \defeq \Gal(N/L)$. Then, fields between $L$ and $K$ correspond to groups between $G$ and $H$.

Given a separable extension $L/K$, we can write $L = L_n - L_{n-1} - \ldots - L_1 - K = L_0$ where there is no field between $L_i$ and $L_{i - 1}$. This is a powerful approach enabling one to study arbitrary $L/K$ be induction, where the induction step addresses a \emph{minimal extension}.

useful because: Galois groups (closures) of minimal separable extensions are massively restricted. Define such a Galois group to be a primitive permutation group.

Facts: If $G$ is a primitive subgroup of $S_n$, then either \begin{itemize}
  \item $L \times L \times \ldots \times L \leq G \leq \Aut(L^k) = \Aut(L)^k \rtimes S_k$
  \item $n = p^k$, $p$ prime, $(C_p)^k \leq G \leq \operatorname{AGL}_k(\F_p) = (\F_p)^k \rtimes \GL_k(\F_p)$ in usual action on $(\F_p)^k$.
\end{itemize} 

Also: for $100\%$ of positive integers $n$, the only primitive subgroups of $S_n$ are $A_n$ and $S_n$.

Also: if $n$ is prime then every transitive subgroup of $S_n$ is: \begin{itemize}
  \item $S_n$ or $A_n$
  \item groups between $\F_n$ and $\operatorname{AGL}_1(\F_n)$.
  \item if $n = \frac{q^k-1}{q-1}$ with $k \geq 2$ and $q$ prime, then $\operatorname{PGL}_k(\F_q) \leq G \leq \operatorname{P \Gamma L}_k(\F_q)$ acting on $P^{k-1}(\F_q)$.
  \item $n = 23$, $M_{23}$ "Mathieu sporadic group"
  \item $n = 11$, $M_{11}$ and $\operatorname{PSL}_2(\F_n)$.
\end{itemize}

Solvability by radicals:

Given $f(X) \in \Q[X]$, when can all roots of $f(X)$ be expressed in terms of nested radicals e.g. $\sqrt[3]{57\sqrt{31} - 1000\sqrt[5]{21+\sqrt{3}}}$

Concretely: an element $\alpha \in \C$ is expressible in terms of nested radicals iff $\alpha \in K_n$ for some field $K_n \st K_n \supseteq K_{n-1} \supseteq \ldots \supseteq K_{0} = \Q$ where $K_i = K_{i-1}(\alpha_i)$ with $d_i \in K_{i-1}$ for some positive integer $d_i$.

\begin{theorem}
  For any separable $f(X) \in \Q[X]$, $f(x)$ is "solvable by radicals" meaning that all its complex roots are expressible as above if and only if the Galois group $G$ of $f(X)$ over $\Q$ is "solvable", i.e. $\exists G \triangleright G_1 \triangleright G_2 \triangleright \ldots \triangleright G_k = 1$ where $G_{i-1}$ is normal in $G$, and $G_i/G_{i-1}$ is cyclic of prime order.
\end{theorem}

\begin{corollary}
  All polynomials in $\Q[X]$ of degree $\leq 4$ are solvable by radicals, but $\forall n \geq 5$, $\exists$ degree-$n$ irreducible $f(x) \in \Q[x]$ which are NOT solvable (since $\exists$ polynomials with groups $S_n$, which is not solvable when $n \geq 5$)
\end{corollary}

Key lemma \begin{lemma}
  If a field $K$ contains $n$ $n$-th roots of unity, and $L/K$ is Galois with $\Gal(L/K) \cong C_n$, then $L = K(\alpha)$ where $\alpha^n \in K$.
\end{lemma}

Converse is easy: if $K$ contains $n$-th roots of unity $\zeta$ and $L = K(\alpha)$ where $\minpol_K(\alpha) = x^n - c$, then $L/K$ is Galois and $\Gal(L/K) \cong C_n$.

For: the roots of $x^n - c$ are $\alpha \phi^i$, $0 \leq i \leq n - 1$, which are all in $K(\alpha) = L$. SO $L = $ splitting field of $x^n - c$ over $K \implies L/K$ is Galois of degree $n$, $\Gal(L/K) = \{\sigma_i i\alpha \mapsto \alpha \phi^i, i \in \Z/n\Z\} \implies \Gal(L/K) = \langle\sigma_1\rangle \cong C_n$
\end{document}
