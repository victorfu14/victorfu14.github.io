\documentclass{article}
\usepackage{amsmath,amssymb,amsthm,textcomp,gensymb,nccmath}
\usepackage{mathtools}
\renewcommand{\qedsymbol}{$\blacksquare$}

\setlength{\topmargin}{0.5in}
\usepackage[margin=4cm]{geometry}
\usepackage{enumerate}

\usepackage{setspace}
\onehalfspacing
\usepackage{parskip}
\setlength{\parskip}{0.5em}
\usepackage[T1]{fontenc}
\usepackage{palatino}

% useful characters/operators
\newcommand{\R}{\mathbb{R}}
\newcommand{\C}{\mathbb{C}}
\newcommand{\Z}{\mathbb{Z}}
\newcommand{\Q}{\mathbb{Q}}
\newcommand{\N}{\mathbb{N}}
\newcommand{\matP}{\mathbb{P}}
\newcommand{\matS}{\mathbb{S}}
\newcommand{\matH}{\mathbb{H}}
\newcommand{\matT}{\mathbb{T}}
\newcommand{\st}{\ s.t.\ }
\newcommand{\ie}{\ i.e.\ }
\newcommand{\eg}{\ e.g.\ }
\def \diam {\operatorname{diam}}
\def \Hom {\operatorname{Hom}}
\def \id {\operatorname{id}}
\def \tr {\operatorname{tr}}
\def \dist {\operatorname{dist}}
\def \intr {\operatorname{int}}
\def \sgn {\operatorname{sgn}}
\def \im {\operatorname{Im}}
\def \re {\operatorname{Re}}
\def \curl {\operatorname{curl}}
\def \divg {\operatorname{div}}
\def \GL {\operatorname{GL}}
\def \End {\operatorname{End}}
\def \Aut {\operatorname{Aut}}
\newcommand{\pdr}[2]{\dfrac{\partial #1}{\partial #2}}
\newcommand{\dr}[2]{\dfrac{\text{d} #1}{\text{d} #2}}
\newcommand{\df}{\text{d}}
\newcommand{\inner}[2]{\left\langle #1, #2\right\rangle}

% arrows and :=, =:
\makeatletter
\providecommand*{\twoheadrightarrowfill@}{%
  \arrowfill@\relbar\relbar\twoheadrightarrow
}
\providecommand*{\twoheadleftarrowfill@}{%
  \arrowfill@\twoheadleftarrow\relbar\relbar
}
\providecommand*{\xtwoheadrightarrow}[2][]{%
  \ext@arrow 0579\twoheadrightarrowfill@{#1}{#2}%
}
\providecommand*{\xtwoheadleftarrow}[2][]{%
  \ext@arrow 5097\twoheadleftarrowfill@{#1}{#2}%
}
\makeatother
\newcommand{\defeq}{\vcentcolon=}
\newcommand{\eqdef}{=\mathrel{\mathop:}}

% integral for measure theory
\newcommand{\lowerint}{\underline{\int_{\R^d}}}
\newcommand{\upperint}{\overline{\int_{\R^d}}}
\newcommand{\lint}[1]{\underline{\int_{\R^d}} #1 (x)dx}
\newcommand{\uint}[1]{\overline{\int_{\R^d}} #1 (x)dx}
\newcommand{\sint}[1]{\simp{\int_{\R^d} #1 (x)dx}}
\newcommand{\lesint}[1]{\int_{\R^d} #1 (x)dx}

% note taking
\newcommand{\fancyem}[1]{\underline{\textsc{#1}}}

% theorem style
\newtheorem*{theorem}{Theorem}
\newtheorem*{corollary}{Corollary}
\newtheorem*{lemma}{Lemma}
\newtheorem*{conjecture}{Conjecture}
\newtheorem*{proposition}{Proposition}

\theoremstyle{definition}
\newtheorem*{definition}{Definition}
\newtheorem*{example}{Example}
\theoremstyle{remark}
\newtheorem*{remark}{Remark}

% for clearer reference
\usepackage{hyperref}
\newcommand{\corollaryautorefname}{Corollary}
\newcommand{\lemmaautorefname}{Lemma}
\newcommand{\definitionautorefname}{Definition}
\newcommand{\exampleautorefname}{Example}
\newcommand{\conjectureautorefname}{Conjecture}
\renewcommand{\subsectionautorefname}{Section}

% other styling
\usepackage{fancyvrb, fancyhdr}
\usepackage{tikz}
\usepackage{tcolorbox}

\pagestyle{fancy}
\fancyhead[LO,L]{Feb 03, 2022}
\fancyhead[RO,R]{Yiwei Fu}
\fancyhead[C]{MATH 494}
\fancyfoot[CO,C]{\thepage}

\usepackage{tikz-cd}

\usepackage{epigraph}

% \epigraphsize{\small}% Default
\setlength\epigraphwidth{8cm}
\setlength\epigraphrule{0pt}

\usepackage{etoolbox}

\makeatletter
\patchcmd{\epigraph}{\@epitext{#1}}{\itshape\@epitext{#1}}{}{}
\makeatother


\begin{document}
% \renewcommand{\ref}[1]{\autoref{#1}}
% \title{Math 494}
% \author{Yiwei Fu}
% \date{Feb 03, 2022}
% \maketitle

\epigraph{Mathematicians are brilliant people. Especially before they have so many fancy tools, all they have is brilliance.}{--- \textup{Micheal Zieve}}
\bigskip

\begin{lemma}
    If $R$ is Noetherian (i.e. an integral domain in which every ideal is finitely generated) then every non-zero non-unit in $R$ is product of irreducible elements.
\end{lemma}
In general, it is easier for elements to be irreducible than prime.

\begin{proof}
    Suppose otherwise. Then $\exists x \in R$ non-zero non-unit, not a product of irreducible elements $\implies x$ is reducible, say $x = yz$. At least on of $y$ or $z$ is neither a unit nor a product of irreducibles.

    Hence $x = x_1y_1$, where $x_1$ is not unit or product of irreducibles, $y_1$ is not a unit.
    Likewise $x_1 = x_2y_2$ where $x_2$ is a not unit or product of irreducibles with $y_2$ is not a unit. $x_n = x_{n+1}y_{n+1}$.
    \[
        (x) \subsetneq (x_1) \subsetneq (x_2) \subsetneq 
        \ldots
    \]
    $\bigcup_{n \geq 1}(x_n)$ is an ideal of $R$ which doesn't contain $1$. It is finitely generated $\implies$ all generator is in $x_n$ for some finite $n \implies (x_n) = (x_{n+1})$, a contradiction. 
\end{proof}

Last time:
$R = $ PID $\implies$ all irreducible elements in $R$ are prime $\implies$ every element or $R$ has $\leq 1$ factorization into irreducible elements (up to permutation).

On the other hand, $R = $ PID $\implies R$ is Noetherian $\implies$ every nonzero element of $R$ has a factorization into irreducible elements.

These two together shows that $R$ is UFD.



\begin{lemma}
    If $R$ is a Euclidean integral domain (i.e. $\exists \phi: R \to \{-\infty\} \cup \Z_{\geq 0} \st \forall a, b \in R$ with $b \neq 0$, $\exists q, r \in R \st a = bq + r$ where $\phi(R) < \phi(b)$) then $R$ is PID.
\end{lemma}
\begin{proof}
    If $I$ is a nonzero ideal of $R$, then $\phi(I) \subset \{-\infty\} \cup \Z_{\geq 0}$. So $\phi(I \setminus \{0\})$ has a smallest element $\phi(b), b \in I, b \neq 0$. Then $I = (b)$, since $(b) \subseteq I$ and also $I \subseteq (b)$ because $a \in I \implies a = bq + r, q, r \in R, \phi(r) < \phi(b)$.

    But $a, b \in I, a = bq + r \implies r \in I$. So the minimalist of $b$ of $\phi(b)$ implies $r = 0$, so $b \mid a \implies a \in (b)$.
\end{proof}
\begin{example}
    $\Z[i]$ Euclidean $\implies$ PID $\implies$ UFD.

    $\phi(a + b \sqrt{3}) \defeq a^2 + 3b^2$ is NOT a Euclidean function on $\Z[\sqrt{-3}]$, since you can't divide $1 + \sqrt{-3}$ by $2$ to get a smaller remainder. 
    
    Moreover, $\Z[\sqrt{-3}]$ is not Euclidean, since it is not a UFD: $(1 + \sqrt{-3})(1 - \sqrt{-3}) = 4 = 2 \cdot 2$, all irreducible and they are not unit multiples of each other. 
\end{example}
But $\Z\left[\frac{1 + \sqrt{-3}}{2}\right]$ is Euclidean with $\phi$ as Euclidean function.

$\Z[i]$ Euclidean $\implies$ PID $\implies$ UFD. What are the primes in $\Z[i]$?

Define the "norm" $N: \Z[i] \to \Z, a + bi \mapsto a^2+b^2 = |a + bi|^2$. Then $N(xy) = N(x)N(y)$ and $N(x) = x \bar{x}$ where $\overline{a + bi} = a - bi$.

\begin{lemma}
    $N(x) \geq 0$, $N(x) = 0 \iff x = 0$, $N(x) = 1 \iff x = \pm 1$ or $\pm i$, $N(x) = 1 \iff x$ is a unit in $\Z[i]$.
\end{lemma}
\begin{proof}
    The first $3$ statements are easy. 
    If $N(x) = 1$ then $x\bar{x} = 1 \implies x = $ unit. If $x = $ unit then $xy = 1, y \in \Z[i] \implies N(xy) = N(x)N(y) = N(1) = 1 \implies N(x) = 1$.
\end{proof}
\begin{corollary}
    If $x \in \Z[i]$ and $N(x)$ is prime in $\Z$ then $x$ is irreducible in $\Z[i]$.
\end{corollary}
But there are other irreducibles in $\Z[i]$ too. Given $x \in R$ non-zero non-unit, then $N(x) \in \Z_{\geq 2}$. If $x$ is irreducible then $\bar{x}$ is also irreducible (since complex conjugation is a homomorphism) so $N(x)$ is a product of two irreducibles in $\Z[i]$. But we can write $N(x) = p_1p_2\ldots p_k$ where $p_i$ is prime numbers in $\Z$ and then write each $p_i$ as product of irreducibles in $\Z[i]$, so either $k = 1$ and $p_1 = $ product of two irreducibles in $\Z[i]$ or $k = 2$ and $p_1, p_2$ are two irreducibles in $\Z[i]$ where $x = up_1, \bar{x} = p_2v$, $u, v$ units $\implies p_1 = p_2$.

Remains to show for $p \in \Z$ prime, $p$ is irreducible in $\Z[i] \iff p \equiv 3 \pmod{4}$.



\end{document}
