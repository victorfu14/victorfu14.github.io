\documentclass{article}
\usepackage{amsmath,amssymb,amsthm,textcomp,gensymb}
\usepackage{mathtools}
\renewcommand{\qedsymbol}{$\blacksquare$}

\setlength{\topmargin}{0.5in}
\usepackage[margin=4cm]{geometry}
\usepackage{enumerate}

\usepackage{setspace}
\onehalfspacing
\usepackage{parskip}
\setlength{\parskip}{0.5em}
\usepackage[T1]{fontenc}
\usepackage{palatino}

% useful characters/operators
\newcommand{\R}{\mathbb{R}}
\newcommand{\C}{\mathbb{C}}
\newcommand{\Z}{\mathbb{Z}}
\newcommand{\Q}{\mathbb{Q}}
\newcommand{\N}{\mathbb{N}}
\newcommand{\matP}{\mathbb{P}}
\newcommand{\matS}{\mathbb{S}}
\newcommand{\matH}{\mathbb{H}}
\newcommand{\matT}{\mathbb{T}}
\newcommand{\st}{\ s.t.\ }
\newcommand{\ie}{\ i.e.\ }
\newcommand{\eg}{\ e.g.\ }
\def \diam {\operatorname{diam}}
\def \Hom {\operatorname{Hom}}
\def \id {\operatorname{id}}
\def \tr {\operatorname{tr}}
\def \dist {\operatorname{dist}}
\def \intr {\operatorname{int}}
\def \sgn {\operatorname{sgn}}
\def \im {\operatorname{Im}}
\def \re {\operatorname{Re}}
\def \curl {\operatorname{curl}}
\def \divg {\operatorname{div}}
\def \GL {\operatorname{GL}}
\def \End {\operatorname{End}}
\def \Aut {\operatorname{Aut}}
\newcommand{\pdr}[2]{\dfrac{\partial #1}{\partial #2}}
\newcommand{\dr}[2]{\dfrac{\text{d} #1}{\text{d} #2}}
\newcommand{\df}{\text{d}}
\newcommand{\inner}[2]{\left\langle #1, #2\right\rangle}

% arrows and :=, =:
\makeatletter
\providecommand*{\twoheadrightarrowfill@}{%
  \arrowfill@\relbar\relbar\twoheadrightarrow
}
\providecommand*{\twoheadleftarrowfill@}{%
  \arrowfill@\twoheadleftarrow\relbar\relbar
}
\providecommand*{\xtwoheadrightarrow}[2][]{%
  \ext@arrow 0579\twoheadrightarrowfill@{#1}{#2}%
}
\providecommand*{\xtwoheadleftarrow}[2][]{%
  \ext@arrow 5097\twoheadleftarrowfill@{#1}{#2}%
}
\makeatother

\newcommand{\defeq}{\vcentcolon=}
\newcommand{\eqdef}{=\mathrel{\mathop:}}

% integral for measure theory
\newcommand{\lowerint}{\underline{\int_{\R^d}}}
\newcommand{\upperint}{\overline{\int_{\R^d}}}
\newcommand{\lint}[1]{\underline{\int_{\R^d}} #1 (x)dx}
\newcommand{\uint}[1]{\overline{\int_{\R^d}} #1 (x)dx}
\newcommand{\sint}[1]{\simp{\int_{\R^d} #1 (x)dx}}
\newcommand{\lesint}[1]{\int_{\R^d} #1 (x)dx}

% note taking
\newcommand{\fancyem}[1]{\underline{\textsc{#1}}}

% theorem style
\newtheorem{theorem}{Theorem}[section]
\newtheorem{corollary}{Corollary}[theorem]
\newtheorem{lemma}{Lemma}[section]
\newtheorem{conjecture}{Conjecture}[section]
\newtheorem{proposition}{Proposition}[section]

\theoremstyle{definition}
\newtheorem{definition}{Definition}[section]
\newtheorem{example}{Example}[section]
\theoremstyle{remark}
\newtheorem*{remark}{Remark}

% for clearer reference
\usepackage{hyperref}
\newcommand{\corollaryautorefname}{Corollary}
\newcommand{\lemmaautorefname}{Lemma}
\newcommand{\definitionautorefname}{Definition}
\newcommand{\exampleautorefname}{Example}
\newcommand{\conjectureautorefname}{Conjecture}
\renewcommand{\subsectionautorefname}{Section}

% other styling
\usepackage{fancyvrb, fancyhdr}
\usepackage{tikz}
\usepackage{tcolorbox}

\usepackage{tikz-cd}

\begin{document}
\renewcommand{\ref}[1]{\autoref{#1}}
\title{Math 494}
\author{Yiwei Fu}
\date{Jan 13, 2022}
\maketitle

\section{}
\fancyem{Fact:} If $G$ is a finitely generated subgroup of $\C^*$, then $\forall n \in \N$, the equation $x_1 + x_2 + \ldots + x_n = 1$ has only finitely many solutions with $x_1, x_2, \ldots, x_n \in G,$ in which no nonempty subset of $x_i$'s sums to $0$.

Extra credit: does $\exists$ such a $G$ for which $\exists$ solutions as above for infinitely many $n$?

\begin{example}
$0$ ring: $0 = 1.$

$0$ homomorphism: for any ring $R, f: R \to ``\text{zero ring}", r \mapsto 0.$
\end{example}

\begin{definition}
If $f: R \to S$ is a ring homomorphism, then the kernel of $f$ is \[\ker{f} = \{r \in R: f(r) = 0\}.\]
\end{definition}
We know $\ker{f}$ is a subgroup of $R$ under $+.$ Also: if $r \in \ker{f}$ and $r' \in R$ then $rr' \in \ker{f}$ since $f(rr') = f(r)f(r') = 0 \cdot f(r') = 0.$ 

\begin{definition}
Suppose $R$ is a ring. An \underline{ideal} of $R$ is a subgroup of $(R, +)$ which is closed under multiplication by $R.$
\end{definition}
Ideals are great.

\textbf{\emph{From NOW ON: ALL rings are commutative.}}

\begin{example}
Ideals in $\Z$: $n\Z, (n \in \Z_{\geq 0})$
\end{example}
\fancyem{Note}: A nonempty subset of $R$ is an ideal $\iff \forall n \geq 0, \forall r_1, \ldots, r_n \in R, i_1, i_2, \ldots, i_n \in I, r_1i_1 + r_2i_2 + \ldots + r_ni_n \in I.$

\begin{definition}
For $r \in R$ the \underline{principal ideal} $(r)$ (also denoted as $rR$) is $\{rr': r' \in R\}.$
\end{definition}
Unit ideal of $R$ is $(1) = 1R = R.$
Zero ideal of $R$ is $(0) = \{0\}.$

A "proper ideal" of $R$ is an ideal which is not $(0)$ or $(1).$

\fancyem{Note}: If $f: R \to S$ is a homomorphism then $\ker{f}$ is an ideal of $R$.
$\ker{f} = (1) \iff S = ``0 \text{ ring}", \ker{f} = (0) \iff f$ is injective.

Suppose $R$ is a ring and $I$ is an ideal. Then $R/I$ is a group under addition.
\begin{proposition}
$R/I$ is a ring.
\end{proposition}
\begin{proof}
Define $(r + I)(r' + I) \defeq rr' + I.$ Note that if $i, i' \in I$ then $(r + i)(r' + i') = rr' + ri' + ir' + ii' \in rr' + I.$

Rest is easy.
\end{proof}

\begin{example}
$R = \Z, I = 3\Z, R/I = \Z/3\Z.$

In general, $\Z/n\Z$ is the quotient of the ring $R$ by the ideal $n\Z.$
\end{example}

\begin{definition}
A \underline{field} is a nonzero ring in which every nonzero element has a multiplicative inverse.
\end{definition}

\begin{example}
$\Q, \C, \R, \Z/p\Z$ where $p$ is prime.
\end{example}
\fancyem{Non-examples:} $\Z, \Z/4\Z.$

\begin{definition}
An \underline{integral domain} is a nonzero ring $R$ with no zero divisors $(a, b \in R, ab = 0 \implies a = 0 \text{ or } b = 0).$
\end{definition}
If $R$ is a field, what are the ideals of $R$?

Only $(0)$ and $(1).$ since if an ideal $I$ contains a nonzero $r \in R$, then $I \ni rr^{-1} = 1 \implies I = (1).$

\begin{proposition}
If $f: R \to S$ is a ring homomorphism andd $R$ is a field. then either $f$ is injective or $S=``0 \text{ ring}".$
\end{proposition}

Notation: often $R = $ ring, $I = $ ideal. for $r \in R$ we denote the element $r + I$ of $R/I$ by $\overline{r}.$

\begin{theorem}
$f: R \to S$ is a ring homomorphism with kernel $K$. Let $I$ be an ideal of $R.$ Let $\pi: R \to R/I$ be the quotient map.
\begin{enumerate}
\item
	If $I \subseteq K$ then $\exists$ a unique homomorphism $\overline{f}: R/I \to S \st \overline{f} \circ \pi = f.$
	\[\begin{tikzcd}
	R \arrow[r, "f"] & S  \\
	& R/I \arrow[ul, leftarrow, "\pi"] \arrow[u, dashrightarrow, "\overline{f}"]
	\end{tikzcd}\]
\item
	If $I = K$ and $f$ is surjective then $\overline{f}$ is $\cong.$
\end{enumerate}
\end{theorem}

\begin{theorem}(Correspondence Theorem)
$f: R \to S$ is a surjective ring homomorphism with kernel $K.$ Then the maps
$I \mapsto f(I)$ and $J \mapsto f^{-1}(J)$ are inverse bijections
$\{$ideals of $R$ containing $K\}$ $\{$ideals of $S\}.$ 
\end{theorem}
\begin{proof}
We know that these maps induce bijections between subgroups of $(R, +)$ containing $K$ and subgroups between $(S, +).$
Check:
\begin{enumerate}
\item
$I =$ ideal of $R$ containing $K \implies f(I) =$ ideal of $S$ since every $s \in S$ is $f(r), r \in R,$ so $i \in I \implies s \cdot f(i) = f(r)f(i) = f(ri) \in f(I).$
\item
$J =$ ideal of $S$ then (from group result) $f^{-1}(J)$ is a subgroup of $(R, +)$ which contains $K,$ and $f^{-1}(J)$ is an ideal since $r \in J, i \in f^{-1}(J) \implies f(ri) = f(r)f(i) \in SJ = J.$ \qedhere
\end{enumerate}
\end{proof}

\fancyem{Supplement:} same notation, $R/I \cong S/f(I).$

\end{document}
