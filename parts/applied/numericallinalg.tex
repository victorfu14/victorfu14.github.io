\documentclass{report}
\usepackage{amsmath,amssymb,amsthm,textcomp,gensymb,nccmath}
\usepackage{mathtools}
\renewcommand{\qedsymbol}{$\blacksquare$}

\setlength{\topmargin}{0.5in}
\usepackage[margin=4cm]{geometry}
\usepackage{enumerate}

\usepackage{setspace}
\onehalfspacing
\usepackage{parskip}
\setlength{\parskip}{0.5em}
\usepackage[T1]{fontenc}
\usepackage{palatino}

% useful characters/operators
\newcommand{\R}{\mathbb{R}}
\newcommand{\C}{\mathbb{C}}
\newcommand{\Z}{\mathbb{Z}}
\newcommand{\Q}{\mathbb{Q}}
\newcommand{\N}{\mathbb{N}}
\newcommand{\F}{\mathbb{F}}
\newcommand{\B}{\mathbb{B}}
\newcommand{\matP}{\mathbb{P}}
\newcommand{\matE}{\mathbb{E}}
\newcommand{\matS}{\mathbb{S}}
\newcommand{\matH}{\mathbb{H}}
\newcommand{\matT}{\mathbb{T}}
\newcommand{\st}{\ s.t.\ }
\newcommand{\ie}{\ i.e.\ }
\newcommand{\eg}{\ e.g.\ }
\def \diam {\operatorname{diam}}
\def \Hom {\operatorname{Hom}}
\def \id {\operatorname{id}}
\def \tr {\operatorname{tr}}
\def \rk {\operatorname{rk}}
\def \sp {\operatorname{span}}
\def \dist {\operatorname{dist}}
\def \intr {\operatorname{int}}
\def \sgn {\operatorname{sgn}}
\def \im {\operatorname{Im}}
\def \re {\operatorname{Re}}
\def \curl {\operatorname{curl}}
\def \divg {\operatorname{div}}
\def \GL {\operatorname{GL}}
\def \Aut {\operatorname{Aut}}
\def \per {\operatorname{per}}
\def \LE {\operatorname{LE}}
\def \indeg {\operatorname{indeg}}
\def \outdeg {\operatorname{outdeg}}
\def \Par {\operatorname{Par}}
\def \Gr {\operatorname{Gr}}
\def \del {\operatorname{del}}
\def \add {\operatorname{add}}
\def \rank {\operatorname{rank}}
\newcommand{\pdr}[2]{\dfrac{\partial #1}{\partial #2}}
\newcommand{\df}{\mathrm{d}}
\newcommand{\dr}[2]{\dfrac{\df #1}{\df #2}}
\newcommand{\inner}[2]{\left\langle #1, #2\right\rangle}
\newcommand{\qbin}[2]{\begin{bmatrix}{#1}\\ {#2}\end{bmatrix}_q}
\newcommand{\gen}[1]{\left\langle #1\right\rangle}
\newcommand{\norm}[1]{\left\| #1 \right\|}


% arrows and :=, =:
\makeatletter
\providecommand*{\twoheadrightarrowfill@}{%
  \arrowfill@\relbar\relbar\twoheadrightarrow
}
\providecommand*{\twoheadleftarrowfill@}{%
  \arrowfill@\twoheadleftarrow\relbar\relbar
}
\providecommand*{\xtwoheadrightarrow}[2][]{%
  \ext@arrow 0579\twoheadrightarrowfill@{#1}{#2}%
}
\providecommand*{\xtwoheadleftarrow}[2][]{%
  \ext@arrow 5097\twoheadleftarrowfill@{#1}{#2}%
}
\makeatother

\newcommand{\defeq}{\vcentcolon=}
\newcommand{\eqdef}{=\mathrel{\mathop:}}

% integral for measure theory
\newcommand{\lowerint}{\underline{\int_{\R^d}}}
\newcommand{\upperint}{\overline{\int_{\R^d}}}
\newcommand{\lint}[1]{\underline{\int_{\R^d}} #1 (x)dx}
\newcommand{\uint}[1]{\overline{\int_{\R^d}} #1 (x)dx}
\newcommand{\sint}[1]{\simp{\int_{\R^d} #1 (x)dx}}
\newcommand{\lesint}[1]{\int_{\R^d} #1 (x)dx}

% note taking
\newcommand{\fancyem}[1]{\underline{\textsc{#1}}}

% theorem style
\newtheorem{theorem}{Theorem}[section]
\newtheorem{corollary}{Corollary}[section]
\newtheorem{lemma}{Lemma}[section]
\newtheorem{conjecture}{Conjecture}[section]
\newtheorem{proposition}{Proposition}[section]

\theoremstyle{definition}
\newtheorem{definition}{Definition}[section]
\newtheorem{example}{Example}[section]
\theoremstyle{remark}
\newtheorem*{remark}{Remark}

% pseudocode and algorithms
\usepackage{algorithm}
\usepackage{algpseudocode}
\usepackage{algorithmicx}
\counterwithin{algorithm}{section}
\renewcommand{\algorithmicrequire}{\textbf{Input:}}
\renewcommand{\algorithmicensure}{\textbf{Output:}}

% for clearer reference
\usepackage{hyperref}
\newcommand{\corollaryautorefname}{Corollary}
\newcommand{\lemmaautorefname}{Lemma}
\newcommand{\definitionautorefname}{Definition}
\newcommand{\exampleautorefname}{Example}
\newcommand{\conjectureautorefname}{Conjecture}
\renewcommand{\subsectionautorefname}{Section}
\newcommand{\algorithmautorefname}{Algorithm}

% other styling
\usepackage{fancyvrb, fancyhdr}
\usepackage{tikz-cd}
\usepackage{tikz}
\PassOptionsToPackage{usenames, x11names}{xcolor}
\usepackage{tcolorbox}
\selectcolormodel{cmy}

\newcommand{\edge}{
    \begin{tikzcd}[cramped, sep=small, labels={font=\everymath\expandafter{\the\everymath\textstyle}}]
        u \arrow[r, "e", no head] & v
    \end{tikzcd}
}


\pagestyle{fancy}
\fancyhead[LO,L]{\leftmark}
\fancyhead[RO,R]{Yiwei Fu}
% \fancyhead[C]{MATH 566}
\fancyfoot[CO,C]{\thepage}
\renewcommand{\sectionmark}[1]{\markboth{#1}{#1}}

\numberwithin{equation}{section}

\newcommand{\fnl}{\parbox[t]{0\linewidth}{}}
\newcommand*\ttlmath[2]{\texorpdfstring{$\boldsymbol{#1}$}{#2}}

\usepackage{epigraph}

% \epigraphsize{\small}% Default
\setlength\epigraphwidth{8cm}
\setlength\epigraphrule{0pt}

\usepackage{etoolbox}

\makeatletter
\patchcmd{\epigraph}{\@epitext{#1}}{\itshape\@epitext{#1}}{}{}
\makeatother

% combinatorics special
\usepackage{pgfopts}
\usepackage{ytableau}

\begin{document}
\title{Notes for Math 571 -- Numerical Linear Algebra}
\author{Yiwei Fu, Instructor: Divakar Viswanath}
\date{FA 2022}
\maketitle


\tableofcontents


\clearpage
\pagenumbering{arabic}

\chapter{}

\section{Rank of Matrices}

\section{Cauchy-Schwarz}

\section{Projection Matrices, Gram-Schmidt Process}
Suppose we have a plane of dimension $n-1$ and a direction orthogonal to it. A unit vector $q$ with $q^Tq = 1$.

Through a $q$ a dim is specified as well as the $n-1$ dimensional plane orthogonal to it.

Given $x$ we want to decompose it as $y + z$ with \begin{enumerate}
    \item $y$ parallel to $q$
    \item $z$ orthogonal to $q$
\end{enumerate}

We note $q^Tz = 0$. So \begin{align*}
    x & = \alpha a + z \\
    q^Tz & = \alpha q^T q = \alpha \\
    x & = q (q^Tx) + z
\end{align*}
If $x = y+z$ is the decomposition of $x$ then \begin{align*}
    y & = q(q^Tx) = (qq^T)x \\
    z & = x - (qq^T)x
\end{align*}
The matrix $qq^T$ is a projection matrix. Applied to $x$ it gives the component along $q$. If $p = qq^T$ then $P_x = $ component of $x$ along $q$. $(I-P)x = $ component of $x$ along the plane orthogonal to $q$.

(All projection are orthogonal throughout the class.)

Let us try to understand $P$. \begin{enumerate}
    \item What is the rank of $P$? $\rank P = 1$. (all columns multiples of $q$)
    \item Eigenvalues and eigenvectors of $P$?
    
    $Pq = q$. Suppose $x \perp q$ then $Px = 0$. And eigenvalue $ = 1$, eigenvalue $ = 0$ with multiplicity $n-1$.
    \item $P^2 = P$. $(qq^T)(qq^T) = q(q^Tq)q^T = qq^T$. $P^2 = Px$ for all $x \implies P^2 = P$.
    \item $(I-P)^2 = I-P$.
    \item $P(I-P) = 0$.
\end{enumerate}

Given $x$ and unit vector $q$, how many operations to compute $(I - P)x = x - q(q^Tx)$?
\begin{enumerate}
    \item $q^Tx$ takes $n$ multiplications and $n-1$ additions
    \item $q(q^Tx)$ takes $n$ multiplications
    \item $x - q(q^Tx)$ takes $n$ subtractions
\end{enumerate}
In total it takes $4n - 1$ or $4n$ arithmetic operations.

Suppose $q_1$ and $q_2$ are unit vectors with $q_1 \perp q_2$. Then which matrix projects to $\inner{q_1}{q_2}$, the plane spanned by $q_1$ and $q_2$?

$P = P_1 + P_2$ with $P_1 = q_1q^T, P_2 = q_2q_2^T$. 

\begin{definition}
    $q_1, \ldots, q_k$ is called an orthonormal set of vectors if \begin{enumerate}
        \item $q_iq_i^T = 1$ for all $i$,
        \item $q_i^Tq_j = 0$ for all $i \neq q$.
    \end{enumerate}
    $Q = \begin{pmatrix}
        \vline & & \vline \\
        q_1 & \cdots & q_k \\
        \vline & & \vline
    \end{pmatrix} \in \R^{n, k}$ is a matrix with orthonormal columns.
\end{definition}

If $q_1, \ldots, q_k$ are orthonormal then \[
    P = q_1q_1^T + q_2q_2^T + \ldots + q_xq_x^T    
\] projects to $\gen{q_1, \ldots, q_k}$.

$P$ can be expressed as $P = QQ^T$ where $Q \in \R^{n, k}$ with $q_j$ as its columns.

What is the interpretation of $Q^Tx$?
$Q^Tx$ gives the coefficients when the projection of $x$ to $\gen{q_1, \ldots, q_k}$ is written as a linear combination of $q_j$.

What is $Q^TQ$? Identity matrix.
Columns of $Q$ forms as orthogonal set iff $Q^TQ = I$.

\begin{enumerate}
    \item What is $I - QQ^T$? Projection to $\gen{q_1, \ldots, q_k}^\perp = (n - k)$ dim plane orthogonal to $\gen{q_1, \ldots, q_k}$.
    \item $\rank(QQ^T) = k$.
    \item $\rank(I - QQ^T) = n - k$.
\end{enumerate}
\begin{definition}
    A matrix $Q \in \R^{n, n}$ with orthonormal columns is called an orthogonal matrix. The columns of an orthogonal matrix $Q$ form an orthogonal basis.
\end{definition}
\begin{enumerate}
    \item $QQ^T = \id$ since $QQ^Tx = x$ for all $x$ (projection to the whole space.)
    \item What is the interpretation of $Q^Tx$? The coefficients for $x$ as linear combinations of columns of $Q$.
    \item $Q^TQ$ still equal to identity.
    \item The rows of $Q$ also form an orthonormal basis.
    \item $Q^{-1} = Q^T$.
\end{enumerate}

\fancyem{Classical Gram-Schmidt Process}

Suppose $A \in \R^{n, k}$ with $n \geq k$ and $\rank A = k$. Let $a_1, \ldots, a_k$ be the columns of $A$. The Gram-Schmidt process generates an orthonormal set $q_1$ through $q_k$ such that \begin{enumerate}
    \item $\gen{q_1} = \gen{a_1}$,
    \item $\gen{q_1, q_2} = \gen{a_1, a_2}$,
    \item[\vdots]
    \item[k.] $\gen{q_1, q_2, \ldots, q_k} = \gen{a_1, a_2, \ldots, a_k}$.
\end{enumerate}

An algorithm for computing $q_1, \ldots, q_k$:
\begin{align*}
    q_1 & = \frac{a_1}{\norm{a_1}} = \frac{a_1}{(a_1^Ta_1)} \\
    \tilde{q_2} & = a_2 - P_1a_2 = a_2 - q_1(q_1^Ta_2), q_2 = \frac{\tilde{q_2}}{\norm{q_2}} \\
    & \vdots \\
    \tilde{q_k} & = a_k - \sum_{i=1}^{k-1} P_ia_k = a_k - \sum_{i=1}^{k-1} q_iq_i^Ta_k, q_k = \frac{\tilde{q_k}}{\norm{q_k}}. 
\end{align*}

Expression of Gram-Schmidt as $A = QR$ with $R$ upper triangles. Suppose $q_1, \ldots, q_k$ are computed by applying Gram-Schmidt to the columns of $A$.

Let $Q$ be the matrix whose columns are $q_1, \ldots, q_k$. Both $A$ and $Q$ are $n \times k$.
\begin{align*}
    A & = Q (k \times k \text{ matrix}) \\
    & = QR.
\end{align*}
Every column of $A$ is expressed as a linear combination of $q_1, \ldots, q_k$. 

What are the entries of $R$?

Note that $a_j = q_1(q_1^Ta_j) + \ldots + a_j(a_j^Ta_j)$ because $a_j \in \gen{q_1, \ldots, q_j}$. Write $a_j = q_1r_{1j} + \ldots + a_jr_{ij}$. Thus \[
    r_{ij}  = \begin{cases}
        q_i^Ta_j & i \leq j, \\
        0 & i > j.
    \end{cases}    
\]
During Gram=Schmidt process these coefficients are computed \begin{align*}
    r_{ij} & = q_i^Ta_j, \quad i < j \\
    r_{jj} & = \norm{\tilde{q_j}}
\end{align*}

\fancyem{Application of classical Gram-Schmidt}

Think of $a_1, \ldots, a_k$ as defining a $k-$dim parallelepiped in $\R^n$. What is the volume of the parallelepiped defined by $a_1, \ldots, a_k$? 

$\prod r_{ii}$. 

\fancyem{Note} Classical Gram-Schmidt (CGS) is not numerically stable. More precisely, when $A$ has near rank efficiency, then CGS does not behave well.

We can use modified Gram-Schmidt (MGS):
\begin{lemma}
    If $q_1, \ldots, q_j$ are an orthonormal set and $P_1, \ldots, P_j$ are corresponding projections, then \[
        I - P_1 - \ldots - P_j = (I - P_j) \ldots (I - P_2) (I - P_1)    
    \]
\end{lemma}
(Projection one at a time (RHS) vs. project at once (LHS))
\begin{proof}
    $P_iP_j = 0$ if $i \neq j$. Expand RHS and we are done.
\end{proof}

MGS has step $j$ given by $\tilde{q}_j = (I - P_{j-1})\ldots(I - P_1)a_j, q_j = \frac{\tilde{q_j}}{\norm{q_j}}$.

In CGS, $A = QR$ with $a_j = q_ir_{ij} + \ldots + q_jr_{jj}$. $r_{ij} = q_i^T a_j i\neq j, \norm{\tilde{q_j}}$.

$q_i^Ta_j$ are not available as intermediate quantities in MGS.
\begin{align*}
    \tilde{q_j} & = (I - P_{j-1})\ldots(I - P_{1})a_j \\
    a_{j, 1} & = (I - P_1)a_j \quad (= a_j - P_1a_j) \\
    a_{j, 2} & = (I - P_2)a_{j, 1} \quad (= a_j - P_2a_j - P_1a_j \text{ mathematically}) \\
    a_{j, 3} & = (I - P_3)a_{j, 2} \quad (= a_j - P_3a_j - P_2a_j - P_1a_j \text{ mathematically}) \\
    & \vdots
\end{align*}
In practice the rounding error will accumulate differently, which makes MGS stable.

So \begin{align*}
    r_{i,j} & = q_i^Ta_{j, i-1} \\
    & = q_i^T (I - P_1 - \ldots - P_{i-1}) a_i \\
    & = q_i^Ta_j
\end{align*}

Operations count for MGS (or CGS):
In step $j$ we have the following:
\begin{align*}
    a_{j, 1} & = (I - P_1) a_j \\
    a_{j, 2} & = (I - P_2) a_{j, 1} \\
    & \vdots \\
    a_{j, j - 1} & = (I - P_{j - 1}) a_{j, j-2} \\
    q_j & = \frac{a_{j, j-1}}{\norm{a_{j, j-1}}}
\end{align*}
To count operations, recall that $(I-P)x$ requires $4n$ operations.
\[
(I - P)x = x - q(q^Tx)    
\] $2n - 1$ for $q^Tx$, $n$ for $q(q^Tx)$, $n$ for $x - q(q^Tx)$. Operation count for step $i$ is $(4n)(j-1) + 3n$.

The total count is \[
    \sum_{j=1}^k (4n-1)(j-1) + 3n = (4n-1)\frac{k(k-1)}{2} 3nk = 2nk^2 \text{ leading terms}    
\]
Also:
\[
    \sum_{j=1}^k 4nj = 4n \sum_{i=1}^k j = 4n \int_0^k x \df x = 4nk^2.
\]

\section{Applications of MGS and QR Factorization}
\subsection{Solution of $Ax = b$ for $A \in \R^{m, n}, \rank(A) = n$}
\[QRx = b \implies Rx = Q^Tb\]

Now $Rx = \tilde{b}$ can be solved by back substitution.

Operation count for solving $Ax = b$ using QR.
\begin{enumerate}
    \item calculating $QR: 2n^3$
    \item $\tilde{b} = Q^Tb, 2n^2 - n$
    \item solving $Rx = \tilde{b}$ using back substitution: $n^2$.
\end{enumerate}
Linear system solved using Gaussian elimination with partial pivoting is $n^3$.

\subsection{Connection with Volumes and QR}
Let $a_1$ and $a_2$ be vectors in $\R^m$. They will define a parallelogram as follows.

\subsection{Determinants}


\chapter{}

\section{Norm}
\begin{definition}
    Suppose $x \in \R^n$, then \begin{align*}
        \norm{x}_1 & = |x_1| + \ldots + |x_n| \\
        \norm{x}_2 & = \sqrt{|x_1|^2 + \ldots + |x_n|^2} \\
        \norm{x}_p & = \left(\sum_{i=1}^n |x_i|^p\right)^{\frac{1}{p}} \\
        \norm{x}_\infty & = \max_j |x_j|.
    \end{align*}
\end{definition}
Some properties \begin{enumerate}
    \item $\norm{x} \geq 0$ with equality iff $x = 0$,
    \item $\norm{x + y} \leq \norm{x} + \norm{y}$,
    \item $\norm{\alpha x} = |\alpha| \norm{x}$ for $\alpha \in \R$.
\end{enumerate}

\begin{lemma}
    If $\norm{\cdot}$ is a norm and $A$ is then \[
        \norm{x}_A = \norm{Ax}    
    \] is also a norm over vectors.
\end{lemma}
\begin{proof}
    $A$

    The unit ball of a norm $\{x \mid \norm{x} \leq 1\}$.
\end{proof}

\end{document}
