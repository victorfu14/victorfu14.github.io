\documentclass{report}
\usepackage{amsmath,amssymb,amsthm,textcomp,gensymb,nccmath}
\usepackage{mathtools}
\renewcommand{\qedsymbol}{$\blacksquare$}

\setlength{\topmargin}{0.5in}
\usepackage[margin=4cm]{geometry}
\usepackage{enumerate}

\usepackage{setspace}
\onehalfspacing
\usepackage{parskip}
\setlength{\parskip}{0.5em}
\usepackage[T1]{fontenc}
\usepackage{palatino}

% useful characters/operators
\newcommand{\R}{\mathbb{R}}
\newcommand{\C}{\mathbb{C}}
\newcommand{\Z}{\mathbb{Z}}
\newcommand{\Q}{\mathbb{Q}}
\newcommand{\N}{\mathbb{N}}
\newcommand{\F}{\mathbb{F}}
\newcommand{\B}{\mathbb{B}}
\newcommand{\matP}{\mathbb{P}}
\newcommand{\matE}{\mathbb{E}}
\newcommand{\matS}{\mathbb{S}}
\newcommand{\matH}{\mathbb{H}}
\newcommand{\matT}{\mathbb{T}}
\newcommand{\st}{\ s.t.\ }
\newcommand{\ie}{\ i.e.\ }
\newcommand{\eg}{\ e.g.\ }
\def \diam {\operatorname{diam}}
\def \Hom {\operatorname{Hom}}
\def \id {\operatorname{id}}
\def \tr {\operatorname{tr}}
\def \rk {\operatorname{rk}}
\def \sp {\operatorname{span}}
\def \dist {\operatorname{dist}}
\def \intr {\operatorname{int}}
\def \sgn {\operatorname{sgn}}
\def \im {\operatorname{Im}}
\def \re {\operatorname{Re}}
\def \curl {\operatorname{curl}}
\def \divg {\operatorname{div}}
\def \GL {\operatorname{GL}}
\def \Aut {\operatorname{Aut}}
\def \per {\operatorname{per}}
\def \LE {\operatorname{LE}}
\def \indeg {\operatorname{indeg}}
\def \outdeg {\operatorname{outdeg}}
\def \Par {\operatorname{Par}}
\def \Gr {\operatorname{Gr}}
\def \del {\operatorname{del}}
\def \add {\operatorname{add}}
\newcommand{\pdr}[2]{\dfrac{\partial #1}{\partial #2}}
\newcommand{\df}{\mathrm{d}}
\newcommand{\dr}[2]{\dfrac{\df #1}{\df #2}}
\newcommand{\inner}[2]{\left\langle #1, #2\right\rangle}
\newcommand{\qbin}[2]{\begin{bmatrix}{#1}\\ {#2}\end{bmatrix}_q}
\newcommand{\gen}[1]{\left\langle #1 \right\rangle}
\newcommand{\floor}[1]{\left\lfloor #1 \right\rfloor}
\newcommand{\ceil}[1]{\left\lceil #1 \right\rceil}

% arrows and :=, =:
\makeatletter
\providecommand*{\twoheadrightarrowfill@}{%
  \arrowfill@\relbar\relbar\twoheadrightarrow
}
\providecommand*{\twoheadleftarrowfill@}{%
  \arrowfill@\twoheadleftarrow\relbar\relbar
}
\providecommand*{\xtwoheadrightarrow}[2][]{%
  \ext@arrow 0579\twoheadrightarrowfill@{#1}{#2}%
}
\providecommand*{\xtwoheadleftarrow}[2][]{%
  \ext@arrow 5097\twoheadleftarrowfill@{#1}{#2}%
}
\makeatother

\newcommand{\defeq}{\vcentcolon=}
\newcommand{\eqdef}{=\mathrel{\mathop:}}

% integral for measure theory
\newcommand{\lowerint}{\underline{\int_{\R^d}}}
\newcommand{\upperint}{\overline{\int_{\R^d}}}
\newcommand{\lint}[1]{\underline{\int_{\R^d}} #1 (x)dx}
\newcommand{\uint}[1]{\overline{\int_{\R^d}} #1 (x)dx}
\newcommand{\sint}[1]{\simp{\int_{\R^d} #1 (x)dx}}
\newcommand{\lesint}[1]{\int_{\R^d} #1 (x)dx}

% note taking
\newcommand{\fancyem}[1]{\underline{\textsc{#1}}}

% theorem style
\newtheorem{theorem}{Theorem}[section]
\newtheorem{corollary}{Corollary}[section]
\newtheorem{lemma}{Lemma}[section]
\newtheorem{conjecture}{Conjecture}[section]
\newtheorem{proposition}{Proposition}[section]

\theoremstyle{definition}
\newtheorem{definition}{Definition}[section]
\newtheorem{example}{Example}[section]
\theoremstyle{remark}
\newtheorem*{remark}{Remark}

% pseudocode and algorithms
\usepackage{algorithm}
\usepackage{algpseudocode}
\usepackage{algorithmicx}
\counterwithin{algorithm}{section}
\renewcommand{\algorithmicrequire}{\textbf{Input:}}
\renewcommand{\algorithmicensure}{\textbf{Output:}}

% for clearer reference
\usepackage{hyperref}
\newcommand{\corollaryautorefname}{Corollary}
\newcommand{\lemmaautorefname}{Lemma}
\newcommand{\definitionautorefname}{Definition}
\newcommand{\exampleautorefname}{Example}
\newcommand{\conjectureautorefname}{Conjecture}
\renewcommand{\subsectionautorefname}{Section}
\newcommand{\algorithmautorefname}{Algorithm}

% other styling
\usepackage{fancyvrb, fancyhdr}
\usepackage{tikz-cd}
\usepackage{tikz}
\PassOptionsToPackage{usenames, x11names}{xcolor}
\usepackage{tcolorbox}
\selectcolormodel{cmy}

\newcommand{\edge}{
    \begin{tikzcd}[cramped, sep=small, labels={font=\everymath\expandafter{\the\everymath\textstyle}}]
        u \arrow[r, "e", no head] & v
    \end{tikzcd}
}


\pagestyle{fancy}
\fancyhead[LO,L]{\leftmark}
\fancyhead[RO,R]{Yiwei Fu}
% \fancyhead[C]{}
\fancyfoot[CO,C]{\thepage}
\renewcommand{\sectionmark}[1]{\markboth{#1}{#1}}

\numberwithin{equation}{section}

\newcommand{\fnl}{\parbox[t]{0\linewidth}{}}
\newcommand*\ttlmath[2]{\texorpdfstring{$\boldsymbol{#1}$}{#2}}

\usepackage{epigraph}

% \epigraphsize{\small}% Default
\setlength\epigraphwidth{8cm}
\setlength\epigraphrule{0pt}

\usepackage{etoolbox}

\makeatletter
\patchcmd{\epigraph}{\@epitext{#1}}{\itshape\@epitext{#1}}{}{}
\makeatother

% combinatorics special
\usepackage{pgfopts}
\usepackage{ytableau}

\begin{document}
\title{Notes for EECS 550: Information Theory}
\author{Yiwei Fu, Instructor: David Neuhoff}
\date{FA 2022}
\maketitle


\tableofcontents
Office hours: 

\clearpage
\pagenumbering{arabic}

\chapter{Introduction}
\section{What is Information Theory}

\section{Lossless Coding}
It is a type of data compression.

\fancyem{Goal} to encode data into bits so that \begin{enumerate}
  \item bits can be decoded perfectly or with very high accuracy back into original data;
  \item we use as few bits as possible.
\end{enumerate}

We need to model for data, a measure of decoding accuracy, a measure of compactness.

\fancyem{Model for data}
\begin{definition}
  A \emph{source} is a sequence of i.i.d (discrete) random variables $U_1, U_2, \ldots$
\end{definition}
We would like to assume a known alphabet $A = \{a_1, a_2, \ldots, a_Q\}$ and known probability distribution either through probability mass functions $p_U(u) = \Pr[U = u]$.

\begin{definition}
  Source coding
\end{definition}

\fancyem{Performance measures}
A measure of compactness (efficiency)

\begin{definition}
  rate = encoding rate = average number of encoded bits per data symbol
\end{definition}
  Two versions:
  empricial avg rate $\langle r \rangle = \lim_{N \to \infty} \frac{1}{N}\sum_{k=1}^N L_k(U_1, \ldots, U_k)$.

  Statistical avg rate:
  \[
    \overline{r} = \lim_{N \to \infty} \frac{1}{N}\sum_{k=1}^N \mathbb{E}[L_k(U_1, \ldots, U_k)]
  \]

  where $L_K$ is the number of bits out of the encoder after $U_k$ and before $U_{k+1}$.

Accuracy
per-letter frequency of error \[
  \langle F_{LE}\rangle = \lim_{N \to \infty} \frac{1}{N} \sum_{k=1}^N I(\hat{U}_k=U_k)
\]
per-letter error probability \[
  p_{LE} = \lim_{N \to \infty} \frac{1}{N} \sum_{k=1}^N \mathbb{E}[I(\hat{U}_k=U_k)] = \lim_{N \to \infty} \frac{1}{N} \sum_{k=1}^N \Pr(\hat{U}_k=U_k)
\]

Fixed-length to fixed-length block codes (FFB)

% \[
% \begin{tikzpicture}
%   \draw[] rectangle (0, 0) to (1, 1);
% \end{tikzpicture}  
% \]

characteristics

A code is perfectly lossless (PL) if the $\beta(\alpha(\underline{u})) = \underline{u}$ for all $\underline{u} \in A_U^k$ (the set of all sequences $u_1, \ldots, u_k$).

In order to be perfectly loss, $\alpha$ must be one-to-one. Encode must assign a distinct codeword ($L$ bits) to each data sequences. rate = $L/K$.
We seek $R^*_{PL}(k)$ the smallest rate of any PL code.

Number of sequences of size $k = Q^k$, and number binary sequence of size $L = 2^L$. We need $2^L \gg Q^K$.

\[
\overline{r} = \frac{L}{k} \geq \frac{k\log_2Q}{k} = \log_2Q
\]
Choose $\left\lceil k\log_2Q\right\rceil$, then we have \[
  R^*_{PL}(k) = \frac{\left\lceil k\log_2Q\right\rceil}{k} \leq\frac{k\log_2Q + 1}{k} =\log_2 Q + \frac{1}{k}.
\]
\[
  \log_2 Q \leq R^*_{PL}(k) \leq \log_2 Q + \frac{1}{k}
\]

Let $R^*_{PL}$ be the least rate of any PL FFB code with any $k$. 
$R^*_{PL}(k) \to \log_2 Q$ as $k \to \infty$.

$R^*_{PL} = \inf_k R^*_{PL}(k)$

Now we want rate less and $\log_2 Q$ almost lossless codes.

$R^*_{AL} =  \inf \{r, \text{there is an FFB code with $\bar{r} \leq n$ and arbitrarily small $P_{LE}$}\}$

$ = \inf \{r, \text{there is an FFB code with $\bar{r} \leq n$ and $P_{LE} < \delta$ for all $\delta > 0$}\}$

Instead of per-letter probability $P_{LE}$, we focus on block error probability $P_{BE} = \Pr(\underline{\hat{U}} \neq \underline{U})$

\begin{lemma}
  $P_{BE} \geq P_{LE} \geq \frac{P_{BE}}{k}$
\end{lemma}
\begin{proof}
  See homework 1.
\end{proof}

To analyze, we focus on the set of correctly encoded sequences. $G = \{\underline{u}: \beta(\alpha(\underline{u})) = \underline{u}\}$

Then we have \[P_{BE} = 1 - \Pr[U \in G], |G| \leq 2^k, L \geq \ceil{\log_2|G|}.\]

\fancyem{Question} How large is the smallest set of sequences with length $k$ form $A_U$ with probability $\approx 1$? 

We need to use weak law of large numbers (WLLN).

\begin{theorem}
  Suppose $A_x = \{1, 2, \ldots, Q\}$ with probability $p_1, \ldots, p_Q$. Given $\underline{u} = (u_1, \ldots, u_k) \in A^k_U$.
  \[
    n_q(\underline{u}) \defeq \# \text{times $a_q$ occurs in $\underline{u}$}, \quad f_q(\underline{u}) = \frac{n_q(\underline{u})}{k} = \text{frequency}
  \]
  Fix any $\varepsilon > 0$, \[
    \Pr[f_q(\underline{u}) \doteq p_q \pm \varepsilon] \to 1 \text{ as } k \to \infty.
  \] Moreover,
  \[
    \Pr[f_q(\underline{u}) \doteq p_q \pm \varepsilon, q = 1, \ldots, Q] \to q \text{ as } k \to \infty.  
  \]
\end{theorem}
\fancyem{Notation} $a \doteq b \pm \varepsilon \iff |a - b| \leq \varepsilon$

Consider subset of $A^k_U$ that corresponds to this event $x$.

$T_k = \{\underline{u}: f_q(\underline{u}) \doteq p_q \pm \varepsilon, q = 1, \ldots, Q\}$.

$\Pr[\underline{U} = \underline{u}] = p(u_1)p(u_2)\ldots p(u_k)$.

By WLLN, $\Pr(T_k) \to 1$ as $k \to \infty$.

\fancyem{Key Fact} all sequences in $T_k$ have approximately the same probability.

For $\underline{u} \in T_k$,
\begin{align*}
  p(\underline{u}) & = p(u_1)p(u_2)\ldots p(u_k) \\
  & = p_1^{n_1(u)}p_2^{n_2(u)}\ldots p_k^{n_k(u)} \\
  & = p_1^{kf_1(u)}p_2^{kf_2(u)}\ldots p_k^{kf_k(u)} \\
  & \approx \tilde{p}^k \text{ where } \tilde{p} = p_1^{p_1}p_2^{p_2} \ldots p_Q^{p_Q}.
\end{align*}
So we have $|T_k| \approx \frac{1}{\tilde{p}^k}$.

Then we have \[
  \overline{r} = \frac{\log_2|T_k|}{k} = -\frac{k\log_2\tilde{p}}{k} = -\log_2\tilde{p}. 
\]

Is that rate good? Can we do better? Can we have a set $S$ with probability $\approx 1$ and significantly smaller?

Since $\Pr[\underline{U} \in A^k_U \setminus T_k] \approx 0 \implies \Pr[\underline{U} \in S] \approx \Pr[\underline{U} \in S \cap T_k] \approx \frac{|S|}{|T_k|}$. So when $k$ is large, $T_k$ is the smallest set with large probability. And $R^*_{AL} \approx -\log \tilde{p}$.

How to express $\tilde{p}$.

\begin{align*}
  -\log \tilde{p} & = -\log \prod_{i=1}^Q p_i^{p_i} \\
  & = -\sum_{i=1}^Q p_i\log p_i \eqdef \text{entropy} = H.
\end{align*}

Some properties of $H$:
\begin{enumerate}
  \item its unit is bits
  \item $H \geq 0$.
  \item $H = 0 \implies \iff p_q = 1$ for some $q$.
  \item $H \leq \log_2 Q$.
  \item $H = \log_2 Q \iff p_q = \frac{1}{Q}$ for all $q$.
\end{enumerate}




\end{document}
