\documentclass{report}
\usepackage{amsmath,amssymb,amsthm,textcomp,gensymb}
\usepackage{mathtools}
\usepackage{centernot}
\renewcommand{\qedsymbol}{$\blacksquare$}

\setlength{\topmargin}{0.5in}
\usepackage[margin=4cm]{geometry}
\usepackage{enumerate}
\renewcommand{\labelenumi}{(\alph{enumi})}
\renewcommand{\labelenumii}{(\arabic{enumii})}

\usepackage{setspace}
\onehalfspacing
\usepackage{parskip}
\setlength{\parskip}{0.5em}
\usepackage[T1]{fontenc}
\usepackage{palatino}

% useful characters/operators
\newcommand{\R}{\mathbb{R}}
\newcommand{\C}{\mathbb{C}}
\newcommand{\Z}{\mathbb{Z}}
\newcommand{\Q}{\mathbb{Q}}
\newcommand{\N}{\mathbb{N}}
\newcommand{\matP}{\mathbb{P}}
\newcommand{\matS}{\mathbb{S}}
\newcommand{\matH}{\mathbb{H}}
\newcommand{\matT}{\mathbb{T}}
\newcommand{\st}{\ s.t.\ }
\newcommand{\ie}{\ i.e.\ }
\newcommand{\eg}{\ e.g.\ }
\newcommand{\cA}{\mathcal{A}}
\newcommand{\cB}{\mathcal{B}}
\newcommand{\cC}{\mathcal{C}}
\newcommand{\cE}{\mathcal{E}}
\newcommand{\cF}{\mathcal{F}}
\newcommand{\cH}{\mathcal{H}}
\newcommand{\cL}{\mathcal{L}}
\newcommand{\cM}{\mathcal{M}}
\newcommand{\cR}{\mathcal{R}}
\def \diam {\operatorname{diam}}
\def \Hom {\operatorname{Hom}}
\def \id {\operatorname{id}}
\def \tr {\operatorname{tr}}
\def \dist {\operatorname{dist}}
\def \intr {\operatorname{int}}
\def \sgn {\operatorname{sgn}}
\def \im {\operatorname{Im}}
\def \re {\operatorname{Re}}
\def \curl {\operatorname{curl}}
\def \divg {\operatorname{div}}
\def \GL {\operatorname{GL}}
\def \supp {\operatorname{supp}}
\def \BV {\operatorname{BV}}
\def \NBV {\operatorname{NBV}}

\newcommand{\pdr}[2]{\dfrac{\partial #1}{\partial #2}}
\newcommand{\dr}[2]{\dfrac{\mathrm{d} #1}{\mathrm{d} #2}}
\newcommand{\df}{\ \mathrm{d}}
\newcommand{\ndf}{\mathrm{d}}
\newcommand{\inner}[2]{\left\langle #1, #2\right\rangle}
\newcommand{\gen}[1]{\left\langle #1 \right\rangle}
\newcommand{\norm}[1]{\left\| #1 \right\|}
\newcommand{\nll}{\centernot{\ll}}

% arrows and :=, =:
\makeatletter
\providecommand*{\twoheadrightarrowfill@}{%
  \arrowfill@\relbar\relbar\twoheadrightarrow
}
\providecommand*{\twoheadleftarrowfill@}{%
  \arrowfill@\twoheadleftarrow\relbar\relbar
}
\providecommand*{\xtwoheadrightarrow}[2][]{%
  \ext@arrow 0579\twoheadrightarrowfill@{#1}{#2}%
}
\providecommand*{\xtwoheadleftarrow}[2][]{%
  \ext@arrow 5097\twoheadleftarrowfill@{#1}{#2}%
}
\makeatother

\newcommand{\defeq}{\vcentcolon=}
\newcommand{\eqdef}{=\mathrel{\mathop:}}

% integral for measure theory
\newcommand{\lowerint}{\underline{\int_{\R^d}}}
\newcommand{\upperint}{\overline{\int_{\R^d}}}
\newcommand{\lint}[1]{\underline{\int_{\R^d}} #1 (x)dx}
\newcommand{\uint}[1]{\overline{\int_{\R^d}} #1 (x)dx}
\newcommand{\sint}[1]{\simp{\int_{\R^d} #1 (x)dx}}
\newcommand{\lesint}[1]{\int_{\R^d} #1 (x)dx}

% note taking
\newcommand{\fancyem}[1]{\underline{\textsc{#1}}}

% theorem style
\newtheorem{theorem}{Theorem}[chapter]
\newtheorem{corollary}[theorem]{Corollary}
\newtheorem{lemma}[theorem]{Lemma}
\newtheorem{conjecture}[theorem]{Conjecture}
\newtheorem{proposition}[theorem]{Proposition}

\theoremstyle{definition}
\newtheorem{definition}[theorem]{Definition}
\newtheorem{example}[theorem]{Example}
\theoremstyle{remark}
\newtheorem*{remark}{Remark}

% for clearer reference
\usepackage{hyperref}
\newcommand{\corollaryautorefname}{Corollary}
\newcommand{\lemmaautorefname}{Lemma}
\newcommand{\definitionautorefname}{Definition}
\newcommand{\exampleautorefname}{Example}
\newcommand{\conjectureautorefname}{Conjecture}
\renewcommand{\subsectionautorefname}{Section}
\usepackage{cleveref}

% other styling
\usepackage{fancyvrb, fancyhdr}
\usepackage{tikz}
\usepackage{tcolorbox}

\usepackage{tikz-cd}

\pagestyle{fancy}
\fancyhead[LO,L]{\leftmark}
\fancyhead[RO,R]{Yiwei Fu}
\fancyfoot[CO,C]{\thepage}
\renewcommand{\sectionmark}[1]{\markboth{#1}{#1}}

\newcommand{\fnl}{\parbox[t]{0\linewidth}{}}
\newcommand*\ttlmath[2]{\texorpdfstring{$\boldsymbol{#1}$}{#2}}

\usepackage{epigraph}

% \epigraphsize{\small}% Default
\setlength\epigraphwidth{8cm}
\setlength\epigraphrule{0pt}

\usepackage{etoolbox}

\makeatletter
\patchcmd{\epigraph}{\@epitext{#1}}{\itshape\@epitext{#1}}{}{}
\makeatother

\begin{document}
\pagenumbering{gobble}
\clearpage
\thispagestyle{empty}
% \renewcommand{\ref}[1]{\autoref{#1}}
\title{Notes for Math 597 -- Real Analysis}
\author{Yiwei Fu}
\date{WN 2022}
\maketitle

\tableofcontents
Office hour is Mon 12:30 - 1:30, Tue 12:30 - 1:30 in person EH 5838, Th 1 - 2 online.

\clearpage
\pagenumbering{arabic}
\chapter{Abstract Measure}

\section{\ttlmath{\sigma}{sigma}-Algebra}
\begin{definition}
Let $X$ be a set. A collection $\mathcal{M}$ of subsets of $X$ is called a $\sigma$-algebra on $X$ if
\begin{itemize}
\item
	$\emptyset \in \mathcal{M}.$
\item
	$\mathcal{M}$ is closed under \emph{complements}: $E \in \mathcal{M} \implies E^c \in \mathcal{M}.$
\item
	$\mathcal{M}$ is closed under \emph{countable unions}: $E_1, E_2, \ldots \in \mathcal{M} \implies \bigcup_{i=1}^\infty E_i \in \mathcal{M}.$
\end{itemize}
\end{definition}
\fancyem{Simple properties:}
\begin{itemize}
\item
	$X = \emptyset^c \in \mathcal{M}.$
\item
	$\bigcap_{i=1}^\infty E_i = \left(\bigcup_{i=1}^n E_i^c\right)^c \in \mathcal{M}.$ It is closed under countable intersections.
\item
	$\bigcup_{i=1}^N E_i = E_i \cup \ldots \cup E_n \cup \emptyset \cup \ldots.$ It is closed under finite unions (similarly, intersections).
sigma\item
	$E \setminus F = E \cap F^c \in \mathcal{M}, E \triangle F = (E \cap F^c) \cup (F \cap E^c) \in \mathcal{M}.$
\end{itemize}

\begin{example}
\begin{enumerate}[(a)]
\item $\mathcal{A} = \mathcal{P}(X)$ power algebra.
\item $\mathcal{A} = \{\emptyset, X\}$ trivial algebra.
\item Let $B \subset X, B \neq \emptyset, B \neq X. \mathcal{A} = \{\emptyset, B, B^c, X\}.$
\end{enumerate}
\end{example}

\begin{lemma}(An intersection of $\sigma$-algebras is a $\sigma$-algebra)
Let $\mathcal{A}_\alpha, \alpha \in I,$ be a family a $\sigma$-algebras of $X$. Then $\bigcap_{\alpha \in I} A_\alpha$ is a $\sigma$-algebra. ($I$ can be uncountable.)
\end{lemma}
\begin{proof}
DIY
\end{proof}

\begin{definition}
For $\mathcal{E} \subset \mathcal{P}(X)$ (not necessarily a $\sigma$-algebra), let $\langle \mathcal{E}\rangle$ be the intersection of all $\sigma$-algebras on $X$ that contains $\mathcal{E}.$ Call it the $\sigma$-algebra generated by $\mathcal{E}.$
\end{definition}
\begin{itemize}
\item $\langle \mathcal{E}\rangle$ is the \emph{smallest} $\sigma$-algebra containing $\mathcal{E}$ and is \emph{unique}.
\item $\{\emptyset, B, B^c, X\} = \langle \{B\} \rangle = \langle \{B^c\}\rangle = \langle \{\emptyset, B\}\rangle.$
\end{itemize}

The above definition gives us (potentially) lots of examples of $\sigma$-algebra on a set $X$

\begin{lemma}\label{le:alg}
\begin{enumerate}[(a)]
\item
	Suppose $\mathcal{E} \subset \mathcal{P}(X), \mathcal{A}$ is a $\sigma$-algebra on $X$. $\mathcal{E} \in \mathcal{A} \implies \langle\mathcal{E}\rangle \in \mathcal{A}.$
\item
	$E \subset F \subset \mathcal{P}(X) \implies \langle \mathcal{E} \rangle \subset \langle\mathcal{F}\rangle.$
\end{enumerate}
\end{lemma}
\begin{proof}

\end{proof}

\begin{definition}
For a topological space $X$, \emph{the Borel $\sigma$-algebra} $\mathcal{B}(X)$ is the $\sigma$-algebra generated by the collection of open sets.
\end{definition}

\begin{example}($X = \R$)
$\mathcal{B}(\R)$ contains the following collections:
\begin{align*}
	\mathcal{E}_1 = \{(a, b) \mid a < b\},& \quad \mathcal{E}_2 = \{[a, b] \mid a < b\},\\
	\mathcal{E}_3 = \{(a, b] \mid a < b\},& \quad \mathcal{E}_4 = \{[a, b) \mid a < b\},\\
	\mathcal{E}_5 = \{(a, \infty) \mid a \in \R\},& \quad \mathcal{E}_6 = \{[a, \infty) \mid a \in \R \},\\
	\mathcal{E}_7 = \{(-\infty, a) \mid a \in \R\},& \quad \mathcal{E}_8 = \{(-\infty, a] \mid a < b\}.
\end{align*}

\end{example}

\begin{proposition}
$\mathcal{B}(\R) = \langle\mathcal{E}_i\rangle$ for each $i = 1, \ldots, 8$. 
\end{proposition}
\begin{proof}
Use \ref{le:alg}.
\end{proof}

\begin{definition}
$(X, \mathcal{A})$ is called a measurable space.
\end{definition}


\section{Measures}
\begin{definition}
A measure on $(X, \mathcal{A})$ is a function $\mu: \mathcal{A} \to [0, \infty] \st$
\begin{enumerate}
\item $\mu(\emptyset) = 0$
\item (countable additive) For $A_1, A_2, \ldots \in \mathcal{A}$ disjoint we have
\[
\mu\left(\bigcup_1^\infty A_i \right) = \sum_{i=1}^\infty \mu(A_i).
\]
\end{enumerate}
$(X,  \mathcal{A}, \mu)$ is then called a measure space.
\end{definition}

\begin{example}
\begin{enumerate}
\item For any $(X, \mathcal{A}), \mu(A) = \#A$ counting measure.
\item For any $(X, \mathcal{A}),$ let $x_0 \in X.$ The Dirac measure at $x_0$ is
\[
\mu(A) = \begin{cases}
1 & x_0 \in A, \\
0 & x_0 \notin A.
\end{cases}
\]
\item For $(\N, \mathcal{P}(\N)),$ let $a_1, a_2, \ldots \in [0, \infty).$ $\mu(A) = \sum_{i \in A} a_i$ is a measure.
\end{enumerate}
\end{example}

$(X,  \mathcal{A})$ measurable space

$(X,  \mathcal{A}, \mu)$ measure space

$\mu: \mathcal{A} \to [0, \infty] \st \mu(\emptyset) = 0, $ countable additivity.

\fancyem{Note}: $A, B \in \mathcal{A}, A \subset B, $ then $ \mu(B \setminus A) + \mu(A) = \mu(B) \implies \mu(B \setminus A) = \mu(B) - \mu(A)$ if $\mu(A) < \infty.$

\setcounter{theorem}{12}
\begin{theorem}
Suppose $(X,  \mathcal{A}, \mu)$ a measure space. Then
\begin{enumerate}
\item(monotonicity)
\[A, B \in \mathcal{A}, A \subset B \implies \mu(A) \leq \mu(B).\]
\item(countable subadditivity)
	\[A_1, A_2, \ldots, \in \mathcal{A}, \implies \mu\left(\bigcup_i^\infty A_i\right) \leq \sum_i^\infty \mu(A_i).\]
\item(continuity from below/(MCT) from sets)
	\[A_1, A_2, \ldots \in \mathcal{A}, A_1 \subset A_2 \subset A_3 \subset \ldots \implies \mu\left(\bigcup_i^\infty A_i\right) = \lim_{n \to \infty} \mu(A_n).\]
\item(continuity from above)
	\[A_1, A_2, \ldots \in \mathcal{A}, A_1 \supset A_2 \supset A_3 \supset \ldots, \mu(A_1) < \infty \implies \mu\left(\bigcap_i^\infty A_i\right) = \lim_{n \to \infty} \mu(A_n).\]
\end{enumerate}
\end{theorem}
\begin{proof}
    (a), (b), DIY.

    For (c), let $B_1 = A_1, B_i = A_i \setminus A_{i-1}, i \geq 2. B_i \in \mathcal{A}$ and are disjoint.
    \begin{align*}
        & \bigcup_i^\infty A_i = \bigcup_i^\infty B_i \\
        \implies\ & \mu\left(\bigcup_i^\infty A_i\right) = \mu\left(\bigcup_i^\infty B_i\right) = \sum_{i}^\infty \mu(B_i) = \lim_{n \to \infty} \sum_{i}^n \mu(B_i) = \lim_{n \to \infty} \mu(A_n).
    \end{align*}


For (d), let $E_i = A_1 \setminus A_i.$ Hence $E_i \in \mathcal{A}, E_1 \subset E_2 \subset \ldots$
We have
\[
\bigcup_i^\infty E_i = \bigcup_i^\infty (A_1 \setminus A_i) = A_1 \setminus \left(\bigcap_1^\infty A_i\right) \implies \bigcap_1^\infty A_i = A_1 \setminus \left(\bigcup_1^\infty E_i\right).
\]
Hence
\[
\mu\left(\bigcap_1^\infty A_i \right) = \mu(A_1) - \mu\left(\bigcup_1^\infty E_i\right) = \mu(A_1) - \lim_{n\to \infty}\mu(E_n) = \mu(A_1) - \lim_{n\to \infty} \mu(A_1) - \mu(A_n). \qedhere
\]
\end{proof}

\fancyem{Note:} the condition that $\mu(A_1) < \infty$ cannot be dropped. \\
For example,
in $(\N, \mathcal{P}(N), \text{counting measure})$, let $A_n = \{n, n+1, n+2\}, A_1 \supset A_2 \supset A_3 \supset \ldots$ We have $\bigcap_1^\infty = \emptyset \implies \mu\left(\bigcap_1^\infty A_i \right) = 0.$

\begin{definition}
For $(X,  \mathcal{A}, \mu)$ measure space,
\begin{itemize}
\item
$A \subset X$ is a $\mu$-null set if $A \in \mathcal{A}, \mu(A)= 0.$
\item
$A \subset X$ is a $\mu$-subnull set if $\exists B, \mu$-null set $A \subset B.$
\item
$(X,  \mathcal{A}, \mu)$ is a complete measure space if every $\mu$-subnull set is $\mathcal{A}$-measurable.
\end{itemize}
\end{definition}

\begin{definition}$(X,  \mathcal{A}, \mu)$ measure space.
A statement $P(x), x \in X$ holds $\mu$-almost everywhere (a.e.) if the set $\{x \in X \mid P(x) \text{does not hold}\}$ is $\mu$-null.
\end{definition}

\begin{definition}$(X,  \mathcal{A}, \mu)$ measure space.
\begin{itemize}
\item
	$\mu$ is a \emph{finite measure} is $\mu(X) < \infty.$
\item
	$\mu$ is a \emph{$\sigma$-finite measure} if $X = \bigcup_1^\infty X_n, X_n \in \mathcal{A}, \mu(X_n) < \infty.$
\end{itemize}
\end{definition}
HW: every measure space can be "completed."

\section{Outer Measures}
\begin{definition}
An \emph{outer measure} on $X$ is $\mu^*: \mathcal{P}(X) \to [0, \infty] \st$
\begin{itemize}
\item
	$\mu^*(\emptyset) = 0$
\item(monotonicity)
	$\mu^*(A) \leq \mu^*(B)$ if $A \subset B.$
\item(countable subadditivity)
	\[\forall A_1, A_2, \ldots \in X, \mu^*\left(\bigcup_i^\infty A_i\right) \leq \sum_i^\infty \mu^*(A_i).\]
\end{itemize}
\end{definition}
\begin{example}
	For $A \subset \R$,
	\[\mu^*(A) = \inf\left\{\sum_{i=1}^\infty(b_i - a_i) \ \biggr\rvert\ \bigcup_1^\infty (a_i, b_i) \supset A\right\}.\]
	is an outer measure due to the next proposition.
\end{example}
\begin{proposition}\label{prop:induce}(1.19)
Let $\mathcal{E} \in \mathcal{P}(X) \st \emptyset, X \in \mathcal{E}.$
Let $\rho: \mathcal{E} \to [0, \infty] \st \rho(\emptyset) = 0.$
Then
\[
\mu^*(A) = \inf\left\{\sum_{i=1}^\infty \rho(E_i) \ \biggr\rvert\ E_i \in \mathcal{E}, \forall i \in N, \bigcup_1^\infty E_i \supset A\right\}\]
is an outer measure on $X$.
\end{proposition}
\begin{proof}
\begin{enumerate}
	\item
	$\mu^*$ is well-defined (inf is taken over non-empty set.)
	\item
	$\mu^*(\emptyset) = 0$
	\item
	$A \subset B \implies \mu^*(A) \leq \mu^*(B).$
\end{enumerate}
We check the countable subadditivity.

Let $A_1, A_2, \ldots \subset X.$ If one of $\mu^*(A_i) = \infty,$ then the result holds. Suppose $\mu^*(A_n) < \infty, \forall n \in \N.$

"Give your self a room of epsilon":

Fix $\varepsilon > 0.$ We will show \[\mu^*\left(\bigcup_1^\infty A_n\right) \leq \sum_1^\infty \mu^*(A_i) + \varepsilon.\]

For each $n \in \N, \exists E_{n,1}, E_{n,2}, \ldots \in \mathcal{E} \st$ \[\bigcup_{k=1}^\infty E_{n, k} \supset A_n \quad \text{and} \quad \mu^*(A_n) + \frac{\varepsilon}{2^n}\geq \sum_{k=1}^\infty \rho(E_{n, k}).\]

Then, \[
\bigcup_1^\infty A_n \subset \bigcup_{n=1}^\infty \bigcup_{k=1}^\infty E_{n, k} = \bigcup_{(n, k) \in \N^2} E_{n, k}.
\]

\fancyem{Recall:} Tonelli's thm for series.
If $a_{ij} \in [0, \infty], \forall i, j \in \N, $ then
\[
\sum_{(i, j) \in \N^2} a_{ij} = \sum_{i=1}^\infty\sum_{j=1^\infty} a_{ij} = \sum_{j=1}^\infty\sum_{i=1}^\infty a_ij.
\]
Hence 
\begin{align*}
\mu^*\left(\bigcup_{n=1}^\infty A_n\right) \leq \sum_{n=1}^\infty \rho(E_{k, n}) = \sum_{n=1}^\infty \sum_{k=1}^\infty \rho(E_{k, n}) \leq \sum_{n=1}^\infty\left(\mu^*(A_n) + \frac{\varepsilon}{2^n}\right) = \sum_{n=1}^\infty \mu^*(A_n) + \varepsilon.
\end{align*}
We have shown countable subadditivity.
\end{proof}

Outer measure is very close to a measure. Here the textbooks diverge.

\cite{taoIntroductionMeasureTheory2011} introduces Lebesgue measure on $\R$ using topological qualities of subsets of $\R.$\\
\cite{follandRealAnalysisModern1999} introduces abstract method by Carathéodory and Kolmogorov.

\begin{definition}
Let $\mu^*$ be an outer measure on $X$. We say $A \subset X$ is Carathéodory measurable with respect to $\mu^*$ if $\forall E \subset X, \mu^*(E) = \mu^*(E \setminus A) + \mu^*(E \cap A).$ 
\end{definition}
\begin{lemma}
Let $\mu^*$ be an outer measure on $X$. Suppose $B_1, B_2, \ldots, B_N$ are disjoint $C$-measurable sets. Then,
\[
\forall E \subset X, \mu^*\left(E \cap \left(\bigcup_{1}^N B_i\right)\right) = \sum_{i=1}^n \mu^*(E \cap B_i)
\]
\end{lemma}
\begin{proof}
\begin{align*}
\mu^*\left(E \cap \left(\bigcup_1^N B_i\right)\right) = \mu^*(E \cap B_1) +  \mu^*\left(E \cap \left(\bigcup_2^N B_i\right)\right)
\end{align*}
because $B_1$ is $C$-measurable.
Then, iterate.
\end{proof}

Improved version: 

$B_1, B_2, \ldots$ $C$-measurable and \emph{disjoint} $\implies \mu^*\left(E \cap \bigcup_1^\infty B_n\right) = \sum_1^\infty \mu^*\left(E \cap B_n\right), \forall E \subset X.$
\begin{proof}
\begin{align*}
\sum_1^\infty \mu^*(E \cap B_n) & \geq \mu^*\left(E \cap \bigcup_1^\infty B_n\right) \\
& \geq \mu^*\left(E \cap \bigcup_1^N B_n \right) = \sum_1^N \mu^*(E \cap B_n.)
\end{align*}
Take $N \to \infty$ or note that $N \in \N$ is arbitrary we get the result.
\end{proof}


First big theorem:
\begin{theorem}[Carathéodory extension theorem]
Let $\mu^*$ be an outer measure on $X.$ Let $\mathcal{A}$ be the collection of $C$-measurable sets with respect to $\mu^*.$ Then
\begin{enumerate}
\item
	$\mathcal{A}$ us a $\sigma$-algebra on $X$.
\item
	$\mu = \mu^*|_{\mathcal{A}}$ is a measure on $(X, \mathcal{A}).$
\item
	$(X, \mathcal{A}, \mu)$ is a complete measure space.
\end{enumerate}
\end{theorem}
\begin{proof}
\begin{enumerate}
\item
	\begin{enumerate}
	\item
		$\emptyset \in \mathcal{A}.$
	\item
		$\mathcal{A}$ is closed under complements.
	\item
		To show $\mathcal{A}$ closed under countable unions.
		\begin{itemize}
		\item(finite union)\\
		\fancyem{Claim} $A, B \in \mathcal{A} \implies A \cup B \in \mathcal{A}.$
		
		\begin{figure}[h]
		\centering
		\begin{tikzpicture}
		\draw (0,2) circle (1.5) node[left] {$A$};
		\draw (2,2) circle (1.5) node[right] {$B$};
		\draw (-0.5,1.5) rectangle (2.5,0) node[above left] {$E$};
		\node[] at (0, 1.0) {$1$};
		\node[] at (1, 1.2) {$2$};
		\node[] at (2, 1.0) {$3$};
		\node[] at (1, 0.3) {$4$};
		\end{tikzpicture}
		\label{fig:AB}
		\caption{Venn diagram of $A, B, E$}
		\end{figure}
		
		Fix arbitrary $E \subset X$. We need to show 
		\[
			\mu^*(E) = \mu^*(E \cap (A \cup B)) + \mu^*(E \setminus (A \cup B)).		
		\]
		\ie
		\[
			\mu^*(1 \cup 2 \cup 3 \cup 4) = \mu^*(1 \cup 2 \cup 3) + \mu^*(4)
		\]
		Since $A$ is $C$-measurable, we have
		\[
			\mu^*(1 \cup 2 \cup 3 \cup 4) = \mu^*(1 \cup 2) + \mu^*(3 \cup 4)
		\]
		\[
			\mu^*(1 \cup 2 \cup 3) = \mu^*(1 \cup 2) + \mu^*(3)
		\]
		Similarly since $B$ is $C$-measurable, we have
		\[
			\mu^*(3 \cup 4) = \mu^*(3) + \mu^*(4)
		\]
		Hence
		\begin{align*}
		\mu^*(1 \cup 2 \cup 3 \cup 4)
		& = \mu^*(1 \cup 2) + \mu^*(3 \cup 4) \\
		& = \mu^*(1 \cup 2 \cup 3) - \mu^*(3) + \mu^*(3) + \mu^*(4) \\
		& = \mu^*(1 \cup 2 \cup 3) + \mu^*(4).
		\end{align*}
		\item(countable disjoint unions)\\
		Let $A_1, A_2, \ldots \in \mathcal{A}$ and \emph{disjoint}.
		
		Fix $E \subset X$ arbitrary.
		Since $\mu^*$ is countably subadditive, 
		\[
		\mu^*(E) \leq \mu^*\left(E \cap \bigcup_1^\infty\right) + \mu^*\left(E \setminus \bigcup_1^\infty A_n\right)
		\]
		Fix $n \in \N.$
		\begin{align*}
		& \implies \bigcup_1^N A_n \in \mathcal{A} \\
		& \implies \mu^*(E) = \mu^*\left(E \cap \bigcup_1^N\right) + \mu^*\left(E \setminus \bigcup_1^N A_n\right) \\
		& \geq \sum_{1}^N \mu^*(E \cap A_n) + \mu^*\left(E \setminus \bigcup_1^\infty A_n\right) \text{ by lemma}.
		\end{align*}
		Take $n \to \infty.$
		\item
		(countable unions)\\
		Let $A_1, A_2, \ldots \in \mathcal{A}.$ Take $E_1 = A_1, E_n = A_n \setminus \left(\bigcup_1^{n-1} A_i\right)$ for $n \geq 2.$ Then $\bigcup A_n = \bigcup E_n$ and $E_n$'s are disjoint.
		\end{itemize}
	\end{enumerate}
	\item
	Firstly we have $\mu(\emptyset) = \mu^*(\emptyset) = 0.$
	
	Countable additvity of $\mu^*$ on $\mathcal{A}$ follows from the improved lemma with $E = X$.	
	
	\item
	HW. \qedhere
\end{enumerate}
\end{proof}

\section{Hahn-Kolmogorov Theorem}
\fancyem{Recall} \ref{prop:induce} Let $\mathcal{E} \subset \mathcal{P}(X) \st \emptyset, X \in \mathcal{E}.$ Let $\rho: \mathcal{E} \to [0, \infty] \st \rho(\emptyset) = 0$

\[
(\mathcal{E}, \rho) \xrightarrow[1.19]{} (\mathcal{P}(X), \mu^*) \xrightarrow[\text{C-theorem}]{} (A, \mu)
\] 
\fancyem{Question} $\mathcal{E} \subset \mathcal{A}$ and $\mu|_\mathcal{E} = \rho$? No!
\begin{definition}
Let $\mathcal{A}_0$ be an algebra on $X$. We say $\mu_0: \mathcal{A}_0 \to [0, \infty]$ is a \emph{pre-measure} if 
\begin{enumerate}
\item $\mu_0(\emptyset) = 0.$
\item (finite additivity)
\[
\mu_0\left(\bigcup_1^N A_i1\right) = \sum_1^N \mu_0(A_i) \text{ if } A_1, \ldots, A_N \in \mathcal{A}_0 \text{ are disjoint.}
\]
\item (countable additivity within the algebra) If $A \in \mathcal{A}_0$ and 
\[
A = \bigcup_1^\infty A_n, A_n \in \mathcal{A}_0 \text{ and are disjoint, then } \mu_0(A) = \sum_1^\infty \mu_0(A_n)
\]
\end{enumerate}
\end{definition}

\fancyem{Notation:} Folland uses $\mathcal{M}$ for $\sigma$-algebra and $\mathcal{A}$ for algebra. (Jinho) uses $\mathcal{A}$ for $\sigma$-algebra and $\mathcal{A}_0$ for alegbra.

\begin{example}
$\mathcal{A}_0$ finite disjoint unions of $(a, b].$
\[
\mu_0\left( \bigcup_1^\infty (a_i, b_i]\right) = \sum_1^\infty (b_i - a_i) \text{ or $b_i^n - a_i^n, e^{b_i} - e^{a_i},$ etc.}
\]
\end{example}

\begin{lemma}
\begin{itemize}
\item (a) + (c) $\implies$ (b).
\item $\mu_0$ is monotone.
\end{itemize}
\end{lemma}

\begin{theorem}[Hahn-Kolmogorov Theorem]
Let $\mu_0$ be a pre-measure on algebra $\mathcal{A}_0$ on $X$. Let $\mu^*$ be the outer measure induced by $(\mathcal{A}_0, \mu_0)$ in \ref{prop:induce}. Let $\mathcal{A}$ and $\mu$ be the Carathéodory $\sigma$-algebra and measure for $\mu^* \implies (\mathcal{A}, \mu)$ extends $(\mathcal{A}_0, \mu_0) \ie \mathcal{A} \supset \mathcal{A}_0, \mu|_{\mathcal{A}_0} = \mu_0.$ 
\end{theorem}
\begin{proof}
\begin{enumerate}
\item ($\mathcal{A} \supset \mathcal{A}_0$)
Let $A \in \mathcal{A}_0$. 

Question: $A \in \mathcal{A}$? i.e. is $A$ $C$-measurable? i.e. $\mu^*(E) = \mu^*(E \cap A) + \mu^*(E \cap A^c), \forall E \subset X.$

Fix $E \subset X.$
\begin{itemize}
\item (countable) subadditivity of $\mu^* \implies \mu^*(E) \leq \mu^*(E \cap A) + \mu^*(E \cap A^c).$
\item If $\mu^*(E) = \infty$ then $\mu^*(E) = \infty \geq \mu^*(E \cap A) + \mu^*(E \cap A^c).$
\item If $\mu^*(E) < \infty.$

Fix $\varepsilon > 0.$ By the definition of $\mu^*, \exists B_1, B_2, \ldots \in \mathcal{A}_0 \st \bigcup_1^\infty B_n \supset E$ and \[\mu^*(E) + \varepsilon \geq \sum_1^\infty \mu_0(B_n) = \sum_1^\infty \left(\mu_0(B_n \cap A) + \mu_0\left(B_n \cap A^c\right))\right).\]

Note that \[
\bigcup_1^\infty (B_n \cap A) \supset E \cap A, \quad \bigcup_1^\infty \left(B_n \cap A^c\right) \supset E \cap A^c \implies \geq
\]
\end{itemize}

\item
Let $A \in \mathcal{A}_0$. We want to show that $\mu(A) = \mu_0(A).$

By definition, $\mu(A) = \mu^*(A).$
\begin{itemize}
\item
Let $B_i = \begin{cases}
A & i = 1, \\
\emptyset & i = 2
\end{cases} \in \mathcal{A}_0$ and $\bigcup_{1}^\infty B_i \supset A.$

Hence $\mu^*(A) \leq \sum_1^\infty \mu_0(B_i) = \mu_0(A).$

\item
Let $B_i \in \mathcal{A}_0, \bigcup_{1}^\infty B_i \supset A$ an arbitrary collection of sets.

Let $C_1 = A \cap B_1, C_i = A \cap B_i \setminus \left(\bigcup_{j=1}^{i-1} B_j\right).$ Then $A = \bigcup_1^\infty$ is a disjoint countable union. By countable additivitiy we have
\[
\mu_0(A) = \sum_1^\infty \mu_0(C_i) \implies \mu_0(A) \leq \sum_1^\infty \mu_0(B_i).
\]
\end{itemize}
Hence we have $\mu_0(A) = \mu^*(A) = \mu(A)$. We have completed our proof. \qedhere
\end{enumerate}
\end{proof}

\begin{definition}
Such $(\mathcal{A}, \mu)$ is called the \emph{Hahn-Kolmogorov extension} of $(\mathcal{A}_0, \mu_0)$, and is also called the \emph{Carathéodory $\sigma$-algebra} for $(\mathcal{A}_0, \mu_0)$.
\end{definition}

\begin{theorem}[uniqueness of HK extension]\label{th:hkunique}
Let $\mathcal{A}_0$ be an algebra on $X$, $\mu_0$ be a pre-measure on $\mathcal{A}_0$, $(\mathcal{A}, \mu)$ be the Hahn-Kolmogorov extension of $(\mathcal{A}_0, \mu_0)$.
And let $(\mathcal{A}', \mu')$ be another extension of $(\mathcal{A}_0, \mu_0)$.

If $\mu_0$ is \emph{$\sigma$-finite}, then $\mu \mid_{\mathcal{A} \cap \mathcal{A}'} = \mu' \mid_{\mathcal{A} \cap \mathcal{A}'}$.
\end{theorem}

\fancyem{Note} $\sigma$-finite means
\[\forall X, X = \bigcup_1^\infty X_n, X_n \in \mathcal{A}_0, \mu_0(X_n) < \infty.\]

\begin{corollary}
Let $\mu_0$ be a pre-measure on algebra $\mathcal{A}_0$ on $X$. Suppose $\mu_0$ is $\sigma$-finite, then $\exists!$ measure $\mu$ on $\langle \mathcal{A}_0 \rangle$ that extends $\mathcal{A}_0$. Furthermore, 
\begin{enumerate}
\item the completion of $(X, \langle \mathcal{A}_0 \rangle, \mu)$ is the HK extension of $(\mathcal{A}_0, \mu_0).$
\item 
\[
\mu(A) = \inf \left\{\sum_{i=1}^\infty \mu_0(B_i) \mid B_i \subset A_0, \forall i \in \N, \bigcup_1^\infty B_i \supset A\right\}, \forall A \in \overline{\langle \mathcal{A}_0\rangle}.
\]
\end{enumerate}
\end{corollary}

\begin{proof}[Proof of \ref{th:hkunique}]
Let $A \in \mathcal{A} \cap \mathcal{A}'$. We need to show $\mu(A) = \mu^*(A) = \mu'(A).$
\begin{itemize}
\item
$\mu^*(A) \geq \mu'(A)$ (HW)

\item
$\mu(A) \leq \mu'(A)$:
\begin{enumerate}[(i)]
\item Assume $\mu(A) < \infty.$ Fix $\varepsilon > 0$. Then $\exists B_i \in \mathcal{A}_0, \forall i \in \N, \bigcup_1^\infty B_i \supset A \st$
\[
\mu(A) + \varepsilon = \mu^*(A) + \varepsilon \geq \sum_1^\infty \mu_0(B_i) = \sum_1^\infty \mu(B_i) \geq \mu \left(\bigcup_1^\infty B_i\right) = \mu(B)
\]
Hence $\mu(B \setminus A) = \mu(B) - \mu(A) \leq \varepsilon.$

On the other hand, 
\[
\mu(B) = \lim_{N \to \infty} \mu\left(\bigcup_1^N B_i\right) = \lim_{N \to \infty} \mu'\left(\bigcup_1^N B_i\right) = \mu'(B)
\]
by continuity of measure from below.

\[
\mu(A) \leq \mu(B) = \mu'(B) = \mu'(A) + \mu'(B \setminus A) \leq \mu'(A) = \varepsilon.
\]
\item
Assume $\mu(A) = \infty.$

Since $\mu_0$ is $\sigma$-finite, $X = \bigcup_1^\infty X_n, X_n \in \mathcal{A}_0, \mu_0(X_0) < \infty.$
Replacing $X_n$ by $X_1 \cup \ldots \cup X_n$, we may assume $X_1 \subset X_2 \subset \ldots$.

\[\forall n \in N, \mu(A \cap X_n) < \infty \implies \mu(A \cap X_n) \leq \mu'(A \cap X_n).\]

Hence
\[
\mu(A) = \lim_{N \to \infty} \mu(A \cap X_n) \leq \lim_{N \to \infty} \mu'(A \cap X_n) = \mu'(A). \qedhere
\]
\end{enumerate}
\end{itemize}
\end{proof}

\section{Borel Measures on \ttlmath{\R}{Reals}}
\begin{definition}
$F: \R \to \R$ is an \emph{increasing} function if $F(x) \leq F(y)$ for $x < y$.
$F: \R \to \R$ is increasing and right-continuous $\implies F$ is distribution function. 
\end{definition}

\begin{example}\fnl
\begin{itemize}
\item
$F(x) = \begin{cases}
1, & x \geq 0 \\
0, & x < 0.
\end{cases}$
\item
$\Q = \{r_1, r_2, \ldots\}, F_n(x) = \begin{cases}
1 & x \geq r_n \\
0 & x < r_n
\end{cases}$. $\displaystyle F(x) = \sum_{n=1}^\infty \frac{F_n(x)}{2^n}$ is a distribution function.
\end{itemize}
\end{example}

\fancyem{Note} If $F$ is increasing, $F(\infty) \defeq \lim_{x \to \infty} F(x), F(-\infty) \defeq \lim_{x \to -\infty} F(x)$ exists in $[-\infty, \infty]$.

In probability theory, cumulative distribution function (CDF) is a distribution function with $F(\infty) = 1$ and $F(-\infty) = 0$.

There are distributions \cite[Ch.9]{follandRealAnalysisModern1999}, but these are different from \emph{distribution functions}.

\begin{definition}
Suppose $X$ a topological space. $\mu$ on $(X, \mathcal{B}(X))$ is called \emph{locally finite} is $\mu(K) < \infty$ for any compact set $K \subset X$.
\end{definition}

\begin{lemma}
Let $\mu$ be a locally finite Borel measure on $\R \implies$
\[
F_\mu(x) = \begin{cases}
\mu((0, x]), & x > 0 \\
0, & x = 0 \\
- \mu((x, 0]), & x < 0
\end{cases} \text{ is a distribution function.}
\]
\end{lemma}
\begin{proof}
DIY. Use continuity of measure.
\end{proof}

\begin{definition}
$h$-intervals are $\emptyset, (a, b], (a, \infty), (-\infty, b], (\infty, \infty)$.
\end{definition}

\begin{lemma}
Let $\mathcal{H}$ be the collections of finite disjoint unions of $h$-intervals. Then $\mathcal{H}$ is an algebra on $\R$.
\end{lemma}
\begin{proof}
DIY.
\end{proof}

\begin{proposition}[Distribution function defines a pre-measure]
Let $F: \R \to \R$ be a distribution function. For an $h$-interval $I$, define
\[
\ell(I) = \ell_F(I) = \begin{cases}
0,  & I = \emptyset \\
F(b) - F(a), & I = (a, b] \\
F(\infty) - F(a), & I = (a, \infty) \\
F(b) - F(\infty), & I = (-\infty, b] \\
F(\infty) - F(-\infty), & I = (-\infty, \infty). 
\end{cases}
\]
Define $\mu_0 = \mu_{0, F}: \mathcal{H} \to [0, \infty]$ by
\[
\mu_0(A) \defeq \sum_{k=1}^N \ell(I_k) \quad \text{if } A = \bigcup_{k=1}^N I_k, \text{ finite disjoint union of $h$-intervals.}
\]
Then $\mu_0$ is a pre-measure.
\end{proposition}
\begin{proof}
\begin{enumerate}
\item $\mu_0$ is well-defined.
\item $\mu_0$ is finite additive.
\item $\mu_0$ is countable additive within $\mathcal{H}$.

Suppose $A \in \mathcal{H}$ and $A = \bigcup_1^\infty A_i$ a disjoint union, $A_i \in \mathcal{H}$. It is enough to consider the case $A = I, A_k = I_k$ all $h$-intervals. (Why?)

Focus on the case $I = (a, b]$: (HW: check other cases)\\
We have
\[(a, b] = \bigcup_1^\infty(a_n, b_n], \text{ a disjoint union}.\]
Check
\[F(b) - F(a) \stackrel{\text{?}}{=} \sum_1^\infty(F(b_n) - F(a_n))\]
$(a, b] \supset \bigcup_1^N (a_n, b_n] \implies F(b) - F(a) \geq \sum_1^N F(b_n) - F(a_n), \forall N \in \N$. (Arranging them in decreasing order) Take $N \to \infty$ we have
\[F(b) - F(a) \geq \sum_1^\infty(F(b_n) - F(a_n)).\]

Since $F$ is right-continuous, $\exists a' > a \st F(a') - F(a) < \varepsilon$.
For each $n \in \N$, $\exists b'_n > b_n \st F(b'_n) - F(b_n) < \frac{\varepsilon}{2^n}$. 
\begin{align*}
& \implies [a', b] \subset \bigcup_1^\infty(a_n, b'_n) \\
& \implies \exists N \in \N \st [a', b] \subset \bigcup_1^n (a_n, b'_n)\\
& \implies F(b) - F(a') \leq \sum_1^N F(b'_n) - F(a_n) \\
& \implies F(b) - F(a) \leq F(b) - F(a') + \varepsilon \leq \sum_1^\infty(F(b'_n)-F(a_n)) + \varepsilon \\
& \quad \quad \leq \sum_1^\infty\left(F(b_n)-F(a_n) + \frac{\varepsilon}{2^n}\right) + \varepsilon \qedhere
\end{align*}
\end{enumerate}
\end{proof}

Once we have this pre-measure, HK theorem allows us to extended it to a measure.

\begin{theorem}[Locally finite Borel measures on $\R$]\fnl
\begin{enumerate}
\item $F: \R \to \R$ is a distribution function $\implies \exists!$ locally finite Borel measure $\mu_F$ on $\R$ satisfying $\mu_F((a, b]) = F(b) - F(a), \forall a, b, a < b$.
\item Suppose $F, G: \R \to \R$ are distribution functions. Then, $\mu_F = \mu_G$ on $\mathcal{B}(\R)$ if and only if $F - G$ is a constant function.
\end{enumerate}
\end{theorem}
\begin{proof}
HW
\end{proof}

\section{Lebesgue-Stieltjes Measures on \ttlmath{\R}{Reals}}
$F$ distribution function  $\implies \mu_F$ on Carathéodory $\sigma$-algebra $\mathcal{A}_{\mu_F}$.\\
Actually $(\mathcal{A}_{\mu_F}, \mu_F) = (\mathcal{B}(\R), \mu_F)$ (HW3).

\begin{definition}
\begin{itemize}
\item $\mu_F$ on $\mathcal{A}_{\mu_F}$ is called the Lebesgue-Stieltjes measure corresponding to $F$.
\item Special case: $F(x) = x \implies$ Lebesgue measure $(\mathcal{B}, m)$.
\end{itemize}
\end{definition} 

\begin{example}\fnl
\begin{enumerate}
\item $\mu_F((a, b]) = F(b) - F(a)$. $F$ is right-continuous and increasing $\implies F(x_-) \leq F(x) = F(x_+)$.

(HW) $\mu_F(\{a\}) = F(a) - F(a_-), \mu_F([a, b]) = F(b) - F(a_-), \mu_F((a, b)) = F(b_-) - F(a).$

\item
\[F(x) = \begin{cases}
1 & x \leq 0 \\ 0 & x < 0
\end{cases} \implies \mu_F(\{0\}) = 1, \mu_F(\R) = 1, \mu_F(\R \setminus \{0\}) = 0.
\]
$\mu_F$ is the Dirac measure at $0$.

\item
\begin{multline*}
\Q = \{r_1, r_2, \ldots\},\ F(x) = \sum_{n=1}^\infty \frac{F_n(x)}{2^n},\ F_n(x) = \begin{cases}
1 & x \leq r_n \\ 0 & x < r_n
\end{cases} \\
\implies \mu_F(\{v\}) > 0, \forall v \in \Q,\ \mu_F(\R \setminus \Q) = 0.
\end{multline*}

\item 
If $F$ is continuous at $a, \mu_F(\{a\}) = 0$.

\item
$F(x) = x \implies m((a, b])) = m((a, b)) = m([a, b]) = b - a.$

\item
$F(x) = e^x, \implies \mu_F((a, b]) = \mu_F((a, b)) = e^b - e^a$
\end{enumerate}
(a), (b) are examples of discrete measure.
\end{example}

\begin{example}
[Middle thirds Cantor set $\mathcal{C} = \bigcup_{n=1}^\infty K_n$]\fnl

$\mathcal{C}$ is uncountable set with $m(\mathcal{C}) = 0.$
\[x \in \mathcal{C} \implies x = \sum_{n=1}^\infty \frac{a_n}{3^n}, a_n \in \{0, 2\}.\]

We are interested in the Cantor function $F$.
\end{example}

\begin{example}
Cantor function $F$ is continuous and increasing.
This defines the Cantor measure $\mu_F(\R \setminus \mathcal{C}) = 0, \mu_F(\mathcal{C}) = 1, \mu_F(\{a\}) = 0.$ Compare with Lebesgue measure $m(\R \setminus \mathcal{C}) = \infty > 0, \mu(\mathcal{C}) = 0, m(\{a\}) = 0$.
\end{example}

\section{Regularity Properties of Lebesgue-Stieltjes Measures}
\begin{lemma}\label{le:LSreal}
$\mu$ is Lebesgue-Stieltjes measure on $\R \implies$ 
\begin{align*}
    \mu(A) & = \inf\left\{\sum_1^\infty\mu((a_i, b_i])\ \biggr\vert\ \bigcup_1^\infty(a_i, b_i] \supset A\right\} \\
           & =\inf\left\{\sum_1^\infty\mu((a_i, b_i))\ \biggr\vert\ \bigcup_1^\infty(a_i, b_i) \supset A\right\}
\end{align*}
\end{lemma}
\begin{proof}
	Using the continuity of measure.
\end{proof}

\begin{theorem}
	$\mu$ is a Lebesgue-Stieltjes measure. Then $\forall A \in \mathcal{A}_\mu$,
	\begin{enumerate}
		\item (outer regularity) \[
			\mu(A) = \inf\{\mu(O) \mid \text{open } O \supset A\}.	
		\]
		\item (inner regularity) \[
			\mu(A) = \sup\{\mu(K) \mid \text{compact } K \subset A\}.
		\]
	\end{enumerate}
\end{theorem}
\begin{proof}
	\begin{enumerate}
		\item Followed from \ref{le:LSreal}.
		\item Let $s = \sup\{\ldots\}$. 
		Monotonicity $\implies \mu(A) \geq s$.
		\begin{itemize}
			\item ($A$ bounded)
			$\overline{A} \in \mathcal{B}(\R) \subset \mathcal{A}_\mu, \overline{A}$ bounded $\implies \mu(\overline{A}) < \infty$.

			Fix $\varepsilon > 0$. By 1, $\exists$ open $O \supset \overline{A} \setminus A, \mu(O) - \mu(\overline{A} \setminus A) = \mu(O \setminus (\overline{A} \setminus A))\leq \varepsilon$.

			Let $K = \underbrace{A \setminus O}_{K \subset A} = \underbrace{\overline{A} \setminus O}_{\text{compact}}$. Show that $\mu(K) \geq \mu(A) - \varepsilon$.

			\item ($A$ unbounded but $\mu(A) < \infty$)
			We have
			\[
				A = \bigcup_1^\infty A_n,\ A_n = A \cap [-n, n],\ A_1 \subset A_2 \subset \ldots	
			\]
			Hence \[
				\lim_{n \to \infty} \mu(A_n) = \mu(A) < \infty.
			\]

			\item ($\mu(A) = \infty$)
			\[
				\lim_{n \to \infty} \mu(A_n) = \mu(A) = \infty.
			\]
			Fix $L > 0$. $\exists N \st \mu(A_N) \geq L$. \qedhere
		\end{itemize}
	\end{enumerate}
\end{proof}

\begin{definition}
	Suppose $X$ a topological space.

	A \emph{$G_\sigma$-set} is $\displaystyle G = \bigcap_1^\infty O_i$, $O_i$ open. 
	An \emph{$F_\sigma$-set} is $\displaystyle F = \bigcup_1^\infty F_i$, $F_i$ closed.
\end{definition}
\begin{theorem}
	Suppose $\mu$ a LS measure. Then the following statements are equivalent:
	\begin{enumerate}
		\item $A \in \mathcal{A}_\mu$.
		\item $A = G \setminus M$, $G$ is a $G_\sigma$-set, and $M$ is $\mu$-null.
		\item $A = F \cup N$, $F$ is an $F_\sigma$-set, and $N$ is $\mu$-null.
	\end{enumerate}
\end{theorem}
\begin{proof}
	(b) $\implies$ (a) and (c) $\implies$ (a) are clear.

	\begin{itemize}
		\item (a) $\implies$ (c) \fnl
		\begin{enumerate}[(i)]
			\item Assume $\mu(A) < \infty$. By inner regularity,
			\[
				\forall n \in \N, \exists \text{ compact } K_n \subset A \st \mu(K_n) + \frac{1}{n} \geq \mu(A).
			\]
			Let $F = \bigcup_1^\infty K_n$. Then $N = A \setminus F$ is $\mu$-null.
	
			\item 
			Assume $\mu(A) = \infty$.
			We construct
			\[
				A = \bigcup_{k \in \Z}A_k, A_k = A \cap (k, k+1].	
			\]
			By (i), $\forall k \in \Z, A_k = F_k \cup N_k$. Hence
			\[
				A = \underbrace{\left(\bigcup_k F_k\right)}_{F\sigma} \cup \underbrace{\left(\bigcup_k N_k\right)}_{\text{$\mu$-null}}.
			\]
		\end{enumerate}
		
		\item (a) $\implies$ (b)
		\[A^c = F \cup N, A = F^c \cup N^c = F^c \setminus N. \qedhere\]
	\end{itemize}
\end{proof}

\begin{proposition}
	Suppose $\mu$ a LS measure, $A \in \mathcal{A}_\mu$, $\mu(A) < \infty$. Then\[
		\forall \varepsilon > 0, \exists I = \bigcup_1^{N = N(\varepsilon)} I_i, \text{ disjoint open intervals} \st \mu(A \triangle I) \leq \varepsilon.	
	\]
\end{proposition}
\begin{proof}
	DIY - use outer regularity.
\end{proof}

Properties of Lebesgue measure
\begin{theorem}
	\[A \in \mathcal{L} \implies A + s \in \mathcal{L}, rA \in \mathcal{L}, \forall r, s \in \R.\]
	In addition, $m(A + r) = m(A)$ and $m(rA) = rm(A)$.
\end{theorem}
\begin{proof}
	DIY.
\end{proof}

\begin{example}\fnl
	\begin{enumerate}
		\item $\Q = \{r_1\}_{i=1}^\infty$, which is dense in $\R$.
		Let $\varepsilon > 0$ and \[
			O = \bigcup_{i=1}^\infty\left(r_i - \frac{\varepsilon}{2^i}, r_i + \frac{\varepsilon}{2^i}\right).
		\] $O$ is open and dense in $\R$.
		We have
		\[
			m(O) \leq \sum_{i=1}^\infty \frac{2\varepsilon}{2^i} = 2\varepsilon, \partial O = \overline{O} \setminus O, m(O) = \infty.
		\]

		\item $\exists$ uncountable set $A$ with $m(A) = 0$.
		\item $\exists A$ with $m(A) > 0$, but $A$ contains no non-empty open interval.
		\item $\exists A \notin \mathcal{L}$ that is Vitali set.
		\item $\exists A \in \mathcal{L} \setminus \mathcal{B}(\R)$. We will deal with that later.
	\end{enumerate}
\end{example}

\chapter{Integration}
\section{Measurable Functions}
\begin{definition}
	Suppose $(X, \mathcal{A})$, $(Y, \mathcal{B})$ two measurable spaces. 
	$f: X \to Y$ is $(\mathcal{A}, \mathcal{B})$-measurable if 
	\[\forall B \in \mathcal{B}, f^{-1}(B) \in \mathcal{A}.\] 
\end{definition}

\begin{lemma}\label{le:clever}
	Suppose $\mathcal{B} = \langle\mathcal{E}\rangle$. Then \[f: X \to Y \text{ is }(\mathcal{A}, \mathcal{B}) \text{-measurable } \iff \forall E \in \mathcal{E}, f^{-1}(E) \in \mathcal{A}.\]
\end{lemma}
\begin{proof}
	\begin{itemize}
		\item [$\implies$] clear
		\item [$\impliedby$] Let $\mathcal{D} = \{E \subset Y \mid f^{-1}(E) \in \mathcal{A}\}$. We have $\mathcal{E} \subset \mathcal{D}$ by assumption. In addition $\mathcal{D}$ is a $\sigma$-algebra $\implies \langle\mathcal{E}\rangle \subset \mathcal{D}$.\qedhere
	\end{itemize} 
\end{proof}

\begin{definition}
	Suppose $(X, \mathcal{A})$ a measurable space.
	\[
	\left.\begin{array}{l}
		f: X \to \R \\
		f: X \to \overline{\R} = [-\infty, \infty] \\
		f: X \to \C
	\end{array}	\right\rbrace \text{ is $\mathcal{A}$-measurable if }
	\left\lbrace\begin{array}{l}
		f \text{ is $(\mathcal{A}, \mathcal{B}(\R))$-measurable} \\
		f \text{ is $(\mathcal{A}, \mathcal{B}(\overline{\R}))$-measurable} \\
		\re f, \im f: X \to \R \text{ are $\mathcal{A}$-measurable.}
	\end{array}\right.
	\]
	Here $\mathcal{B}(\overline{\R}) = \{E \subset \overline{\R} \mid E \cap R \in \mathcal{B}(\R)\}$.
\end{definition}

\begin{lemma}
	Suppose $f: X \to \R$. Then the followings are equivalent:
	\begin{enumerate}
		\item $f$ is $\mathcal{A}$-measurable
		\item $\forall a \in \R, f^{-1}((a, \infty)) \in \mathcal{A}$.
		\item $\forall a \in \R, f^{-1}([a, \infty)) \in \mathcal{A}$.
		\item $\forall a \in \R, f^{-1}((-\infty, a)) \in \mathcal{A}$.
		\item $\forall a \in \R, f^{-1}((-\infty, a]) \in \mathcal{A}$.
	\end{enumerate}
	For $f: X \to \overline{\R}$, change the interval to include $-\infty$ and $\infty$.
\end{lemma}
\begin{proof}
	By \ref{le:clever}.
\end{proof}



\begin{example}
	$\mathcal{A} = \mathcal{P}(X) \implies$ every function is $\mathcal{A}$ measurable.

	$\mathcal{A} = \{\emptyset, X\} \implies$ only $\mathcal{A}$ functions are constant functions.
\end{example}

\fancyem{Properties} Suppose $f, g: X \to \R$, $\mathcal{A}$-measurable functions.
\begin{enumerate}
	\item $\phi: \R \to \R$, $\mathcal{B}(\R)$ measurable (i.e. Borel measurable) $\implies \phi \circ f: X \to \R$ is $\mathcal{A}$-measurable.
	\item $-f, 3f, f^2, |f|$ are $\mathcal{A}$-measurable, $\frac{1}{f}$ is $\mathcal{A}$-measurable if $f(x) = 0, \forall x \in X$.
	\item $f + g$ is $\mathcal{A}$-measurable
	\[
		(f + g)^{-1}((a, \infty)) = \bigcup_{r \in \Q}\left(f^{-1}((r, \infty)) \cap g^{-1}((a-r,\infty))\right).
	\]
	\item $fg$ is $\mathcal{A}$-measurable \[
		f(x)g(x) = \frac{1}{2}\left((f(x)+g(x))^2 - f(x)^2 - g(x)^2\right).
	\]
	\item $(f \wedge g)(x) = \min\{f(x), g(x)\}$, $(f \vee g)(x) = \max\{f(x), g(x)\}$ are $\mathcal{A}$-measurable.
	\item $f_n: X \to \overline{\R}$ are a sequence of $\mathcal{A}$-measurable functions $\implies$
	\[
		\sup f_n, \inf f_n, \limsup_{n \to \infty} f_n, \liminf_{n \to \infty} f_n \text{ are $\mathcal{A}$-measurable.}
	\]
	\item If $f(x) = \lim_{n \to \infty}f_n(x)$ converges for every $x \in X$, then $f$ is measurable.
\end{enumerate}

\begin{example}
	Suppose $f: \R \to \R$ is continuous. Then $f$ is Borel measurable $\implies f$ is Lebesgue measurable. (Preimage of an open set of a continuous function is open.)
\end{example}
\begin{definition}
	For $f: X \to \overline{\R}$, let $f^+ = f \vee 0$, $f^- = (-f) \vee 0$.
\end{definition}
\fancyem{Note} $\supp f^+ \cap \supp f^- = \emptyset$.
$f(x) = f^+(x) - f^-(x)$. $f$ is $\mathcal{A}$-measurable $\iff f^+, f^-$ measurable.

\begin{definition}
	For $E \subset X$, characteristic (indicator) funtion of $E$
	\[
		\chi_E(x) = 1_E(x) = \begin{cases}
			1 & x \in E \\
			0 & x \in E^c.
		\end{cases}	
	\]
	$1_E$ is $\mathcal{A}$-measurable $\iff E \in \mathcal{A}$.
\end{definition}

\begin{definition}
	Suppose $(X, \mathcal{A})$ a measurable space. A \emph{simple function} $\phi: X \to \C$ that is $\mathcal{A}$-measurable and takes only finitely many values.
\end{definition}
\[
	\phi(X) = \{c_1, \ldots, c_N\}, c_i \neq \pm \infty, E_i = \phi^{-1}(c_i) \in \cA \implies \phi = \sum_{i=1}^N c_i 1_{E_i}.
\]

\begin{theorem}
	Suppose $(X, \cA)$ a measurable space and $f: X \to [0, \infty]$. Then the followings are equivalent:
	\begin{enumerate}
		\item $f$ is $\cA$-measurable.
		\item $\exists$ simple functions $0 \leq \phi_1(x) \leq \phi_2(x) \leq \ldots \leq f(x)$ such that \[\lim_{n \to \infty} \phi_n(x) = f(x),\ \forall x \in X.\] ($f$ is the pointwise upward limit of simple functions.)
	\end{enumerate}
\end{theorem}
\begin{proof}
	\begin{itemize}
		\item (b) $\implies$ (a) is easy: $\displaystyle f(x) = \sup_{n \in \N} \phi_n(x)$.
		\item (a) $\implies$ (b): suppose $f$ is $\cA$-measurable.
		
		Fix $n \in \N$. Let $F_n = f^{-1}([2^n, \infty]) \in \cA$. For
		\[0 \leq k \leq 2^{2n} - 1,\ E_{n, k} = f^{-1}\left(\left[\frac{k}{2^n}, \frac{k+1}{2^n}\right]\right) \in \cA.\]

		Let $\displaystyle \phi_n(x) = \sum_{k=0}^{2^{2n} - 1} 1_{E_{n, k}} + 2^n1_{F_n}$.

		This shows that \begin{itemize}
			\item $\displaystyle 0 \leq \phi_1(x) \leq \phi_2(x) \leq \ldots \leq f(x),\ \forall x \in X$.
			\item $\displaystyle \forall x \in X \setminus F_n, 0 \leq f(x) - \phi_n(x) \leq \frac{1}{2^n}$.
		\end{itemize}
		Since $F_1 \supset F_2 \supset \ldots$ and $\displaystyle \bigcap_1^\infty F_n = f^{-1}(\{\infty\})$, we have \begin{itemize}
			\item $\displaystyle x \in f^{-1}([0, \infty)) = X \setminus \left(\bigcap_1^\infty F_n\right) \implies \lim_{n \to \infty} \phi_n(x) = f(x)$.
			\item $\displaystyle x \in f^{-1}(\{\infty\}) = \bigcap_1^\infty X_n \implies \phi_n(x) \geq 2^n \implies \lim_{n \to \infty} \phi_n(x) = \infty = f(x)$. \qedhere 
		\end{itemize}
	\end{itemize}
\end{proof}

\begin{corollary}
	If $f$ is bounded on a set $A \subset \R$ (i.e. $\exists L > 0 \st |f(x)| \leq L,\ \forall x \in A)$ then $\phi_n \to f$ uniformly on $A$.
\end{corollary}
\begin{proof}
	DIY.
\end{proof}

\begin{corollary}
	$f: X \to \C$, measurable function $\iff \exists$ simple functions $\phi_n: X \to \C \st 0 \leq |\phi_1| \leq |\phi_2| \leq \ldots \leq |f|$ and $\phi_n$ converges to $f$ pointwise. (Again, if $f$ is bounded the convergence can be uniform.)   
\end{corollary}

\section{Integration of Nonnegative Functions}
\begin{definition}\label{def:nonneg}
	Suppose $(X, \cA, \mu)$ a measure space and $\phi = \sum_{i=1}^N c_i1_{E_i}: X \to [0, \infty]$ a simple function.
	Let \[
		\int \phi = \int \phi \df\mu = \int_X \phi \df \mu = \sum_1^N c_i\mu(E_i).	
	\]
\end{definition}

\begin{proposition}
	Suppose $\phi, \psi \geq 0$ are simple functions. Then, 
	\begin{itemize}
		\item \ref{def:nonneg} is well-defined.
		\item $\displaystyle \int c\phi = c \int \phi, c \in [0, \infty)$.
		\item $\displaystyle \int (\phi + \psi) = \int \phi + \int \psi$.
		\item $\displaystyle \phi(x) \geq \psi(x),\ \forall x \implies \int \phi \geq \int \psi$.
		\item $\displaystyle \nu(A) = \int_A \phi \df \mu$ is a measure on $(X, \cA)$.
	\end{itemize}
\end{proposition}
\begin{proof}
	DIY.
\end{proof}

\begin{definition}
	Suppose $(X, \cA, \mu), f: X \to [0, \infty]$ is $\cA$-measurable. 

	Define \[
		\int f = \int f \df\mu = \sup\left\lbrace \int \phi \mid 0 \leq \phi \leq f, \phi \text{ simple} \right\rbrace.	
	\]
\end{definition}
\begin{proposition}\fnl
	\begin{itemize}
		\item If $f$ is a simple function then two definitions are the same.
		\item $\displaystyle \int cf = c\int f$.
		\item $\displaystyle f \geq g \geq 0 \implies \int f \geq \int g$.
		\item $\displaystyle \int f + g = \int f + \int g$. (A bit harder to check)
	\end{itemize}
\end{proposition}
\begin{theorem}[Monotone convergence theorem]
	Suppose $(X, \cA, \mu)$ a measure space and \begin{itemize}
		\item $f: X \to [0, \infty]$ is $\cA$-measurable, $\forall n \in \N$.
		\item $0 \leq f_1(x) \leq \ldots $.
		\item $\displaystyle \lim_{n \to \infty} f_n(x) = f(x)$.
	\end{itemize}
	Then
	\[
		\int f = \lim_{n \to \infty} \int f_n.	
	\]
\end{theorem}
\begin{proof}
	Note that $\lim_{n \to \infty} f_n(x)$ converges $\forall x \in X$ and $\lim_{n \to \infty} f_n(x)$ converges.
	\begin{itemize}
		\item $\displaystyle f_n \leq f \implies \int f_n \leq \int f \implies \lim_{n \to \infty} \int f_n \leq \int f$.
		\item Fix simple function $0 \leq \phi \leq f$. Enough to show that $\displaystyle \lim_{n \to \infty} \int f_n \geq \int \phi$.
		
		Now fix $\alpha \in (0, 1)$. Enough to prove that $\displaystyle \lim_{n \to \infty} \int f_n \geq \alpha \int \phi$.

		Let $A_n = \{x \mid f_n(x) \geq \alpha \phi(x)\}$. \begin{itemize}
			\item $A_n \in \cA$.
			\item $A_1 \subset A_2 \subset \ldots$
			\item $\displaystyle \bigcup_{n=1}^\infty A_n = X$. (check!)
		\end{itemize}
		So we have
		\[
			\int f_n \geq \int f_n1_{A_n} \geq \int \alpha \phi 1_{A_n}	= \alpha \nu(A_n)
		\] where $\nu(A) = \int_A \phi$ is a measure.
		\[\implies \lim_{n \to \infty} \int f_n \geq \lim_{n \to \infty} \nu(A_n) = \alpha \nu(x) = \alpha \int \phi. \qedhere\]
	\end{itemize}
\end{proof}
\begin{corollary}
	$f, g \geq 0$ measurable $\displaystyle \implies \int f + g = \int f + \int g$. 
\end{corollary}
\begin{proof}
	$\exists$ simple functions $0 \leq \phi_1 \leq \phi_2 \leq \ldots, \phi_n \to f$ pointwise and $0 \leq \psi_1 \leq \psi_2 \leq \ldots, \psi_n \to g$ pointwise.

	By MCT, we have
	\[
	 	\int (f + g) =  \lim_{n \to \infty} \int (\phi_n + \psi_n) = \lim_{n \to \infty} \int \phi_n + \int \psi_n = \int f + \int g. \qedhere
	\]
\end{proof}

\begin{corollary}[Tonelli's theorem for series and integrals]
	Given $s_n \geq 0, \forall n \in \N$ measurable functions.
	Then
	\[
		\int \sum_{n=1}^\infty	s_n = \sum_{n=1}^\infty \int s_n.
	\]
\end{corollary}
\begin{proof}
	Let $f_N = \sum_{n=1}^N s_n, 0 \leq f_1 \leq f_2 \leq \ldots$.
	\[
		\lim_{N \to \infty}f_N(x) = \sum_{n=1}^\infty s_n(x)\]
	By MCT, we have \[
		\lim_{N \to \infty} \sum_1^N s_n = \sum_1^\infty s_n	
	\]
\end{proof}
\begin{theorem}[Fatou's lemma]
	Suppose $f_n \geq 0$ measurable. Then \[\int \liminf_{n \to \infty} f_n \leq \liminf_{n \to \infty}\int f_n.\]
\end{theorem}
Recall that \[
\liminf_{n \to \infty} f_n \defeq \lim_{k \to \infty} \inf_{n \geq k} f_n = \sup_{k \in \N} \inf_{n \geq k} f_n,
\]
and
\[
\lim_{n \to \infty} a_n \text{ exists} \iff \limsup_{n \to \infty} a_n = \liminf_{n \to \infty} a_n.
\]
\begin{proof}
	Let $g_k = \inf_{n \geq k} f_n \implies s_k$ measurable and $0 \leq g_1 \leq g_2 \leq \ldots$. By MCT, we have \[
		\int \liminf_{n \to \infty} = \int\lim_{k \to \infty} s_k = \lim_{k \to \infty} \int s_k = \lim_{k \to \infty} \int \inf_{n \geq k} f_n
	\]
	\begin{align*}
		& \inf_{n \geq k} f_n \leq f_m, \forall m \geq k \\
		\implies & \int \inf_{n \geq k}f_n \leq \int f_m, \forall m \geq k \\
		\implies & \int \inf_{n \geq k}f_n \leq \inf_{m \geq k}\int f_m
	\end{align*}

\end{proof}

\begin{example}
	Suppose $(\R, \mathcal{L}, m)$
	\begin{enumerate}
		\item (escape to horizontal infinity) $f_n = 1_{(n, n+1)}$.
		
		We see that $f_n \to 0 = f$ pointwise and $\int f_n = 1, \forall n, \int f = 0$.

		\item (escape to width infinity) $f_n = \frac{1}{n}1_{(0, n)}$.
		\item (escape to vertical infinity) $f_n = n1_{(0, 1/n)}$.
	\end{enumerate}	
\end{example}

\begin{lemma}[Markov's inequality]
	$f \geq 0$ is measurable $\implies$ \[\forall c \in (0, \infty),\ \mu\left(\left\lbrace x \mid f(x) \geq c \right\rbrace\right) \leq \frac{1}{c}\int f.\]
\end{lemma}
\begin{proof}
	Let $E = \left\lbrace x \mid f(x) \geq c \right\rbrace$. Then \[f(x) \geq c1_{E}(x) \implies \int f \geq c\int 1_E = c \mu(E). \qedhere\] 
\end{proof}
\begin{proposition}\label{prop:zeroint}
	Suppose $f \geq 0$ measurable. Then $\int f = 0 \iff f = 0$ almost everywhere (a.e.) \[\int f \df\mu = \mu(A) = 0,\ A = \{x \mid f(x) > 0\} = f^{-1}((0, \infty])\]
\end{proposition}
\begin{proof}
	\begin{enumerate}
		\item Assume $f = \phi$ a simple function. We may assume \[\phi = \sum_{i=1}^N c_i1_{E_i},\ c_i \in (0, \infty),\ E_i\text{'s are disjoint.}\]
		\begin{align*}
			& \int \phi = \sum_{i=1}^N c_i \mu(E_i) = 0 \\
		\iff & \mu(E_1) = \ldots = \mu(E_N) = 0 \\
		\iff & \mu(A) = 0,\ A = \bigcup_{i=1}^N E_i.
		\end{align*}

		\item General $f \geq 0$. \begin{enumerate}
			\item Assume $\mu(A) = 0$ (i.e. $f = 0$ a.e.)
			
			Let $0 \leq \phi \leq f, \phi$ is simple.
			\begin{align*}
				& \implies \phi(x) = 0,\ \forall x \in A^c \\
				& \implies \phi = 0 \text{ a.e.} \\
				& \implies \int \phi = 0
			\end{align*}
			Then $\int f = 0$ by the definition of $\int f$.

			\item Assume $\inf f = 0$. Let $A_n = f^{-1}\left(\left[\frac{1}{n}, \infty\right]\right)$
			\begin{align*}
				\implies & A_1 \subset A_2 \subset \ldots \\
				& \bigcup_1^\infty A_n = f^{-1}\left(\bigcup_1^\infty\left[\frac{1}{n}, \infty\right]\right) = f^{-1}((0, \infty)) = A \\
				& \mu(A_n) = \mu\left(\left\lbrace x \mid f(x) \geq \frac{1}{n}\right\rbrace\right) \leq n \int f = 0 \\
				\implies & \mu(A) = \lim_{n \to \infty} \mu(A_n) = 0
			\end{align*}
			by the continuity of measure from below. \qedhere
		\end{enumerate}
	\end{enumerate}
\end{proof}

\begin{corollary}
	$f, g \geq 0$ are measurable, $f = g$ a.e. $\implies \int f = \int g$.
\end{corollary}
\begin{proof}
	Let $A = \{x \mid f(x) \geq g(x)\}$. $A$ is measurable (why?). By assumption $\mu(A) = 0$. Hence $f1_A = 0$ a.e.
	\begin{align*}
		\int f & = \int f(1_A + 1_{A^c}) \\
		& = \int f1_A + \int f1_{A^c} \\
		& = \int f1_{A^c} \\
		& = \int g1_{A^c} = \int g1_{A} + \int g1_{A^c} = \int g. \qedhere
	\end{align*}
\end{proof}

\begin{corollary}
	$f_n \geq 0$ measurable. Then 
	\begin{enumerate}
		\item \[\left.\begin{array}{r}
			0 \leq f_1 \leq f_2 \leq  \ldots \leq f\ a.e. \\
			\lim_{n \to \infty} f_n = f\ a.e.
		\end{array}\right\} \implies \lim_{n \to \infty} f_n = \int f.\]
		\item \[
			\lim_{n \to \infty} f_n = f\ a.e \implies \int f \leq \liminf_{n \to \infty} \int f_n.	
		\]
	\end{enumerate}
\end{corollary}

\section{Integration of Complex Functions}
\epigraph{I was afraid that you are bored.}{--- \textup{Jinho Baik on homework}}
\begin{definition}
	$(X, \mathcal{A}, \mu)$ measure space. \begin{itemize}
		\item $f:X \to \overline{\R}$ or $f: X \to \C$ measurable functions is called \emph{integrable} if $\int |f| < \infty$. Then \[\int f = \int f^+ - \int f^- \text{ or } \int f = \int u^+ - \int u^- + i\left(\int v^+ - \int v^-\right).\]
		\item Suppose $f: X \to \overline{\R}$. Define \[\int f = \begin{cases}
			\infty & \displaystyle \int f^+ = \infty, \int f^- < \infty, \\[1em]
			-\infty & \displaystyle  \int f^+ < \infty, \int f^- = \infty.
		\end{cases}\]
	\end{itemize}
\end{definition}

\begin{lemma}
	Suppose $f, g: x \to \overline{\R} \to \C$ integrable.
	Assume $f(x) + g(x)$ is well-defined $\forall x \in X$. (i.e. $\infty + (-\infty)$, $-\infty + \infty$ do not occur) \begin{enumerate}
		\item $f + g$, $cf, c \in \C$ are integrable.
		\item $\displaystyle \int f + g = \int f + \int g$.
		\item $\displaystyle \left|\int f\right| \leq \int |f|$. (This is essentially triangle inequality.)
	\end{enumerate}
\end{lemma}
\begin{proof}
	Check \cite[p.53]{follandRealAnalysisModern1999}.
\end{proof}
\begin{lemma}
	$(X, \mathcal{A}, \mu)$ measure space and $f$ \emph{integrable} function on $X$.
	\begin{enumerate}
		\item $f$ is finite a.e. (i.e. $\{x \in X: |f(x)| = \infty\}$ is a null set)
		\item The set $\{x \in X: f(x) \neq 0\}$ is $\sigma$-finite.
	\end{enumerate}
\end{lemma}
\begin{proof}
	HW5Q8.
\end{proof}

\begin{proposition}
	Suppose $(X, \mathcal{A}, \mu)$ a measure space.
	\begin{enumerate}
		\item If $h$ is integrable on $X$, then \[\int_E h = 0, \forall E \in \mathcal{A} \iff \int |h| = 0 \iff h = 0\text{ a.e.}\]
		\item If $f, g$ are integrable on $X$ then \[\int_E f = \int_E g, \forall E \in \mathcal{A} \iff f = g\text{ a.e.}\]
	\end{enumerate}
\end{proposition}
\begin{proof}
	\begin{enumerate}
		\item $\int |h| = 0 \iff h = 0$ is shown in \ref{prop:zeroint}.
		\[
			\int |h| = 0 \implies \left|\int_E h\right| \leq \int_E |h| \leq \int |h| = 0.
		\]
		On the other hand, assume $\int_E h = 0, \forall E \in \mathcal{A}$. $h = u + iv = u^+ - u^- + i(v^+ - v^-)$. Let $B = \{x \mid u^+(x) > 0\}$.
		\[0 = \re\int_B h = \int_B u = \int_B u^+ = \int_B u^+ + \int_{B^c} u^+ = \int u^+ \implies u^+ = 0 \text{ a.e.}\]
		Similarly, we get $u^-, v^+, v^- = 0$ a.e..
		\item follows from (a). \qedhere
	\end{enumerate}
\end{proof}
\begin{theorem}[Dominated convergence theorem]
	Suppose $(X, \mathcal{A}, \mu)$ a measure space and
	\begin{enumerate}
		\item $f_n$ \emph{integrable} on $X$, $\forall n \in \N$.
		\item $\displaystyle \lim_{n \to \infty} f_n(x) = f(x)$ a.e. (pointwise)
		\item $\exists g: X \to [0, \infty] \st$ \begin{itemize}
			\item $g$ is integrable.
			\item $|f_n(x)| \leq g(x)$ a.e., $\forall n \in \N$.
		\end{itemize}
	\end{enumerate}
	Then \[\lim_{n \to \infty} \int f_n = \int f.\]
\end{theorem}
\begin{proof}
	Let $F$ be the countable union of null sets on which (a)-(c) may fail. Modifying the def of $f_n, f, g$ on $F$ we may assume (a)-(c) hold everywhere. (b)+(c) $\implies f$ is integrable.

	We consider $\overline{\R}$-valued case only. ($\C$-valued case follows)
	\begin{align*}
		& g + f_n \geq 0, g - f_n \geq 0 \\
		\xRightarrow[]{\text{Fatou}} & \int g + f \leq \liminf_{n \to \inf} \int g + f_n,\quad \int g - f \leq \liminf_{n \to \inf} \int g - f_n  \\
		\implies & \int g + \int f \leq \int g + \liminf_{n \to \infty} \int f_n,\quad \int g - \int f \leq \int g - \limsup_{n \to \infty} \int f_n \\
		\xRightarrow[]{\int g < \infty} & \int f \leq \liminf_{n \to \infty} \int f_n,\quad - \int f \leq - \limsup_{n \to \infty} \int f_n. \\
		\implies & \int f \leq \liminf_{n \to \infty} \int f_n \leq \limsup_{n \to \infty} \int f_n \leq \int f
	\end{align*}
	So we should have \[ \int f = \liminf_{n \to \infty} \int f_n = \limsup_{n \to \infty} \int f_n. \qedhere \]
\end{proof}
Next we investigate the question:
\[\int \sum_{1}^\infty f_n \stackrel{\text{?}}{=} \sum_{1}^\infty \int f_n.\]
Tonelli: yes if $f_n \geq 0$.
Fubini:
\begin{corollary}[Fubini's theorem for series and integrals]
	\[\left.\begin{array}{r}
		f_n \text{ integrable} \\
		\displaystyle \sum_1^\infty \int |f_n| < \infty
	\end{array}\right\} \implies \int \sum_{1}^\infty f_n = \sum_{1}^\infty \int f_n.\]
\end{corollary}
\begin{proof}
	$\displaystyle G(x) = \sum_1^\infty |f_n(x)| \geq |F_N(x)|, F_N(x) = \sum_1^N f_n(x)$.
\end{proof}

\section{\ttlmath{L^1}{L-1} space}
\begin{definition}\label{def:seminorm}
	Suppose $V$ is a vector space over field $\R$ or $\C$. A \emph{seminorm}
on $V$ is $\|\cdot\|: V \to [0, \infty) \st$
\begin{itemize}
	\item $\|cv\| = |c|\|v\|, \forall v \in V, \forall c$ scalar
	\item $\|v + w\| \leq \|v\| + \|w\|$, triangle inequality
\end{itemize}
A \emph{norm} is a seminorm such that $\|v\| \iff v = 0$.
\end{definition}

\begin{lemma}
	A normed vector space is a metric space with metric $\rho(v, w) = \|v - w\|$.
\end{lemma}
\begin{proof}{(DIY)}
	\begin{itemize}
		\item $\rho(v, w) = 0 \iff \|v - w\| = 0 \iff v - w = 0 \iff v = w$.
		\item $\rho(v, w) = \norm{v - w} = \norm{-1(w - v)} = |-1|\norm{w - v} = \rho(w, v)$.
		\item $\rho(v, w) + \rho(w, z) = \norm{v - w} + \norm{w - z} \geq \norm{v - w + w - z} = \norm{v - z} = \rho(v, z)$. \qedhere
	\end{itemize}
\end{proof}
\begin{example}
	$\R^d$ with $\|x\|_p = \begin{cases}
		\displaystyle \left(\sum_{1}^d |x_i|^p\right)^{1 / p} & p \in [1, \infty) \\
		\displaystyle \max_{1 \leq i \leq d} |x_i| & p = \infty
	\end{cases}$ is a normed vector space.
	Unit ball $\{x: \|x\|_p < 1\}$.
\end{example}
All $\|\cdot\|_p$ norm induce the same topology i.e. if $U$ is open in $p$-norm then it is open in $p'$-norm. This implies that a sequence converging under $p$-norm also converges under $p'$-norm.

\fancyem{Recall} $f$ is integrable $\implies \int|f| < \infty$. $f = g $ a.e. $\implies \int f = \int g$.

\begin{definition} 
	Suppose $(X, \cA, \mu)$ a measure space.\\
	$f \in L^1(X, \cA, \mu) = L^1(X, \mu) = L^1(X) = L^1(\mu)$ means $f$ is an integrable function on $X$.
\end{definition}

\begin{lemma}
	$L^1(X, \cA, \mu)$ is a vector space with seminorm $\displaystyle \|f\|_1 = \int |f|$.
\end{lemma}

\begin{definition}
	Define $f \sim g$ if $f = g$ a.e. 
	$L^1(X, \cA, \mu)/_\sim = L^1(X, \cA, \mu)$.
	$``="$ is just a notation for convenience!
\end{definition}

With new definition we have $L^1(X, \cA, \mu)$ is a normed vector space.
$\displaystyle\rho(f, g) = \int |f - g|$.

Something interesting to discuss is what are the dense subsets of $L^1$.
\begin{theorem}\fnl
	\begin{enumerate}
		\item $\{$ integrable simple functions $\}$ is dense in $L^1(X, \cA, \mu)$ (with respect to $L^1$ metric)
		\item $(X, \cA, \mu) = (\R, \cA_\mu, \mu)$, $\mu$ is Lebesgue-Stieltjes measure $\implies$ $\{$ integrable step functions $\}$ is dense in $L^1(X, \cA, \mu)$ 
		\item $C_c(\R)$ is dense in $L^1(\R, \mathcal{L}, m)$.
	\end{enumerate}
\end{theorem}
\begin{definition}\fnl
	\begin{itemize}
		\item A step function on $\R$ is $\psi + \sum_1^N c_i 1_{I_i}$, where $I_i$ is an interval.
		\item $C_c(\R)$ is the collection of continuous functions with compact support $\supp(f) = \overline{\{x \in \R \mid f(x) \neq 0\}}$.
	\end{itemize}
\end{definition}
\begin{proof}
	\begin{enumerate}
		\item $\exists$ simple functions $0 \leq |\phi_1| \leq |\phi_2| \leq \ldots \leq |f|$, $\phi_n \to f$ pointwise $\implies \displaystyle \lim_{n \to \infty} \int |\phi_n - f| = 0$ by DCT. ($|\phi_n - f| \leq |\phi_n| + |f| \leq 2|f|$)
		\item $1_E$ approx by $\sum_1^N c_i1_{I_i}$? Regularity theorem for Lebesgue-Stieltjes measure $\implies \forall \varepsilon' > 0, \exists I = \bigcup_1^N I_i \st \mu(E \triangle I) < \varepsilon'$.
		\item Suppose $1_{(a, b)}$, $g \in C_c(\R)$. $\displaystyle \int |1_{(a, b)} - g|\df m \leq 1 \cdot \frac{\varepsilon}{2} + 1 \cdot \frac{\varepsilon}{2} = \varepsilon$. \qedhere
	\end{enumerate}
\end{proof}

\section{Riemann Integrability} 
Suppose $P = \{a = t_0 < t_1 < \ldots < t_k = b\}$ a partition of $[a, b]$. Lower Riemann sum of $f$ using $P$
\[L_P = \sum_{i = 1}^k \left(\inf_{[t_{i-1}, t_i]} f\right)(t_i - t_{i - 1})\] and upper Riemann sum
\[U_p = \sum_{i = 1}^k \left(\sup_{[t_{i-1}, t_i]} f\right)(t_i - t_{i - 1})\]
Lower Riemann integral of $f = \underline{I} = \sup_P L_P$. Upper Riemann integral of $f = \overline{I} = \inf_P U_P$.

\begin{definition}
	A \emph{bounded} function $f:[a, b] \to \R$ is called Riemann (Darboux) integrable if $\underline{I} = \overline{I}$. (If so, $\underline{I} = \overline{I} = \int_a^b f(x)\df x$.)
\end{definition}
\fancyem{Note} \begin{itemize}
	\item If $P \subset P'$, then $L_P \leq L_{P'}, U_{P'} \leq U_P$.
	\item Recall that continuous functions on $[a, b]$ are Riemann integrable on $[a, b]$.
\end{itemize}

\begin{theorem}
	Let $f: [a, b] \to \R$ be a bounded function.
	\begin{enumerate}
		\item If $f$ is Riemann integrable, then $f$ is Lebesgue measurable. (thus Lebesgue integrable) and $\displaystyle \int_a^b f(x) \df x = \int_{[a, b]}f \df m$.
		\item $f$ is Riemann integrable $\iff f$ is continuous Lebesgue a.e.
	\end{enumerate}
\end{theorem}
\begin{proof}
	$\exists$partitions $P_1 \subset P_2 \subset P_3 \subset \ldots \st L_{P_n} \nearrow \underline{I}, U_{P_n} \searrow \overline{I}$.

	Define simple (step) functions 
	\[\phi_n = \sum_{i=1}^k\left(\inf_{[t_{i - 1}, t_i]}\right)1_{(t_{i-1}, t_i]}\]
	\[\psi_n = \sum_{i=1}^k\left(\sup_{[t_{i - 1}, t_i]}\right)1_{(t_{i-1}, t_i]}\]

	Define $\phi = \sup_n \phi_n$, $\psi = \inf_n \psi_n$. Then $\phi, \psi$ are Lebesgue measurable functions.

	\fancyem{Note} \begin{itemize}
		\item $\exists M > 0 \st |\phi_n|, |\psi_n| \leq M1[a, b], \forall n \in \N$.
		\item $\int \phi_n \df m = L_{P_n}, \int \psi_n \df m = U_{P_n}$.
	\end{itemize}
	By DCT, $\displaystyle \underline{I} = \lim_{n \to \infty} \int \phi_n \df m = \int \phi \df m,  \overline{I} = \int \psi \df m$.

	Thus, $f$ is Riemann integrable $\iff \int \phi = \int \psi \iff \int (\phi - \psi) = 0 \iff \phi = \psi$ Lebesgue a.e.

	Recall that $\phi \leq f \leq \psi, \forall x \in (a, b]$. So $f = \phi$ a.e. Since $(\R, \mathcal{L}, \mu)$ is complete, $f$ is Lebesgue measurable (see HW). The second statement hence follows.
\end{proof}

\section{Modes of Convergence}
Suppose $f_n, f: X \to \C, S \subset X$.
\begin{itemize}
	\item $f_n \to f$ \emph{pointwise} on $S$: $\forall x \in S, \forall \varepsilon > 0, \exists N \in \N \st \forall n \geq \N, |f_n(x) - f(x)| < \varepsilon$.
	\item $f_n \to f$ \emph{uniformly} on $S$: $\forall \varepsilon > 0, \exists N \in \N \st \forall x \in X, \forall n \geq \N, |f_n(x) - f(x)| < \varepsilon$.
\end{itemize}
We can change $\forall \varepsilon > 0$ to $\forall k \in \N$ and bound the distance by $\frac{1}{k}$.

\begin{lemma}\label{le:conv-sets}
	Let $B_{n, k} = \{x \in X \mid |f_n(x) - f(x)| < \frac{1}{k}\}$.
	\begin{enumerate}
		\item $f_n \to f$ pointwise on $\displaystyle S \iff S \subset \bigcap_{k = 1}^\infty \bigcup_{N = 1}^\infty \bigcap_{n = N}^\infty B_{n, k}$.
		\item $f_n \to f$ uniformly on $\displaystyle S \iff \exists N_1, N_2, \ldots \in \N \st S \subset \bigcap_{k = 1}^\infty \bigcap_{n = N_k}^\infty B_{n, k}$.
	\end{enumerate}
\end{lemma}
\begin{definition}
	Suppose $(X, \cA, \mu)$ a measure space.
	\begin{enumerate}
		\item $f_n \to f$ a.e means $\exists$ null set $E \st f_n \to f$ pointwise on $E^c$.
		\item $f_n \to f$ in $L^1$ means $\displaystyle \lim_{n \to \infty} \norm{f_n - f} = 0$.
	\end{enumerate}
\end{definition}

\begin{example} $(\R, \mathcal{L}, \mu)$. $f = 0$.
	\begin{enumerate}
		\item $f_n = 1_{(n, n+1)}, f_n = \frac{1}{n}1_{(0, n)}, f_n = n1_{\left(0, \frac{1}{n}\right)}$. All of $f_n \to f$ pointwise but $\not\to f$ in $L^1$.
		\item Typewriter functions: $f_n \to f$ in $L^1$. $f_n \not\to f$ a.e.
	\end{enumerate}
\end{example}

\begin{proposition}[Fast $L^1$ convergence $\implies$ a.e. convergence]
	Suppose $(x, \cA, \mu)$ measure space. $f_n, f$ measurable function on $X$.
	\[
	\sum_1^\infty \norm{f_n-f}_1 < \infty \implies f_n \to f\ a.e.	
	\]
\end{proposition}
\begin{proof}
	\fancyem{Recall} Markov's inequality.

	Let $\displaystyle E = \bigcup_{k = 1}^\infty\bigcap_{N=1}^\infty \bigcup_{n = N}^\infty B_{n, k}^c = \{x \mid f_n(x) \not\to f(x)\}$. By Markov we have
	\begin{align*}
		& \forall k, \forall N, \mu(B_{n, k}^c) \leq k \int |f_n - f| \\
		\implies & \forall k, \mu\left(\bigcap_{n=N}^\infty B_{n, k}^c\right) \leq \sum_{n = N}^\infty k \norm{f_n - f}_1 \to 0 \text{ as } n \to 0 \\
		\implies & \forall k, \mu\left(\bigcap_{N=1}^\infty\bigcap_{n=N}^\infty B_{n, k}^c\right) = \lim_{N \to \infty} \mu\left(\bigcap_{n=N}^\infty B_{n, k}^c\right) = 0 \\
		\implies & \mu(E) = 0. \qedhere
	\end{align*}
\end{proof}

\begin{corollary}
	$f_n \to f$ in $L^1 \implies \exists$subsequence $f_{n_j} \to f\ a.e$.
\end{corollary}
\begin{proof}
	$\forall j \in \N, \exists n_j \in \N \st \norm{f_{n_j} - f}_1 < \frac{1}{j^2}$. Then $\sum_{j=1}^\infty \norm{f_{n_j} - f}_1 < \infty$.
\end{proof}
\begin{definition}
	$f_n, f$ measurable functions on $(X, \cA, \mu)$. $f_n \to f$ \emph{in measure} means \[\forall \varepsilon > 0, \lim_{n \to \infty}\mu\left(\left\{x \in X \mid |f_n(x) - f(x)| \geq \varepsilon \right\}\right) = 0.\]
\end{definition}
\begin{example}
	\begin{itemize}
		\item $f_n = n1_{\left(0, \frac{1}{n}\right)}, f = 0$.
		\[
			\forall \varepsilon > 0, \left\lbrace x \mid |f_n(x) - f(x)| > \varepsilon\right\rbrace = \left(0, \frac{1}{n}\right).	
		\]
		(Recall that $f_n \not\to 0$ in $L^1$.)

		\item Typewriter function. (Recall that $f_n \not\to 0$ a.e.)
	\end{itemize}
\end{example}

We can easily check that $f_n \to f$ in $L^1 \implies f_n \to f$ in measure. But the converse is not true.

$f_n \to f$ in measure $\implies \exists f_{n_j} \to f$ a.e. (Check \cite{follandRealAnalysisModern1999})

We have then the following diagram:
\[
	\begin{tikzcd}
	\text{$f_n \to f$ fast $L^1$} 
		\arrow[r, Rightarrow] 
		\arrow[dr, Rightarrow] 
		& \text{$f_n \to f$ in $L^1$} 
			\arrow[r, Rightarrow, shift left=1.5] 
			\arrow[r, "/" marking, Leftarrow, shift right=1.5] 
				& \text{$f_n \to f$ in measure}  \\
		& \text{$f_n \to f$ a.e.}  \arrow[u, "/" marking, Leftarrow, shift left=1.5] \arrow[u, "/" marking, Rightarrow, shift right=1.5] & \text{$\exists f_{n_j} \to f$ a.e.}  \arrow[u, Leftarrow]
	\end{tikzcd}
\]

\begin{definition}
	$f_n, f$ measurable functions on $(X, \cA, \mu)$.
	\begin{enumerate}
		\item $f_n \to f$ \emph{uniformly a.e} means $\exists$ null set $F \st f_n \to f$ uniformly on $F^c$.
		\item $f_n \to f$ \emph{almost uniformly} means $\forall \varepsilon > 0, \exists F \in \cA, \st \mu(F) < \varepsilon, f_n \to f$ uniformly on $F^c$.
	\end{enumerate}
\end{definition}
Recall \ref{le:conv-sets}.

% \epigraph{Still you can massage it and make it work.}{}

\begin{theorem}[Egoroff]
	$f_n, f$ measurable on $(X, \cA, \mu)$. Suppose $\mu(X) < \infty$. Then $f_n \to f$ a.e $\iff f_n \to f$ almost uniformly.
\end{theorem}
\begin{proof}
	"$\impliedby$": DIY
	
	"$\implies$": Fix $\varepsilon > 0$.

	$f_n \to f$ a.e $\implies \displaystyle \mu\left(\bigcup_{k = 1}^\infty \bigcap_{N = 1}^\infty \bigcup_{n = N}^\infty B^c_{n, k}\right) = 0 \implies \forall k, \mu\left(\bigcap_{N = 1}^\infty \bigcup_{n = N}^\infty B^c_{n, k}\right) = 0$.

	By the continuity of measure from above and since $\mu(X) < \infty$, 
	\begin{align*}
		\forall k, \lim_{N \to \infty} \mu\left(\bigcup_{n = N}^\infty B^c_{n, k}\right) = 0 \implies \forall k, \exists N_k \in \N, \mu\left(\bigcup_{n = N_k}^\infty B^c_{n, k}\right) < \frac{\varepsilon}{2^k}.
	\end{align*}
	Let $\displaystyle F = \bigcup_{k = 1}^\infty \bigcup_{n = N_k}^\infty B^c_{n, k} \implies \mu(F) < \varepsilon, f_n \to F$ uniformly on $F^c$.
\end{proof}

\chapter{Product Measures}
(p.22 - 36, section 1.2 and section 2.5, 2.6 of \cite{follandRealAnalysisModern1999}) \\ 
The ultimate goal is to prove Fubini's theorem. This is also related to probability in in the sense that a series of events is in product measure.
\section{Product \ttlmath{\sigma}{sigma}-algebra}
\begin{itemize}
	\item Product space $X = \prod_{\alpha \in I}X_\alpha, x = (x_\alpha)_{\alpha \in I}$.
	\item Coordinate map $\pi_\alpha: X \to X_\alpha$.
\end{itemize}
\begin{definition}
	$(X_\alpha, \cA_\alpha)$ measurable space. $\forall \alpha \in I$, the \emph{product $\sigma$-algebra} on $\displaystyle X = \prod_{\alpha \in I}X_\alpha$ is \[\bigotimes_{\alpha \in I} \cA_\alpha  = \gen{\bigcup_{\alpha \in I}\pi_\alpha^{-1}\left(\cA_\alpha\right)}\]
	where \[
		\pi^{-1}_\alpha\left(A_\alpha\right) = \{\pi^{-1}_\alpha(E) | E \in \cA_\alpha\}.
	\]
\end{definition}
\fancyem{Notation} \[I = \{1, \ldots, d\} \implies X = \prod_{i=1}^d X_i, x = (x_1, \ldots, x_d), \bigotimes_{i=1}^d \cA_i = \cA_1 \otimes \ldots \otimes \cA_d.\]

\begin{lemma}
	If $I$ is countable, then 
	\[
		\bigotimes_{\alpha \in I} \cA_\alpha = \gen{\left\lbrace \prod_{i=1}^\infty E_i \mid E_i \in \cA_i \right\rbrace}
	\]
\end{lemma}

\begin{lemma}
	Suppose $\cA_\alpha = \gen{\cE_\alpha}, \forall \alpha \in I$.
	\begin{enumerate}
		\item $\pi^{-1}_\alpha(\cA_\alpha) = \gen{\pi^{-1}_\alpha(\cE_\alpha)}$.
		\item $\displaystyle \bigotimes_\alpha \cA_\alpha = \gen{\bigcup_\alpha\pi^{-1}_\alpha(\cE_\alpha)}$.
		\item If $I$ is countable, then $\displaystyle \bigotimes_{\alpha \in I} \cA_\alpha = \gen{\left\lbrace \prod_{i=1}^\infty E_i \mid E_i \in \cE_i \right\rbrace}$.
	\end{enumerate}
\end{lemma}
\begin{proof}\fnl
	\begin{enumerate}
		\item  \begin{itemize}
			\item $f: Y \to Z$, $\cB$ a $\sigma$-algebra on $Z \implies f^{-1}(\cB)$ is a $\sigma$-algebra since set union commutes with preimage. Hence $\pi^{-1}_\alpha(\cA_\alpha)$ is a $\sigma$-algebra on $X$. Since $\pi^{-1}_\alpha(\cE_\alpha) \subset \pi^{-1}_\alpha(\cA_\alpha) \implies \gen{\pi^{-1}_\alpha(\cE_\alpha)} \subset \pi^{-1}_\alpha(\cA_\alpha)$.
			\item Let $\cM = \{B \subset X_\alpha \mid \pi^{-1}_\alpha(B) \in \gen{\pi^{-1}_\alpha(\cE_\alpha)}\}$. We show that $\cA_\alpha \subset \cM$. \begin{itemize}
				\item $\cM$ is a $\sigma$-algebra. (easy)
				\item $\cE_\alpha \subset \cM$. (by definition)
			\end{itemize}
			So $\cA_\alpha = \gen{\cE_\alpha} \subset \cM$. Hence, if $E \in \cA_\alpha$, $E \subset \cM \implies \pi^{-1}_\alpha(E) \in \gen{\pi^{-1}_\alpha(\cE_\alpha)}$ i.e. $\cA_\alpha \subset \gen{\pi^{-1}_\alpha(\cE_\alpha)}$.
			\end{itemize}
		\item[(b, c)] DIY. \qedhere
	\end{enumerate}
\end{proof}

\begin{theorem}
	Suppose $X_1, \ldots, X_d$ metric spaces. Let $\displaystyle X = \prod_1^d X_i$ with product metric $\displaystyle \rho(x, y) = \sum_{i=1}^d \rho_i(x, y)$. Then \begin{enumerate}
		\item $\displaystyle \bigotimes_{i=1}^d \cB(X_i) \subset \cB(X)$.
		\item If, in addition, each $X_i$ has a countable dense subset, then $\displaystyle \bigotimes_{i=1}^d \cB(X_i) = \cB(X)$.
	\end{enumerate}
\end{theorem}
\begin{proof}
	DIY.
\end{proof}

As a consequence, we have $\cB(\R^d) = \cB(\R) \otimes \ldots \otimes \cB(\R)$.

Suppose $f = u + iv: X \to \C$. $f$ is measurable $\iff u^{-1}(E) \in \cA, v^{-1}(E) \in \cA, \forall E \in \cB(\R) \iff f^{-1}(F) \in \cA, \forall F \in \cB(\C) = \cB(\R^2) = \cB(\R) \otimes \cB(\R)$. 

p.65.
Let's focus on finite product.

\epigraph{You like Minecraft right? It's all rectangles.}{}

\begin{definition}
	Suppose $X, Y$ sets.
	\begin{enumerate}
		\item For a $E \subset X \times Y$, $E_x = \{y \in Y \mid (x, y) \in E\}$ and $E^y = \{x \in X \mid (x, y) \in E\}$.
		\item For $f: X \times Y \to \C$, define $f_x: Y \to \C, f^y: X \to \C$ by $f_x(y) = f(x, y) = f^y(x)$.
		\item 
	\end{enumerate}
\end{definition}
\begin{example}
	$(1_E)_x = 1_{E_x}$. $(1_E)^y = 1_{E^y}$.
\end{example}

\begin{proposition}
	$(X, \cA), (Y, \cB)$ measurable spaces.
	\begin{enumerate}
		\item $E \in \cA \otimes \cB \implies E_x \in \cB, E^y \in \cA, \forall x \in X, y \in Y$.
		\item $f: X \times Y \to \C$ is $\cA \otimes \cB$-measurable $\implies f_x$ is $\cB$-measurable, $f^y$ is $\cA$-measurable, $\forall x \in X, y \in Y$.
	\end{enumerate}
\end{proposition}
\begin{proof}
	\begin{enumerate}
		\item Let $\mathcal{F} = \{E \subset X \times Y \mid \text{(a) holds}\}$. \begin{itemize}
			\item $\mathcal{F}$ is a $\sigma$-algebra (easy)
			\item $\mathcal{R}_0 \defeq \{A \times B \mid A \in \cA, B \in \cB\} \subset \mathcal{F}$ (easy) $\implies \cA \otimes \cB  = \gen{\mathcal{R}_0} \subset \mathcal{F}$
		\end{itemize}
		\item DIY. \qedhere
	\end{enumerate}
\end{proof}
MIDTERM is up till here.

\section{Product Measures}
\begin{definition}
	Suppose $(X, \cA), (Y, \cB)$. A (measurable) rectangle is $R = A \times B, A \in \cA, b \in \cB$.

	Let $\displaystyle \mathcal{R}_0 \defeq \left\lbrace R = A \times B \mid A \in \cA, B \in \cB \right\rbrace$.\\

	$\displaystyle \mathcal{R} \defeq \left\lbrace \bigcup_1^N R_i \mid N \in \N, R_1, \ldots, R_N \text{ disjoint rectangles} \right\rbrace$.

\end{definition}

\begin{lemma}
	$\mathcal{R}$ is an algebra. $\gen{\mathcal{R}_0} = \gen{\mathcal{R}} = \cA \otimes \cB$.
\end{lemma}

\begin{theorem}
	Suppose $(X, \cA, \mu)$, $(Y, \cB, \nu)$ measure spaces. 
	\begin{enumerate}
		\item $\exists$ measure $\mu \times \nu$ on $\cA \otimes \cB$ satisfying $(\mu \times \nu)(A \otimes B) = \mu(A)\nu(B), \forall A \in \cA, B \in \cB$.
		\item If $\mu, \nu$ are $\sigma$-finite, then $\mu \times \nu$ is unique.
	\end{enumerate}
\end{theorem}
\begin{proof}
	\begin{enumerate}
		\item Define $\pi: \mathcal{R} \to [0, \infty]$ by $\pi(A \times B) = \mu(A)\nu(B)$ and extend linearly.

		\fancyem{Claim} $\pi$ is a pre-measure on $\cR$.
	
		Enough to check $\displaystyle \pi(A \times B) = \sum_1^\infty \pi(A_n \times B_n)$ if $\displaystyle A \times B = \bigcup_{1}^\infty(A_n \times B_n)$ disjoint union.
	
		Since $A_n \times B_n$ are disjoint, \[1_{A \times B}(x, y) = \sum_1^\infty 1_{A_n \times B_n}(x, y),\ 1_A(x)1_B(y) = \sum_1^\infty1_{A_n}(x)1_{B_n}(y).\]
	
		By Tonelli's theorem for series and integrals, we have 
		\begin{multline*}
			\mu(A)1_B(y) = \int_x 1_A(x)1_B(y) \df \mu(x) \\ = \sum_1^\infty \int_x 1_{A_n}(x)1_{B_n}(y) \df \mu(x) = \sum_{1}^\infty \mu(A_n)1_{B_n}(y).
		\end{multline*}
			
		We then integrate with respect to $y$ to complete the claim.
	
		By HK theorem, $\exists \mu \otimes \nu$ on $\gen{\cR} = \cA \otimes \cB$ extending $\pi$ on $\cR$.

		\item $\mu, \nu$ $\sigma$-finite $\implies \pi$ is $\sigma$-finite on $\cR \implies$ HK uniqueness them applies. \qedhere
	\end{enumerate}
\end{proof}

So we have a measure 
\[
	(\mu \times \nu)(E) = \inf \left\lbrace\sum_{1}^\infty \mu(A_u)\nu(B_i) \biggr\rvert E \subset \bigcup_1^\infty A_i \times B_i, A_i \in \cA, B_i \in \cB \right\rbrace.
\]

Then one questions naturally arises: suppose $f: X \times Y \to \C$, \[\int_{X \times Y} f \df(\mu \times v) \stackrel{\text{?}}{=} \int_y \left(\int_x f \df \mu\right) \df \nu.\]

\section{Monotone Class Lemma}
\begin{definition}
	Suppose $X$ is a set, $\mathcal{C} \subset \mathcal{P}(X)$. $\mathcal{C}$ is a \emph{monotone class} on $X$ if 
	\begin{itemize}
		\item closed under \emph{countable increasing unions} \\
		(i.e. $E_n \in \mathcal{C}, E_1 \subset E_2 \subset \ldots \implies \bigcup_1^\infty C_i \in \mathcal{C}$.)
		\item closed under \emph{countable decreasing intersections} \\
		(i.e. $E_n \in \mathcal{C}, E_1 \supset E_2 \supset \ldots \implies \bigcap_1^\infty C_i \in \mathcal{C}$.)
	\end{itemize}
\end{definition}

\begin{example}
	\begin{itemize}
		\item $\sigma$-algebra is a monotone class.
		\item $\displaystyle \bigcap_\alpha \mathcal{C}_\alpha$ is a monotone class $\implies$ if $\mathcal{E} \in \mathcal{P}(X)$, there is unique smallest monotone class containing $\mathcal{E}$.
	\end{itemize}
\end{example}

The importance of this definition shows up in the following theorem:
\begin{theorem}
	Suppose $\mathcal{A}_0$ is an \emph{algebra} on $X$. Then $\gen{\cA_0}$ is the monotone class generated by $\cA_0$.
\end{theorem}

\begin{proof}
	Let $\cA = \gen{\cA_0}$, $\mathcal{C} = $ monotone class generated by $\cA_0$.

	\begin{enumerate}
		\item $\cA$ is a $\sigma$-algebra $\implies \cA$ is a monotone class containing $\cA_0 \implies \cA \supset \mathcal{C}$.
		\item To show that $\mathcal{C} \supset \cA$, we show that $\mathcal{C}$ is a $\sigma$-algebra. \begin{enumerate}
			\item $\emptyset \subset \cA_0 \subset \mathcal{C}$.
			\item Let $\cC' = \{E \subset X \mid E^c \subset \cC\}$. \begin{itemize}
				\item $\cC'$ is a monotone class (easy)
				\item $\cA_0 \subset \cC'$ since $(E \in \cA_0 \implies E^c \in \cA_0 \subset \cC)$.
			\end{itemize}
			These two show that $\cC \subset \cC'$. So $E \in \cC \implies E \in \cC' \implies E^c \in \cC$. So $\cC$ is closed under complements.
			\item For $E \subset X$, let $\mathcal{D}(E) = \{F \in \cC \mid E \cup F \in \cC\}$. \begin{itemize}
				\item $\mathcal{D}(E) \subset \cC$ by definition.
				\item $\mathcal{D}(E)$ is a monotone class (easy). $E \cup \left(\bigcup_1^\infty F_n\right) = \bigcap_1^\infty \left(E \cup F_n\right)$.
				\item  If $E \in \cA_0$, then $\cA_0 \subset \mathcal{D}(E)$. ($F \in \cA_0 \implies E \cup F \in \cA_0 \subset \cC$.)
			\end{itemize}
			These show that $\cC = \mathcal{D}(E)$ if $E \in \cA_0$.
			\item Let $\mathcal{D} = \{E \in \cC \mid \mathcal{D}(E) = \cC\} = \{E \in \cC \mid E \cup F \in \cC, \forall F \in \cC\}$. \begin{itemize}
				\item $\cA_0 \subset \mathcal{D}$ by (3).
				\item $\mathcal{D}$ is a monotone class (easy).
				\item $\mathcal{D} \subset \cC$ by definition.
			\end{itemize}
			So we conclude that $\mathcal{D} = \cC$.
			Now we have $\cC$ is closed under finite unions.
			\item $\cC$ is closed under finite unions and countable increasing unions $\implies \cC$ is closed under countable unions. (check) \qedhere
		\end{enumerate}
	\end{enumerate}
\end{proof}

\fancyem{Recall} $E \in \cA \otimes \cB \implies E_x \in \cB, E^y \in \cA, \forall x \in X, y \in Y$. However, the inverse is not necessarily true.

Now comes the main thing:
\section{Fubini-Tonelli Theorem}
\begin{theorem}[Tonelli for characteristic functions] Suppose $(X, \cA, \mu)$, $(Y, \cB, \nu)$ are \emph{$\sigma$-finite} measure spaces. Suppose $E \in \mathcal{A} \otimes \cB$. Then \begin{enumerate}
	\item $\alpha(x) \defeq \nu(E_x): X \to [0, \infty]$ is a $\cA$-measurable function.
	\item $\beta(y) \defeq \mu(E^y): Y \to [0, \infty]$ is a $\cB$-measurable function.
	\item $\displaystyle (\mu \times \nu)(E) = \int_X \nu(E_x)\df \mu(x) = \int_Y \mu\left(E^y\right) \df \nu(y)$.
\end{enumerate}
\end{theorem}
\begin{proof}
	\begin{enumerate}
		\item Assume $\mu, \nu$ are finite measures. Let \[\cC = \left\{E \in \cA \otimes \cB \mid \text{(a), (b), (c) hold}\right\}.\]
		Enough to prove that $\gen{\cR} = \cA \otimes \cB \subset \cC$. 

		Because of monotone class lemma and that $\cR$ is a $\sigma$-algebra, it is enough to show that $\cR \subset \cC$ and $\cC$ is a monotone class.
		\begin{itemize}
			\item Show that $\cR \subset \cC$. \[\alpha(x) = \nu((A \times B)_x) = \begin{cases}
				\nu(B) & x \in A \\ 0 & x \notin A
			\end{cases} = \nu(B)1_A(x).\]
			\begin{multline*}
				(\mu \times \nu)(A \times B) = \mu(A)\nu(B) \\ \iff \int_X \nu((A \times B)_x)\df \mu(x) = \nu(B)\mu(A)
			\end{multline*}

			\item Show that $\cC$ is a monotone class. \begin{enumerate}
				\item Let $E_n \in \cC, E_1 \subset E_2 \subset \ldots$. Need to show that $E = \bigcup_1^\infty E_n \in \cC$. 
				\begin{align*}
					& E_n \in \cC, E_1 \subset E_2 \subset \ldots \\
					\implies & E_x = \bigcup_1^\infty (E_n)_x, (E_1)_x \subset (E_2)_x \subset \ldots \\
					\implies & \alpha(x) = \nu(E_x) = \lim_{n \to \infty} \nu\left((E_n)_x\right), \forall x \in X, \quad \alpha_n(x) \text{ $\cA$-measurable} 
				\end{align*}
				This satisfies (a), (b).
				For (c), we have
				\begin{multline*}
					(\mu \times \nu)(E) = \lim_{n \to \infty}(\mu \times \nu)(E_n) \\
					= \lim_{n \to \infty} \int_X \nu\left((E_n)_x\right) \df \mu(x) \stackrel{MCT}{=} \int_X \nu(E_x) \df \mu(X). 
				\end{multline*}
				So we have shown countable increasing unions.

				\item Let $F_n \in \cC$, $F_1 \supset F_2 \supset \ldots$. Need to show that $F \bigcup_1^\infty F_n \in \cC$. Using continuity of measure from above instead of below, DCT instead of MCT, we obtained a similar result.
			\end{enumerate}
		\end{itemize}
		\item Now assume that $\mu, \nu$ are $\sigma$-finite. Since $X \times Y = \bigcup_1^\infty (X_n \times Y_n)$, where $X_1 \subset X_2 \ldots, Y_1 \subset Y_2 \subset \ldots$ with $\mu(X_k), \nu(Y_k)$ finite. Apply results from then finite case. (DIY) \qedhere
	\end{enumerate}
\end{proof}
\begin{theorem}[Fubini-Tonelli] Suppose $(X, \cA, \mu)$ and $(Y, \cB, \nu)$ are $\sigma$-finite measure spaces. \begin{enumerate}
	\item (Tonelli) If $f: X \times Y \to [0, \infty]$ is $\cA \otimes \cB$-measurable then \begin{enumerate}
		\item $\displaystyle g(x) \defeq \int_Y f(x, y)\df \nu(y): X \to [0, \infty]$ is a $\cA$-measurable function.
		\item $\displaystyle h(y) \defeq \int_X f(x, y)\df \mu(x): Y \to [0, \infty]$ is a $\cB$-measurable function.
		\item We have the iterated integral formula \begin{align*}
			\int_{X \times Y} f \df (\mu \times \nu) 
			& = \int_X \left[\int_Y f(x, y) \df \nu(y)\right] \df \mu(x) \\
			& = \int_X \left[\int_X f(x, y) \df \mu(x)\right] \df \nu(y).
		\end{align*}
	\end{enumerate}
	\item (Fubini) If $f \in L^1(X \times Y, \mu \times \nu)$, then 
	\begin{enumerate}
		\item $f_x \in L^1(Y, \nu)$ for $\mu$-a.e $x$ and $g(x)$ (which is defined $\mu$-a.e) $\in L^1(X, \mu)$.
		\item $f^y \in L^1(X, \mu)$ for $\nu$-a.e $y$ and $h(y)$ (which is defined $\nu$-a.e) $\in L^1(Y, \nu)$.
		\item The iterated integral formula from (a).(3) hold.
	\end{enumerate}
\end{enumerate}
\end{theorem}

Usually we apply Tonelli to $|f|$ to show $f \in L^1(X \times Y, \mu \times \nu)$ and then apply Fubini to evaluate.

\begin{proof}
	See \cite{follandRealAnalysisModern1999}.
\end{proof}

\section{Lebesgue Measure on \ttlmath{\R^d}{R-d}}
\begin{example}[$(\R^2, \cL \otimes \cL, m \times m)$ is not complete]
	Let $A \in \cL, A \neq \emptyset, m(A) = 0$. Let $B \subset [0, 1], B \notin \cL$ (e.g. Vitali set). Then let $E = A \times B, F = A \times [0, 1]$. We can see that $E \subset F$ and $F \in \cL \otimes \cL, (m \times m)(F) = m(A)m([0, 1]) = 0$. 
	
	So $E$ is a subnull set but not $\cL \otimes \cL$-measurable. (otherwise each section of $E$ is measurable, a contradiction.)
\end{example}
\begin{definition}
	Let $(\R^d, \cL^d, m^d)$ be the \emph{completion} of $(\R^d, \cB(\R^d), m \times \ldots \times m)$, which is same(check!) as the \emph{completion} of $(\R^d, \cL \otimes \ldots \otimes \cL, m \times \ldots \times m)$.
\end{definition}
So how do we compute $m^d$?

A \emph{rectangle} in $\R^d$ is $\displaystyle R = \prod_{i=1}^d E_i$, $E_i \in \cB(\R)$. Then 
\[
	m^d(E) = \inf \left\lbrace \sum_1^\infty m^d{R_k}\ \biggr\rvert\ E \subset \bigcup_1^\infty R_k, R_k \text{ rectangle}\right\rbrace.
\]

\begin{theorem}
	Let $E \in \cL^d$.
	\begin{enumerate}
		\item $m^d(E) = \inf\left\lbrace m^d(O) \mid \text{open } O \supset E\right\rbrace = \sup \left\lbrace m^d(K) \mid \text{compact } K \subset E \right\rbrace$.
		\item $E = \underbrace{A_1}_{F\sigma} \cup \underbrace{N_1}_{\text{null}} = \underbrace{A_2}_{G\sigma} \setminus \underbrace{N_2}_{\text{null}}$.
		\item If $m^d(E) < \infty, \forall \varepsilon > 0, \exists R_1, \ldots, R_m$ rectangles whose sides are \emph{intervals} such that $\displaystyle m^d \left(E \triangle \left(\bigcup_1^m R_i\right)\right) < \varepsilon$.
	\end{enumerate}
\end{theorem}
\begin{proof}
	Similar to $d=1$ case.
\end{proof}

\begin{theorem}
	Integrable "step functions" and $C_c(\R^d)$ are dense in $L^1(\R^d, \cL^d, m^d)$.
\end{theorem}
\begin{theorem}
	Lebesgue measure in $\R^d$ is translation-invariant.
\end{theorem}
\begin{theorem}\label{thm:GLonmeasure}
	"Effect of linear transformations on Lebesgue measure"
\end{theorem}
Skip p. 71-81 of \cite{follandRealAnalysisModern1999} except \ref{thm:GLonmeasure}.

\chapter{Differentiation on Euclidean Space}
Suppose $f: [a, b] \to \R$. There are two versions of fundamental theorem of Calculus:
\begin{itemize}
	\item $\displaystyle \int_a^b f'(x) \df x = f(b) - f(a)$.
	\item $\displaystyle \frac{\df}{\df x}\int_a^x f(t)dt = f(x)$.
\end{itemize}
We focus on the second statement, which implies that
\[
	\lim_{r \to 0^+} \frac{1}{r} \int_x^{x + r} f(t) \df t = \lim_{r \to 0^+} \frac{1}{r} \int_{x - r}^x f(t) \df t
\]
Write $\displaystyle f(x) = \frac{1}{r}\int_x^{x + r}f(x) \df t$, then
\[
	\lim_{r \to 0^+} \frac{1}{r} \int_x^{x + r} (f(t) - f(x)) \df t = \lim_{r \to 0^+} \frac{1}{r} \int_{x - r}^x (f(t) - f(x)) \df t.
\]
This generalizes well in $\R^d$:
\[
f: \R^d \to \R, \quad \lim_{r \to 0^+}	\frac{1}{v(B(x, r))} \int_{B(x, r)} f(t) - f(x) \df t = 0.
\]

\fancyem{Question} to what extent does this hold?

Start from \cite[3.4]{follandRealAnalysisModern1999}.
\section{Hardy-Littlewood Maximal Function}
Suppose an open ball in $\R^d, B = B(a, r)$. Denote $cB = B(a, cr), c > 0$.
\begin{lemma}[Vitali-type covering lemma]
	Let $B_1, \ldots, B_k$ be a finite collection of open balls in $R^d$. Then $\exists$ a sub-collection $B'_1, \ldots, B'_m$ of \emph{disjoint} open balls such that
	\[
		\bigcup_1^m(3B'_j) \supset \bigcup_1^k B_i.	
	\]
\end{lemma}
\begin{proof}
	Greedy algorithm.
\end{proof}

\fancyem{Notation}: $\displaystyle \int_E f \df m = \int_E f(x)\df x$.

\begin{definition}
	$f: \R^d \to \C$ is Lebesgue measurable. $f$ is \emph{locally integrable} if 
	\[
		\int_K |f| \df m < \infty, \forall \text{ compact }	 K \subset \R^d.
	\]
	We write $f \in L^1_{\text{loc}}(\R^d)$.
\end{definition}
\begin{example}
	$f(x) = x^2 \in L^1_{\text{loc}}(\R^d)$. (in fact all continuous functions $\in L^1_{\text{loc}}(\R^d)$).
\end{example}
\begin{definition}
	For $f \in L^1_{\text{loc}}(\R^d)$, define Hardy-Littlewood maximal function for $f$
	\[ Hf(x) = \sup \{A_r(x) \mid r > 0\}, \quad A_r(x) = \frac{1}{m(B(x, r))} \int_{B(x, r)}|f(y)| \df y.\]
\end{definition}
\begin{lemma}
	Let $f \in L^1_{\textup{loc}}(\R^d)$. Then, \begin{enumerate}
		\item $A_r(x)$ is jointly continuous for $(x, r) \in \R^d \times (0, \infty)$.
		\item $Hf(x)$ is Borel measurable.
	\end{enumerate}
\end{lemma}
\begin{proof} \fnl
	\begin{enumerate}
		\item $(x, r) \to (x^*, r^*) \implies A_r(x) \to A_{r^*}(x^*)$.

		Let $(x_n, r_n)$ be any sequence $\to (x^*, r^*)$.
		\[ A_{r_n}(x_n) \leq \int |f(y)| 1_{B(x_n, r_n)}(y).\]

		Apply DCT.

		\item $\displaystyle (Hf)^{-1}((a, \infty)) = \bigcup_{r > 0}A_r^{-1}((a, \infty))$ is open. \qedhere
	\end{enumerate}
\end{proof}

\fancyem{Recall} Markov inequality
\[
	m\left(\left\lbrace x \mid |f(x)| \geq c \right\rbrace\right) \leq \frac{1}{c}\int |f(x)| \df x
\]
\begin{theorem}[Hardy-Littlewood maximal inequality] $\exists C_d > 0 \st \forall f \in L^1_{\textup{loc}}(\R^d), \forall \alpha > 0$,
	\[
		m\left(\left\lbrace x \mid Hf(x) > \alpha \right\rbrace\right) \leq \frac{C_d}{\alpha}\int |f(x)| \df x.
	\]
\end{theorem}
\begin{proof}
	Fix $f \in L^1$ and $\alpha > 0$. Let $E = \{x \mid (Hf)(x) > \alpha\}$. $E$ is a Borel measurable set. Then
	\[\displaystyle x \in E \implies \exists r_x > 0, \st A_{r_x}(x) > \alpha \implies m(B(x, r_x)) < \frac{1}{\alpha} \int_{B(x, r_x)}|f(y)| \df y.\]
	By inner regularity, we have $m(E) = \sup \{m(K) \mid \text{ compact } K \subset E\}$. Let $K \subset E$ be compact. Then 
	\begin{align*}
		& K \subset \bigcup_{x \in K} B(x, r_x) \\
		\implies & K \subset \bigcup{i=1}^N B_i \\
		\implies & K \subset \bigcup_{j=1}^m (3B_j'), B_1', \ldots, B_m' \text{ disjoint} \\
		\implies & m(K) \leq \sum_{j=1}^n m(3B'_j) = 3^d\sum_{j=1}^n m(B'_j) \\
		\implies & m(K) \leq \frac{3^d}{\alpha} \sum_{j=1}^N \int_{B'_j} |f(y)| \df y \\
		\implies & m(K) \leq \frac{3^d}{\alpha} \int_{\R^d} |f(y)| \df y. \qedhere
	\end{align*}
\end{proof}

\section{Lebesgue Differentiation Theorem}
\begin{theorem}
	Let $f \in L^1(\R^d)$. Then \[\displaystyle \lim_{r \to 0} \frac{1}{m(B(x, r))} \int_{B(x, r)} |f(y) - f(x)| \df y = 0 \text{ for a.e } x.\]
\end{theorem}
\begin{proof}
	\begin{enumerate}
		\item The result holds for $f \in C_c(\R^d)$ (check!)
		\item Let $f \in L^1(\R^d)$. Fix $\varepsilon > 0$. $\exists g \in C_c(\R^d) \st \norm{f - g}_1 < \varepsilon$. Then \begin{multline*}
			\int_{B(x, r)} |f(y) - f(x)| \df y \\
			\leq \int_{B(x, r)}|f(y) - g(y)| \df y + \int_{B(x, r)}|g(y) - g(x)| \df y + \int_{B(x, r)}|g(x) - f(x)| \df y.
		\end{multline*}
		Let $Q(x) = \displaystyle\limsup_{r \to 0}\frac{1}{m(B(x,r))}\int_{B(x, r)}|f(y) - f(x)| \df y$. We want to show that \[m\left(\{x \mid Q(x) > 0\}\right) = m \left(\bigcup_{n=1}^\infty\left\lbrace x \mid Q(x) > \frac{1}{n}\right\rbrace\right) = 0.\]

		Enough to show that $m(E_\alpha) = 0, \forall \alpha > 0, E_\alpha = \{x \mid Q(x) > \alpha\}$.

		But $Q(x) \leq (H(f - g))(x) + 0 + |g(x) - f(x)| \implies$ 
		\[\{x \mid Q(x) > \alpha \} \subset \left\lbrace x \mid H(f - g)(x) > \frac{\alpha}{2}\right\rbrace \bigcup \left\lbrace x \mid |g(x) - f(x)| > \frac{\alpha}{2}\right\rbrace.\]
		So we have 
		\[m\left(\left\lbrace x \mid Q(x) > \alpha \right\rbrace\right) \leq \frac{2C_d}{\alpha} \norm{f - g}_1 + \frac{2}{\alpha}\norm{f - g}_1 \leq \frac{2(C_d + 1)}{\alpha}\varepsilon. \qedhere \]
	\end{enumerate}
\end{proof}

\begin{corollary}
	This also holds for $f \in L^1_{\textup{loc}}(\R^d)$.
\end{corollary}
\begin{proof}
	DIY.
\end{proof}
\begin{corollary}
	For $f \in L^1_{\textup{loc}}(\R^d)$, 
	\[
		\lim_{r \to 0} \frac{1}{m(B(x, r))} \int_{B(x, r)} f(y) \df y = 0 \text{ for a.e }x.
	\]
\end{corollary}
\begin{proof}
	DIY.
\end{proof}

\begin{definition}
	Let $f \in L^1_{\textup{loc}}(\R^d)$. The point $x \in \R^d$ is called a \emph{Lebesgue point} of $f$ if 
	\[
		\lim_{r \to 0} \frac{1}{m(B(x, r))} \int_{B(x, r)} |f(y) - f(x)| = 0.	
	\]
\end{definition}
$f \in L^1_{\textup{loc}}(\R^d) \implies$ a.e point is a Lebesgue point of $f$.

\begin{definition}
	$\{E_r\}_{r > 0}$ \emph{shrinks nicely} to $x$ as $r \to 0$ means $E_r \subset B(x, r)$ and $\exists c > 0 \st cm(B(x, n)) \leq m(E_r)$.
\end{definition}

\begin{corollary}[Lebesgue differentiation theorem]
	\[\left.
		\begin{array}{r}
		E_r \text{ shrinks nicely to } 0 \\
		f \in L^1_{\textup{loc}}(\R^d) \\
		x \text{ a Lebesgue point of } f
	\end{array}\right\rbrace \implies \lim_{r \to 0} \frac{1}{m(E_r)}\int_{E_r + x} |f(y) - f(x)| \df y = 0.\]
\end{corollary}
\begin{proof}
	DIY.
\end{proof}

\begin{corollary}
	$f \in L^1_{\textup{loc}}(\R^d) \implies F(x) = \int_0^x f(y) \df y$ is differentiable and $F'(x) = f(x)$ a.e.
\end{corollary}

Rest of \cite[Ch.3]{follandRealAnalysisModern1999} will be covered later.

\chapter{Normed Vector Spaces}
Topological spaces $\supset$ metric spaces $\supset$ normed spaces $\supset$ inner product spaces.

Let's start with metric spaces. \cite[5.1, 6.1, 6.2]{follandRealAnalysisModern1999}

\section{Metric Spaces and Normed Spaces}
\begin{definition}
	Suppose $Y$ is a set. A \emph{metric} of $Y$ is $\rho: Y \times Y \to [0, \infty) \st$
	\begin{enumerate}
		\item $\rho(x, y) = \rho(y, x)$
		\item $\rho(x, y) \leq \rho(x, z) + \rho(z, y)$
		\item $\rho(x, y) = 0 \iff x = y$.
	\end{enumerate}
\end{definition}

\begin{example} \fnl
	\begin{enumerate}
		\item $\Q, \rho(x, y) = |x - y|$. 
		\item $\R, \rho(x, y) = |x - y|$.
		\item $\displaystyle \R_+, \rho(x, y) = \left| \ln \left(\frac{y}{x}\right) \right|$.
		\item $\displaystyle \R^d, \rho_1(x, y) = \sum_{i=1}^d |x_i - y_i|, \rho_p(x, y) = \left(\sum_{i=1}^d |x_i - y_i|^p\right)^{1/p}, \rho_\infty(x, y) = \max_{1 \leq i \leq d} |x_i - y_i|$.
		\item $\displaystyle C([0, 1]), \rho_p(f, g) = \left(\int_0^1 |f - g|^p\right)^{1/p}, \rho_\infty = \max_{x \in [0, 1]} |f(x) - g(x)|$.
	\end{enumerate}
	They are all metric spaces.
\end{example}

\begin{definition}[Recall \ref{def:seminorm}]
	Suppose $V$ is a vector space over field $\R$ or $\C$. A \emph{seminorm}
	on $V$ is $\|\cdot\|: V \to [0, \infty) \st$
	\begin{itemize}
	\item $\|cv\| = |c|\|v\|, \forall v \in V, \forall c$ scalar
	\item $\|v + w\| \leq \|v\| + \|w\|$, triangle inequality
	\end{itemize}
	A \emph{norm} is a seminorm such that $\|v\| \iff v = 0$.
\end{definition}

Norm gives rise to a metric where $\rho(v, w) = \norm{v - w}$.

$v_n \to v \iff \lim_{n \to \infty}\norm{v_n - v} = 0$.

\begin{example}
	\begin{enumerate}
		\item $L^1(X, \cA, \mu)$
		\item $C([0, 1]), \norm{f}_1 = \int_0^1 |f(x)|\df x, \norm{f}_\infty \max_{0 \leq x \leq 1} |f(x)|$.
		\item $\R^d, \norm{x}_2 = \sqrt{\sum_{1}^d |x_i|^2}, \norm{x}_1 = \sum_{1}^d|x_i, \norm{x}_\infty \max_{1 \leq i \leq d} |x_i|$.
	\end{enumerate}
\end{example}

\section{\ttlmath{L^p}{L\textasciicircum p} Spaces}
\begin{definition}
	Suppose $(X, \cA, \mu)$ a measure space. $f$ is measurable function. For $0 < p < \infty$, define $\displaystyle \norm{f}_p = \left(\int_{X} |f|^p \df \mu\right)^{1/p}$. Define $L^p(X, \cA, \mu) = \left\lbrace f\ \biggr\rvert\ \norm{f}_p < \infty\right\rbrace$.
\end{definition}
\begin{example}
	
\end{example}

\begin{definition}
	$\ell^p = \ell^p(N) = \{a = (a_1, a_2, \ldots) \mid \norm{a}_p = \left(\sum_{1}^\infty |a_i|^p\right)^{1 / p} < \infty\}$.
\end{definition}

\begin{lemma}
	$L^p$ is a vector space, $\forall p \in (0, \infty)$.
\end{lemma}
\begin{proof}
	\[
		\left(\int |cf|^p\right)^{1/p} = |c|\norm{f}_p.
	\]
	Given the following inequality
	\[
		(\alpha + \beta)^p \leq (2 \max(|\alpha|, |\beta|))^p = 2^p \max\left(|\alpha|^p, |\beta|^p\right) \leq 2^p(|\alpha|^p + |\beta|^p)
	\]
	we have \[
		\int |f+g|^p \leq 2^p \left(\int \left(|f|^p + |g|^p\right)\right) \implies \norm{f + g}_p \leq 2 \left(\int \left(|f|^p + |g|^p\right)\right)^{1/p}. \qedhere
	\]
\end{proof}

But we want to know that whether \[
	\norm{f + g}_p \leq \norm{f}_p + \norm{g}_p
\] holds.

\begin{theorem}[Hölder's Inequality]
	Let $p < \infty, q = \frac{p}{p - 1}$ so $\frac{1}{p} + \frac{1}{q} = 1$. Then 
	\[
		\norm{fg}_1 \leq \norm{f}_p\norm{g}_q	
	\]
\end{theorem}
\begin{proof}
	\[
		t \leq \frac{t^p}{p} + 1 - \frac{1}{p}, \forall t \geq 0	
	\]
	(Take $F(t) = t - \frac{t^p}{p}$)
	\begin{equation}
		\alpha\beta \leq \frac{\alpha^p}{p} + \frac{\beta^q}{q}, \forall \alpha, \beta \geq 0 \text{ (Young's inequality)}
	\end{equation}
		
	WLOG assume $0 \leq \norm{f}_p, \norm{g}_q < \infty$. Let $F(x) = \frac{f(x)}{\norm{f}_p}, G(x) = \frac{g(x)}{\norm{g}_q}$.

	$\implies \norm{F}_p = 1 = \norm{G}_q$.

	By (5.1), 
	\[
		\int |F(x)G(x)| \leq \int \frac{|F(x)|^p}{p} + \int \frac{|G(x)|^q}{q}	
	\] 
	\[
		\frac{\int |f(x)g(x)|}{\norm{f}_p\norm{g}_q} \leq \frac{1}{p} + \frac{1}{q} = 1.	
	\]
\end{proof}

\begin{theorem}[Minkowski's inequality]
	Let $1 \leq p < \infty$. For $f, g \in L^p, \norm{f + g}_p \leq \norm{f}_p + \norm{g}_p$.
\end{theorem}
\begin{proof}
	$p = 1$ is easy.

	Assume $1 < p < \infty$. WLOG assume $\norm{f + g}_p \neq 0$.
	We have
	\begin{align*}
		\int |f(x) + g(x)|^p & \leq \int |f(x) + g(x)|^{p - 1}(|f(x)| + |g(x)|) \\
		& \leq \left(\int (|f + g|^{p-1})^q\right)^{1 / q}\left(\int |f|^p\right)^{1 / p} + \left(\int (|f + g|^{p-1})^q\right)^{1 / q}\left(\int |g|^p\right)^{1 / p} \\
		& \leq \left(\int (|f + g|^{p-1})^q\right)^{1 / q} \left[\left(\int |f|^p\right)^{1 / p}  + \left(\int |g|^p\right)^{1 / p} \right] \\
		& \leq \left(\int (|f + g|^{p-1})^q\right)^{1 / q} \left[\norm{f}_p  + \norm{g}_p\right]
	\end{align*}
	Since $q(p-1) = p$, divide by $\left(\int (|f + g|^{p-1})^q\right)^{1 / q}$ on both sides we have
	\[
	\left(\int |f(x) + g(x)|^p\right)^{1 - 1/q} \leq \norm{f}_p + \norm{g}_p. \qedhere
	\]	
\end{proof}

Hölder: $\norm{fg}_1 \leq \norm{f}_p\norm{g}_q, \frac{1}{p} + \frac{1}{q} = 1$.

Minkowski: $\norm{f + g}_p \leq \norm{f}_p + \norm{g}_p, 1 \leq p < \infty$.
\begin{definition}
	For a measurable function $f$ on $(X, \cA, \mu)$, let \[S = \{\alpha \geq 0 \mid \mu(\{x \mid |f(x)| > \alpha \}) = 0\} = \{\alpha \geq 0 \mid f(x) \leq \alpha \text{ a.e}\}.\]
	Define $\norm{f}_\infty = \begin{cases}
		\inf S & S \neq \emptyset \\
		\infty & S = \emptyset.
	\end{cases}$. Let $L^\infty(X, \cA, \mu) = \{f \mid \norm{f}_\infty < \infty\}$.
\end{definition}
\begin{example}\fnl
	\begin{itemize}
		\item $(\R, \cL, m)$, $f(x) = \frac{1}{x} 1_{(0, \infty)}(x) \neq L^\infty$, $f(x) = x1_{\Q}(x) + \frac{1}{1 + x^2} \in L^\infty$.
		\item If $f$ is \emph{continuous} on $(\R, \cL, m)$, $\norm{f}_\infty = \sup_{x \in \R} |f(x)|$. For $a \in \ell^\infty$, $\norm{a}_\infty = \sup_{i \in \N} |a_i|$. ($\ell^\infty = \{a = (a_1, a_2, \ldots) \mid \norm{a}_\infty < \infty\} = \{a \mid  \exists M \geq 0 \st |a_i| \leq M_i, \forall i\}$)
	\end{itemize}
\end{example}

\begin{lemma}
	\begin{enumerate}
		\item For $\alpha \geq \norm{f}_\infty, \mu(\{x \mid |f(x)| > \alpha\}) = 0$. For $\alpha < \norm{f}_\infty, \mu(\{x \mid |f(x)| > \alpha\}) > 0$.
		\item $|f(x)| \leq \norm{f}_\infty$ a.e.
		\item $f \in L^\infty \iff \exists$ \emph{bounded} measurable function $g$ such that $f = g$ a.e.
	\end{enumerate}
\end{lemma}
\begin{proof}
	DIY.
\end{proof}

\begin{theorem}\fnl
	\begin{enumerate}
		\item $\norm{fg}_1 \leq \norm{f}_1\norm{g}_\infty$.
		\item $\norm{f + g}_\infty \leq \norm{f}_\infty + \norm{g}_\infty$.
		\item $f_n \to f$ in $L^\infty \iff f_n \to f$ uniformly a.e.
	\end{enumerate}
\end{theorem}
\begin{proof}
	DIY
	For (c): $\implies$ Let $A_n = \{x \mid |f_n(x) - f(x)| > \norm{f_n - f}_\infty\}$. Then $\mu(A_n) = 0$.

	Let $A = \bigcup_1^\infty A_n, \mu(A_n) = 0$. $\forall x \in A^c = \bigcap_1^\infty A_n^c, \forall n, |f_n(x) - f(x)| \leq \norm{f_n - f}_\infty$. The latter converges to $0$ by assumption.

	Given $\varepsilon > 0, \exists N \st \norm{f_n - f}_\infty < \varepsilon, \forall n \geq N$. So $\forall x \in A^c, \forall n \geq N, |f_n(x) - f(x)| \leq \norm{f_n - f}_\infty < \varepsilon$.
\end{proof}

\begin{proposition}\fnl
	\begin{enumerate}
		\item For $1 \leq p < \infty$, the collection of simple functions with finite measure support is dense in $L^p(X, \cA, \mu)$.
		\item For $1 \leq p < \infty$, the collection of step functions (by definition they have finite measure support) is dense in $L^p(\R, \cL, m)$. So is $C_c(\R)$.
		\item For $p = \infty$, the collection of simple functions is dense in $L^\infty(X, \cA, \mu)$.
	\end{enumerate}
\end{proposition}
\begin{proof}
	DIY
\end{proof}
\fancyem{Note}: $C_c(\R)$ is \emph{not dense} in $L^\infty(\R, \cL, m)$.

\section{Embedding Properties of \ttlmath{L^p}{L\textasciicircum p} spaces}
\begin{definition}
	Two norms $\norm{\cdot}, \norm{\cdot}'$ on the same spaces $V$ are said to be \emph{equivalent} if \[\exists c_1, c_2 > 0 \st c_1\norm{v} \leq \norm{v}' \leq c_2 \norm{v}, \forall v \in V.\]
\end{definition}
So on equivalent norms we have same open sets, same convergence.
\begin{example} \fnl
	\begin{itemize}
		\item For $\R^d$, $\norm{\cdot}_p$, $1 \leq p \leq \infty$ are equivalent.
		\item For $1 \leq p, q \leq \infty, p \neq q$, $L^p(\R, m)$-norm and $L^q(\R, m)$-norm are \emph{not equivalent}. $L^p(\R, m) \not\subset L^q(\R, m), L^p(\R, m) \not\supset L^q(\R, m)$.
	\end{itemize}
\end{example}

\begin{proposition}
	Suppose $\mu(X) < \infty$, then for any $0 < p < q \leq \infty, L^q \subseteq L^p$.
\end{proposition}
\begin{proof}
	\begin{itemize}
		\item $p = \infty$ is easy.
		\item Suppose $p < \infty$. \qedhere
	\end{itemize}
\end{proof}

\begin{proposition}
	If $0 < p < q \leq \infty$ then $\ell^p \subseteq \ell^q$.
\end{proposition}


\begin{proposition}
	$\forall 0 < p < q < r \leq \infty$, $L^p \cap L^r \subset L^q$.
\end{proposition}
\begin{proof}
	\begin{itemize}
		\item $p = \infty$ is easy.
		\item Suppose $p < \infty$. Hölder on $p/\lambda q, r/(1-\lambda)q, \lambda = \frac{q^{-1} - r^{-1}}{p^{-1} - r^{-1}}$. \qedhere
	\end{itemize}
\end{proof}

\section{Banach Spaces}

\begin{theorem}
	Suppose $(V, \norm{\cdot})$ a normed space. Then it is complete $\iff$ Every absolutely convergent series is convergent (i.e. if $\sum_{1}^\infty \norm{v_n} < \infty$ then $\exists s \in V \st \sum_{1}^N v_n \to s$ as $N \to \infty$)
\end{theorem}
\begin{proof}
	$\implies$: DIY. (partial sums form a Cauchy Sequence)

	$\impliedby$: Suppose $v_n, n \in \N$ is a Cauchy sequence. $\forall j \in \N, \exists N_j \in \N \st \norm{v_n - v_m} < \frac{1}{2^j}, \forall n, m \geq N_j$.

	WLOG we may assume $N_1 < N_2 < \ldots$. Let $w_1 = v_{N_1}, w_j = v_{N_j} - v_{N_{j-1}}, \forall j \geq 2 \implies \sum_{1}^\infty \norm{w_j} \leq \norm{v_{N_1}} + \sum_{j=2}^\infty \frac{1}{2^{j-1}} < \infty \implies \sum_{1}^k w_j \to \exists s \in V$.
	
	Thus $V_{N_{k}} \to s$ as $k \to \infty$. $v_n$ is Cauchy $\implies v_n \to s$ as $n \to \infty$.
\end{proof}

\section{Bounded Linear Transformation}
\begin{definition}
	Suppose $(V, \norm{\cdot}), (W, \norm{\cdot}')$ two normed spaces. A linear map $T: V \to W$ is said to be a \emph{bounded map} is $\exists c \geq 0 \st \norm{T_v}' \leq C\norm{v}, \forall v \in V$.
\end{definition}
\begin{proposition}
	Suppose $T: (V, \norm{\cdot}) \to (W, \norm{\cdot}')$ is a linear map. Then the followings are equivalent:
	\begin{enumerate}
		\item $T$ is continuous
		\item $T$ is continuous at $0$
		\item $T$ is a bounded map
	\end{enumerate}
\end{proposition}
\begin{proof}
	(a) $\implies$ (b) is clear.

	(b) $\implies$ (c): For $\varepsilon = 1$, $\exists \delta > 0 \st \norm{Tu}' < \varepsilon = 1$ if $\norm{u} < \delta$. Suppose $v \in V, v \neq 0$. Let $u = \frac{\delta}{2\norm{v}}v \implies \norm{u} = \frac{\delta}{2} < \delta \implies \norm{Tu}' < 1 \implies \frac{\delta}{2\norm{v}}\norm{Tv}' < 1 \implies \norm{Tu}' < \frac{2}{\delta}\norm{v}$.

	(c) $\implies$ (a): Fix $v_0 \in V$. $\norm{Tv - Tv_0}' = \norm{T(v - v_0)}' \leq C\norm{v - v_0}$.
\end{proof}

\begin{example}
	\begin{enumerate}
		\item $T: \ell^1 \to \ell^1, Ta = (a_2, a_3, \ldots)$, $\norm{Ta}_1 \leq \norm{a}_1$. $T$ is BLT.
		\item $T: (C([-1, 1]), \norm{\cdot}_1) \to \C, Tf=f(0)$. This is not continuous.
		\item $T: (C([-1, 1]), \norm{\cdot}_\infty) \to \C, Tf=f(0)$ is BLT.
		\item Let $A$ be a $n \times m$ matrix. $T: \R^n \to \R^m, v \mapsto Av$ is BLT.
		\item Let $K(x, y)$ be a continuous function on $[0, 1] \times [0, 1]$. \[T: (C_[0, 1], \norm{\cdot}_\infty) \to (C[0, 1], \norm{\cdot}_\infty), Tf = \int_0^1 K(x, y)f(y) \df y\] is a BLT.
		\item $\displaystyle T: L^1(\R) \to (C(\R), \norm{\cdot}_\infty), (Tf)(t) = \int_{-\infty}^\infty e^{-itx}f(x) \df x$ (Fourier transform of $f$)
		\item $T: (C^\infty([0, 1]), \norm{\cdot}_\infty) \to (C^\infty([0, 1]), \norm{\cdot}_\infty), (Tf)(x) = f'(x)$ is not bounded.
	\end{enumerate}
\end{example}

\begin{definition}
	Let $L(V, W) = \{T: V \to W \mid T \text{ is BLT}\}$. For $T \in L(V, W)$, the \emph{operator norm} of $T$ is
	\begin{align*}
		\norm{T} & \defeq \inf\{c \geq 0 \mid \norm{Tv}' \leq c\norm{v}, \forall v \in V\} \\
		& = \sup\left\{\frac{\norm{Tv}'}{\norm{v}}\ \biggr\rvert\ v \neq 0, v \in V\right\} \\
		& = \sup\left\{\norm{Tv}' \mid \norm{v} = 1\right\}.
	\end{align*}
\end{definition}
\begin{lemma}
	\begin{enumerate}
		\item Above three definitions are equivalent.
		\item It is indeed a normed space.
	\end{enumerate}
\end{lemma}
\begin{proof}
	DIY.
\end{proof}

\section{Dual of \ttlmath{L^p}{L\textasciicircum p} Spaces}

\chapter{Signed and Complex Measures}
\cite[Ch.3]{follandRealAnalysisModern1999}.

\fancyem{Recall} Suppose $(X, \cA, \mu)$ a measure space. $f: X \to [0, \infty]$ measurable. Let $\displaystyle \nu(E) = \int_E f \df \mu, E \in \cA \implies \nu$ is a measure on $(X, \cA)$. 

\section{Signed Measures}
\begin{definition}
	Suppose $(X, \cA)$ a measurable space. A signed measure is $\nu: \cA \to [-\infty, \infty)$ or $\nu: \cA \to (-\infty, \infty]$ such that \begin{itemize}
		\item $\nu(\emptyset) = 0$.
		\item $A_1, A_2, \ldots \in \cA$, $A_i$ disjoint $\implies \displaystyle \nu\left(\bigcup_1^\infty A_i\right) = \sum_1^\infty \nu(A_i)$ where the series converges absolutely if $\displaystyle \nu\left(\bigcup_1^\infty A_i\right) \in (-\infty, \infty)$.
	\end{itemize}
\end{definition}
\begin{example}\fnl
	\begin{itemize}
		\item $\nu$ positive measure $\implies \nu$ is a signed measure.
		\item $\mu_1, \mu_2$ positive measures such that either $\nu_1(X) < \infty$ or $\nu_2(X) < \infty \implies  \nu = \mu_1 - \mu_2$ a signed measure.
		\item $f: X \to \bar{\R} \st \displaystyle \int_X f^+ \df \mu < \infty$ or $\displaystyle \int_X f^- \df \mu < \infty \implies \nu(E) = \int_E f \df \mu$.
	\end{itemize}
\end{example}

\fancyem{Note}: 
\begin{enumerate}
	\item $A \subset B \nRightarrow \nu(A) \leq \nu(B)$ since $\nu(B) = \nu(A) + \nu(B \setminus A)$.
	\item $A \subset B, \nu(A) = \infty \implies \nu(B) = \infty$.
\end{enumerate}

\begin{lemma}
	$\nu$ is a signed measure on $(X, \cA)$. Then
	\begin{itemize}
		\item $E_n \in \cA, E_1 \subset E_2 \subset \ldots \implies \displaystyle \nu\left(\bigcup_1^\infty E_n\right) = \lim_{n \to \infty}\nu(E_n)$.
		\item $E_n \in \cA, E_1 \supset E_2 \supset \ldots, -\infty < \nu(E_1) < \infty \implies \displaystyle \nu\left(\bigcap_1^\infty E_n\right) = \lim_{N \to \infty} \nu(E_n)$.
	\end{itemize}
\end{lemma}

\begin{definition}
	$\nu$ is a signed measure on $(X, \cA)$. Let $E \in \cA$. We say \begin{enumerate}
		\item $E$ is \emph{positive} for $\nu$ (a positive set for $\nu$) if $\forall F \subset E, F \in \cA$, $\nu(F) \geq 0$.
		\item $E$ is \emph{negative} for $\nu$ (a negative set for $\nu$) if $\forall F \subset E, F \in \cA$, $\nu(F) \leq 0$.
		\item $E$ is \emph{null} for $\nu$ (a null set for $\nu$) if $\forall F \subset E, F \in \cA$, $\nu(F) = 0$.
	\end{enumerate}
\end{definition}

\fancyem{Note} $E$ positive set, $F \subset E \implies \nu(F) \leq \nu(E)$. $E$ negative set, $F \subset E \implies \nu(F) \geq \nu(E)$.

\begin{definition}
	Suupose $\mu, \nu$ are signed measure on $(X, \cA)$. $\nu \perp \nu$ (singular to each other) means $\exists E, F \in \cA \st E \cap F = \emptyset, E \cup F = X$, $F$ is null for $\mu$, $E$ is null for $\nu$.
\end{definition}
\begin{example}
	For $(\R, \cB(\R))$, \begin{enumerate}
		\item Lebesgue measure $m$
		\item Cantor measure $\mu_C((a, b])$.
		\item Discrete measure $\mu_D = \delta_1 + 2 \delta_{-1}$.
	\end{enumerate}
	For (a), (c), take $E = \R \setminus \{-1, 1\}$, $F = \{-1, 1\}$. For (a), (b), take the cantor set $K$, $E = \R \setminus K, F = K$.
\end{example}

\begin{lemma}
	$\nu$ is a signed measure on $(X, \cA)$.
	\begin{enumerate}
		\item $E$ is positive (for $\nu$) and $G \subset E$ measurable $\implies G$ is positive (for $\nu$).
		\item $E_1, E_2, \ldots$ positive sets $\implies \displaystyle \bigcup_1^\infty E_n$ is positive.
	\end{enumerate}
\end{lemma}
\begin{proof}
	DIY.
\end{proof}

\begin{lemma}
	$\nu$ is a signed measure on $(X, \cA)$. Suppose $E \in \cA$ and $0 < \nu(E) < \infty \implies \exists$ measurable set $A \subset E \st A$ is a positive set (for $\nu$) and $\nu(A) > 0$.
\end{lemma}
\begin{proof}[Proof in \cite{roydenRealAnalysis2010}]
	If $E$ is a positive set, we are done.

	Otherwise, $E$ contains sets of negative measure. Let $n_1 \in \N$ be the smallest such that $\exists E_1 \subset E$ with $\nu(E_1) < - \frac{1}{n_1}$. If $E \setminus E_1$ is a positive set then we are done. Otherwise, $E \setminus E_1$ contain sets of measure.

	Inductively if $E \setminus \bigcup_1^{k_1} E_i$ is not a positive set. Let $n_k \in \N$ be the \emph{smallest} such that $\exists E_k \subset E \setminus \bigcup_1^{k_1} E_i$ with $\nu(E_k) < - \frac{1}{n_k}$.

	Note: if $n_k \geq 2, \forall B \subset E \setminus \bigcup_1^{k-1}E_i, \nu(B) \geq  - \frac{1}{n_{k-1}}$.

	Let $A = E \setminus \bigcup_1^\infty E_k$. Since $E = A \cup \bigcup_1^\infty E_k, \nu(E) = \nu(A) + \sum_1^\infty \nu(E_k) \implies \nu(A) > 0$.

	Since $\nu(E), \nu(A)$ are finite, then $\sum_1^\infty \frac{1}{n_k}$ need to be convergent $\implies \lim_{k \to \infty} n_k = \infty$.

	Now, if $B \subset A$ then $B \subset E \setminus \bigcup_1^{k - 1}E_i$. If $\nu(B) \geq -\frac{1}{n_{k-1}} \implies \nu(B) \geq 0$. Thus $A$ is positive.
\end{proof}

\begin{theorem}[The Hahn decomposition theorem]
	Suppose $\nu$ is a signed measure of $(X, \cA)$. Then $\exists P, N \in \cA \st P \cap N = \emptyset, P \cup N = X$, $P$ is positive for $\nu$, and $N$ is negative for $\nu$. If $P', N'$ are another such pair, then $P \triangle P' (= N \triangle N')$ is null for $\nu$.
\end{theorem}
\begin{proof}
	\emph{Uniqueness:} $P \setminus P' \subset P \cap n' \implies P \setminus P'$ is positive and negative, thus a null set. Same for $P \setminus P'$.

	\emph{Existence:} WLOG assume $\nu: \cA \to [-\infty, \infty)$. Let $s = \sup\{\nu(E) \mid E \text{ positive for } \nu\}$. $\exists P_1, P_2, \ldots$ positive sets such that $\lim_{n \to \infty} \nu(P_n) = s$. 
	
	Let $P = \displaystyle\bigcup_1^\infty E_n \implies P$ is positive $\implies \begin{cases}
		s \geq \nu(P) \\
		\nu(P) \geq \nu(P_n)
	\end{cases} \implies \nu(P) = s$. Note that $0 \leq s = \nu(P) < \infty$.

	Let $N = X \setminus P$. Is $N$ a negative set?

	Suppose not. Then $\exists E \subset N \st \nu(E) > 0$. Note that $\nu(E) < \infty \implies \exists$ positive set  $A \subset EA$ with $\nu(A) > 0$. The $P, A$ are disjoint, $P \cup A$ is a positive set, and $\nu(P \cup A) = \nu(P) + \nu(A) > s$, a contradiction. 

	So $N$ is a negative set.
\end{proof}

\begin{theorem}[Jordan decomposition theorem]
	$\nu$ signed measure on $(X, \cA)$. $\exists!$ positive measures $\nu^+, \nu^-$ on $(X, \cA) \st \nu(E) = \nu^+(E) - \nu^-(E), \forall E \in \cA$ and $\nu^+ \perp \nu^-$.
\end{theorem}
\begin{proof}
	$\nu^+(E) = \nu(E \cap P)$, $\nu^-(E) = - \nu(E \cap N)$. DIY.
\end{proof}
\begin{example}
	$(X, \cA, \mu), f: X \to \bar{\R}$. Let $\nu(E) = \int_E f \df \mu$. $\nu^+ = \int_E f^+ \df \mu, \nu^- = \int_E f^- \df \mu$.
\end{example}

\begin{definition}
	Suppose $\nu$ a signed measure on $(X, \cA)$. \emph{Total variation measure} of $\nu$ is $|\nu| = \nu^+ + \nu^-$ (a positive measure on $(X, \cA)$).
\end{definition}
\begin{definition}
	$|\nu|(E)= \int_E |f| \df \nu$
\end{definition}

\begin{lemma}
	\begin{enumerate}
		\item $|\nu(E)| \leq |\nu|(E)$,
		\item $E$ is a null set for $\nu \iff E$ is a null set for $|\nu|$,
		\item Suppose $\kappa$ is another signed measure. $\kappa \perp \nu \iff \kappa \perp |\nu| \iff \kappa \perp \nu^+$ and $\kappa \perp \nu^-$.
	\end{enumerate}
\end{lemma}
\begin{proof}
	DIY.
\end{proof}

\begin{definition}
	$\nu$ is finite ($\sigma$-finite) if $|\nu|$ is a finite ($\sigma$-finite) measure. ($\iff \nu^+, \nu^-$ are finite ($\sigma$-finite) measures.)
\end{definition}

\section{Absolutely Measurable Spaces}
\begin{definition}
	$\mu$ a positive measure, $\nu$ a signed measure on $(X, \cA)$. $\nu \ll \mu$ ($\nu$ is absolutely continuous with respect to $\mu$) $\iff (E \in \cA, \mu(E) = 0 \implies \nu(E) = 0) \iff $ all $\mu$-null sets and $\nu$-null sets. (check)
\end{definition}
\begin{example}
	$(X, \cA, \mu), f: X \to \bar{\R}$. $\nu(E) = \int E f \df \mu \implies \nu \ll \mu$.
\end{example}
\fancyem{Notation}: $\df \nu = f \df \mu$ means $\nu$ is the measure defined by $\nu(E) = \int_E f \df \mu$.

\begin{lemma}
	$\mu$ positive measure, $\nu$ signed measure.
	\begin{enumerate}
		\item $\nu \ll \mu \iff |\nu| \ll \mu \iff \nu^+ \ll \mu$ and $\nu^- \ll \mu$.
		\item $\nu \ll \mu$ and $\nu \perp \mu \implies \nu = 0$. 
	\end{enumerate}
\end{lemma}
\begin{proof}
	
\end{proof}

\begin{theorem}[Radon-Nikodym]\label[thm]{R-N}
	Suppose $\mu$ a $\sigma$-finite positive measure, $\nu$ a $\sigma$-finite signed measure on $(X, \cA)$. Suppose $\nu \ll \mu$. Then $\exists f: X \to \bar{\R}$ measurable function such that $\nu(E) = \int_E f \df \mu$. If $g$ is another such function then $f = g$ a.e.
\end{theorem}
\begin{proof}
	Will follow by proof of Lebesgue-Radon-Nikodym on Monday.
\end{proof}

\begin{definition}
	Suppose $\nu \ll \mu$. A \emph{Radon-Nikodym derivative} of $\nu$ with respect to $\mu$ is a function $\frac{\df \nu}{\df \mu}: X \to \bar{\R}$ satisfying $\nu(E) = \int_E \frac{\df \nu}{\df \mu} \df \mu, \forall E \in \cA$.
\end{definition}
\fancyem{Note}: \ref{R-N} shows the existence of such functions. If there is another such function $g$, then $\frac{\df \nu}{\df \mu} = g$ $\mu$-a.e.

\fancyem{Notation}: \[\df \nu = \frac{\df \nu}{\df \mu} \df \mu.\]

\begin{example}
	$F(x) = e^{2x}: \R \to \R$ is continuous and increasing.

	The Lebesgue-Stieltjes measure $\mu_F$ on $(\R, \cB(\R))$ is the unique locally finite Borel measure satisfying $\mu((a, b]) = e^{2b} - e^{2a}, \forall a < b$.
	\[\mu_F(E) \stackrel{\text{why?}}{=}\int_E 2e^{2x} \df x.\]
	So $\mu_F \ll m$ and $\frac{\df \mu_F}{\df m} = 2e^{2x}$.
\end{example}
\begin{example}
	$F(x) = C(x): \R \to \R$ the Cantor function. $C'(x) = 0$ Lebesgue a.e.
	\[\mu_C(E) \neq \int_E 0 \df x.\]
	In particular, $c(b) - c(a) \neq \int_a^b c'(x) \df x$ even if $c$ is continuous and has derivative a.e. So $\mu_c \nll m$. But $\mu_c \perp m$.
\end{example}

\begin{lemma}
	Let $\mu, \nu$ be finite positive measures on $(X, \cA)$. Then either \begin{enumerate}
		\item $\mu \perp \nu$, or
		\item $\exists \varepsilon > 0, \exists F \in \cA \st \nu(F) > 0$ and $F$ is a positive set for $\nu - \varepsilon\mu$. (i.e. $\forall G \subset F, \nu(G) \geq \varepsilon \mu(G)$)
	\end{enumerate}
\end{lemma}
\begin{proof}
	Let $\kappa_n = \nu - \frac{1}{n}\mu$. By Hahn decomposition, write $X = P_n \cup N_n$ where $P_n$ is positive and $N_n$ is negative for $\kappa_n$.
	
	Let $P = \bigcup_1^\infty P_n$ and $N = \bigcap_1^\infty N_n = X \setminus P$.
	We have $\kappa_n(N) \leq 0$ since $N \subset N_n, \forall n \implies 0 \leq \nu(N) \leq \frac{1}{n}\mu(N), \forall n \implies \nu(N) = 0$.

	Now if $\mu(P) = 0$ then $\mu \perp \nu$. Otherwise $\exists n \st \mu(P_n) > 0$. Take $F = P_n, \varepsilon = \frac{1}{n}$ we have that $F$ is a positive set for $\nu - \varepsilon\mu$ and $\nu(F) > 0$.
\end{proof}

\begin{theorem}[Lebesgue-Radon-Nikodym] 
	Suppose $\mu$ a $\sigma$-finite positive measure, $\nu$ a $\sigma$-finite signed measure on $(X, \cA)$. Then $\exists! \lambda, \rho$ $\sigma$-finite signed measures on $(X, \cA)$ such that $\lambda \perp \mu, \rho \ll \mu, \nu = \lambda + \rho$.

	Furthermore, $\exists f: X \to \bar{\R}$ measurable function that $\ndf \rho = f\df\mu$. And if there exists another $g$ then $f = g$ $\mu$-a.e. 
\end{theorem}
\begin{proof}
	\begin{enumerate}
		\item Assume $\mu, \nu$ finite positive measure.
		Let \begin{align*}
			\cF & = \left\{g: X \to [0, \infty]\ \biggr\rvert \int_E g \df \mu \leq \nu(E), \forall E \in \cA \right\} \\
			& = \left\{g: X \to [0, \infty]\ \biggr\rvert\ \ndf\nu - g\ndf\nu \text{ is a positive measure}\right\}.
		\end{align*}
		Note that $\cF \neq \emptyset$ since $g = 0 \in \cF$. Let $s = \sup \left\{\int_X g \df \mu \mid g \in \cF\right\}$.
		\begin{enumerate}
			\item $\exists f \in \cF \st s = \int_X f \df \mu$. \begin{enumerate}
				\item $g, h \in \cF \implies u(x) = \max\{g(x), h(x)\} \in \cF$.
				Since setting $A = \{x \mid g(x) \geq h(x)\}$, we have \[
					\int_E u \df \mu = \int_{E \cap A} g \df \mu + \int_{E \cap A^c} h \df \mu.	
				\]
				\item $\exists g_1, g_2, \ldots \st \lim_{n \to \infty} \int_X g_n \df \mu = S$.
				By i, WLOG we can assume $0 \leq g_1(x) \leq g_2(x) \leq \ldots$ and $\st \lim_{n \to \infty} \int_X g_n \df \mu = S$.

				Let $f(x) = \sup_n g_n(x) = \lim_{n \to \infty} g_n(x)$. By MCT, \[\int_E f \df \mu = \lim_{n \to \infty} \int_E g_n \df \mu \leq \nu(E) = S\] when $E = X$.
			\end{enumerate}

			\item Define $\rho(E) = \int_E f \df \mu \implies \rho \ll \mu$ and $\rho(X) = \int_X f \df \mu \leq \nu(X) < \infty$.
			\item Define $\lambda(E) = \nu(E) - \rho(E) = \nu(E) - \int_E f \df \mu \geq 0$. Then $\lambda$ is a positive measure and $\lambda(X) \leq \nu(X) < \infty$.
			\item $\lambda \perp \mu$. Suppose it is not. Then by lemma, $\exists \varepsilon > 0, F \in \cA \st \mu(F) > 0$ and $F$ is a positive set for $\lambda - \varepsilon \mu$.
			
			Let $g(x) = f(x) + \varepsilon 1_F(x)$. Then $\forall E \in \cA$, \begin{align*}
				\int_E g \df \mu & = \int_E f \df \mu + \varepsilon \mu(E \cup F) = \nu(E) - \lambda(E) + \varepsilon \mu(E \cup F) \\
				& \leq \nu(E) - \lambda(E \cap F) + \varepsilon\mu(E \cap F) \\
				& \leq \nu(E)
			\end{align*}
			since $\lambda(E \cap F) - \varepsilon \mu(E \cap F) \geq 0$. 

			But $s \geq \int_X g \df \mu = \int_X f \df \mu + \varepsilon \mu(F) = s + \varepsilon\mu(F) > s$, a contradiction. \qedhere
		\end{enumerate}
	\end{enumerate}
\end{proof}


\section{Lebesgue Differentiation Theorem for Regular Borel Measures on \ttlmath{\R^d}{R\textasciicircum d}}
\cite[p. 99]{follandRealAnalysisModern1999}
\begin{definition}
	A Borel signed measure $\nu$ on $\R^d$ is called \emph{regular} if \begin{enumerate}
		\item $|\nu|(\kappa) < \infty, \forall$ compact $K$.
		\item $|\nu|(E) = \inf\{ m(O) \mid \text{ open } O \supset E\}, \forall $ Borel set $E$.
	\end{enumerate}
\end{definition}
\begin{example}
	LS measure on $\R$ are regular. Lebesgue measure on $\R^d$ is regular (so, the difference of two of them) Note: from (a), $\nu$ regular $\implies \nu$ is $\sigma$-finite,

	If $\df \nu = f \df m$ regular, then $|\nu|(\kappa) = \int_K |f| \df m < \infty$, so $f \in L^1_\textup{loc}(\R^d)$.
\end{example}

\begin{lemma}
	If $f \in L^1_\textup{loc}(\R^d) \iff \df \nu = f \df m$ is regular
\end{lemma}
\begin{proof}
	Read the book.
\end{proof}

\fancyem{Recall} Lebesgue differentiation theorem

\begin{corollary}
	Let $\rho$ be a \emph{regular} signed Borel measure on $\R^d$. Suppose $\rho \ll m \implies$ For Lebesgue a.e.-$x$, $\lim_{r \to 0}\frac{\rho(E_r)}{m(E_r)} = \frac{\df \rho}{\df m}(x)$ for every $E_r \to x$ nicely. 
\end{corollary}

\begin{proposition}
	Let $\lambda$ be a \emph{regular} positive Borel measure on $\R^d$. Suppose $\lambda \perp m$. For Lebesgue a.e.-$x$, $\lim_{r \to 0}\frac{\lambda(E_1)}{m(E_1)} = 0$ for every $E_r \to x$ nicely.
\end{proposition}
\begin{proof}
	Enough to consider $E_1 = B(x, r)$ 
	\[
		\left\{x \mid \limsup_{r \to 0} \frac{\lambda(E_1)}{m(E_1)} \neq 0\right\} = \bigcup_{n=1}^\infty G_n, G_n = \left\{x \mid \limsup_{r \to 0} \frac{\lambda(E_1)}{m(E_1)} > \frac{1}{n} \right\}
	\]
	Enough to show that $m(G_n) = 0, \forall n$.

	$\lambda \perp m \implies \R^d = A \cup B$ disjoint. $\lambda(A) = 0, m(B) = 0$, Enough to show $m(G_n \cap A) = 0$.

	Fix $\varepsilon > 0$. Since $\lambda$ is regular, $\exists$ open $O \supset A \st \lambda(O) \leq \lambda(A) + \varepsilon = \varepsilon$. $\forall x \in G_n \cap A, \exists r_x > 0 \st \frac{\lambda(B(x, r_x))}{m(B(x, r_x))} > \frac{1}{n}$ and $B(x, r_x) \subset O$.

	Let $K \subset G_n \cap A$, compact. $K \subset \bigcup_{x \in K} B(x, r_x) \implies \exists$ finite subcover $\implies \exists B_1, B_2, \ldots, E_N$ disjoint, $K \subset \bigcup_1^N 3B_i$. 

	$\implies m(K) \leq 3^d \sum_1^N m(B_i) \leq 3^d n \sum_{1}^N \lambda(B_i) = 3^dn\lambda\left(\bigcap_1^N B_i\right) \leq 3^d n \lambda(O) \leq 3^dn\varepsilon \implies m(G_n \cap A) \leq 3^dn\varepsilon$.
\end{proof}

\begin{theorem}[LDT for regular Borel measures] Suppose $\nu$ is a regular Borel signed meaaure on $\R^d$ and $\df \nu = \df \lambda + f \df m$, $\lambda \perp m \implies$ for Leb a.e. $x$, $\lim_{r \to 0} \frac{\nu(E_r)}{m(E_r)} = f(x)$ for every $E_r \to x$ nicely.
\end{theorem}
\begin{proof}
	$\nu$ regular $\implies \lambda, f\df m$ are regular.
\end{proof}

\section{Monotone Differentiation Theorem}
\cite[3.5]{follandRealAnalysisModern1999}
\begin{definition}
	For $F: \R \to \R$ that is increasing, denote $F(x+) = \lim_{y \downarrow x}F(y) = \inf_{y > x} F(y), F(x-) = \lim_{y \uparrow x} F(y) = \sup_{y < x} F(y)$.
\end{definition}

\begin{lemma}
	$F$ is increasing $\implies D = \{x \mid F \text{ is discontinuous at } x\}$ is countable.
\end{lemma}
\begin{proof}
	$x \in D \implies F(x+) > F(x-)$ since $F \nearrow$. For $x, y \in D, x \neq y \implies I_x, I_y$ disjoint. For each $x \in D$, let $I_x = (F(x-), F(x+)) \implies \exists f: D \to \Q$ is $1$-$1$. $I_x$ is open interval, not empty $\implies D$ is countable. 
\end{proof}

\begin{theorem}[Monotone differentiation theorem]
	Suppose $F \nearrow \implies$ \begin{itemize}
		\item $F$ is differentiable Lebesgue a.e.
		\item $G(x) = F(x+)$ is differentiable Lebesgue a.e.
		\item $G' = F'$ a.e.
	\end{itemize}
\end{theorem}
\begin{proof}
	$G$ is increasing, right-continuous on $\R \implies \exists$ Lebesgue-Stieltjes measure $\mu_G$ on $\R$ (so, regular). 
	\[
		\frac{G(x + h) - G(x)}{h} = \begin{cases}
			\dfrac{\mu_G((x, x+h])}{m((x, x+h])} & h > 0, \\[1em]
			\dfrac{\mu_G((x+h, x])}{m((x+h, x])} & h < 0
		\end{cases}	
	\] converges for Lebesgue a.e $x$ by LDT. So $G'$ exists a.e.

	Let $H(x) = G(x) - F(x) \geq 0$. We have \[\{x \mid H(x) > 0\} \subset \{x \mid x \text{ is discontinuous at } x\}.\]
	So $\{x \mid H(x) > 0\}$ it is countable. Denote the set as $\{x_n\}$.
	
	Let $\mu = \sum_{n} H(x_n) \delta_{x_n}$. Then \[\mu((-N, N)) = \sum_{x_n \in (-N, N)} H(x_n) \stackrel{check}{\leq} G(N) - F(-N) < \infty.\]

	So $\mu$ is a locally finite Borel measure on $\R \implies \mu$ is regular. Hence \[
		\left|\frac{H(x + h) - H(x)}{h}\right| \leq \frac{H(x + h) + H(x)}{|h|} \leq 4 \frac{\mu((x-2h, x+2h))}{4|h|} \xrightarrow[]{\text{LDT}, \mu \perp m} 0
	\] for Lebesgue a.e. $x$.

	So $H$ is differentiable a.e and $H' = 0$ a.e.
\end{proof}

\begin{proposition}
	$F \nearrow \implies \displaystyle \int_a^b F'(x) \df x \leq F(b) - F(a)$.
\end{proposition}

\begin{example}\fnl
	\begin{itemize}
		\item $F(x) = \begin{cases}
			0 & x \leq 0 \\
			1 & x > 0
		\end{cases}$. $F'(x) = 0$ a.e and $\displaystyle \int_{-1}^1 F'(x) \df x = 0 < F(1) - F(-1) = 1$.
		\item $F(x)$ Cantor function. $F'(x) = 0$ a.e. and $\displaystyle \int_0^1 F'(x)\df x = 0 \leq F(1) - F(0) = 1$.
	\end{itemize}
\end{example}

\section{Functions of Bounded Variation}
\begin{definition}
	For $F: \R \to \R$, the total variation function of $F$ is $T_F: \R \to [0, \infty]$, \[
		T_F(x) = \sup\left\{\sum_{i=1}^n |F(x_i) - F(x_{i-1})| \mid n \in \N, -\infty < x_0 < x_1 < \ldots < x_n = x\right\}.
	\]
\end{definition}
\begin{lemma}
	For $a < b$, \[
		T_F(b) = T_F(a) + \sup\left\{\sum_{i=1}^n |F(x_i) - F(x_{i-1})| \mid n \in \N, a = x_0 < x_1 < \ldots < x_n = b\right\}
	\]
\end{lemma}
Note that $T_F$ is increasing.

\begin{definition}
	$F \in \BV$ ($F$ is of bounded variation) means $T_F(\infty) = \lim_{x \to \infty}T_F(x) < \infty$. 

	$F \in \BV([a, b])$ means $\sup\left\{\sum_{1}^N |F(x_i) - F(x_{i-1})| \mid a = x_0 < x_1 < \ldots < x_n = b\right\} < \infty$.
\end{definition}
Note that $F \in \BV \implies F$ is bounded.

\begin{example}\fnl
	\begin{enumerate}
		\item $F(x) = \sin x \notin \BV, \in \BV([a, b])$.
		\item $F(x) = \begin{cases}
			\frac{\sin x}{x} & x \neq 0 \\
			1 & x = 0
		\end{cases} \notin \BV([a, b])$ for $a < 0 < b$.
		\item $F, G \in \BV \implies \alpha F + \beta G \in \BV$.
		\item $F \nearrow$ and bounded $\implies F \in \BV$.
		\item $F$ \emph{Lipschitz} on $[a, b] \implies F \in \BV([a, b])$. (\emph{Lipschitz} $\implies \exists M \geq 0 \st |F(x) - F(y)| \leq M|x - y|, \forall x, y$.)
		\item $F$ differentiable, $F'$ bounded on $[a, b] \implies F \in \BV([a, b])$.
		\item $F(x) = \int_{-\infty}^x f(t) \in L^1(\R) \implies F \in \BV$ since \[
			\sum_{1}^N|F(x_i) - F(x_{i-1})| \leq \sum_1^N \int_{x_{i-1}}^{x_i} |f(t)| \df t = \int_{x_0}^x |f(t)| \df t \leq \int_{-\infty}^\infty |f(t)| \df t < \infty.
		\]
	\end{enumerate}
\end{example}

\begin{definition}
	$\NBV = \{G \in \BV \mid G \text{ right-continuous}, G(-\infty) = 0\}$.
\end{definition}
\begin{example}\fnl
	\begin{enumerate}
		\item $F \nearrow$, bounded, right-continuous, $F(-\infty) = 0$.
		\item $F(x) = \int_{-\infty}^x f(t)\df t, f \in L^1(\R)$. (Midterm $\implies F$ is uniformly continuous.)
	\end{enumerate}
\end{example}
\begin{lemma}
	$F \in \operatorname{BV}$ and right-continuous $\implies T_F \in \operatorname{NBV}$.
\end{lemma}
\begin{proof}
	$T_F \nearrow$, bounded $\implies T_F \in \operatorname{BV}, T_F(-\infty) = 0$. Is $T_F$ right-continuous?

	Suppose it is not. $\exists a \in \R \st c \defeq T_F(a+) - T_F(a) > 0$. Fix $\varepsilon > 0$. Since $F(x)$ and $g(x) \defeq T_F(x+)$ are right continuous, $\exists \delta > 0 \st$ \[
		|F(y) - F(a)| < \varepsilon, \quad |g(y) - g(a)| < \varepsilon\quad \forall y \in (a, a + \delta].
	\]
	So $T_F(y) - T_F(a+) \leq T_F(y+) - T_F(a+) < \varepsilon$.

	$\exists a = x_0 < x_1 < x_2 < \ldots < x_n = a + \delta \st$
	\begin{align*}
		\sum_{i=1}^n |F(x_i) - F(x_{i-1})| & \geq T_F(a + \delta) - T_F(a) - \frac{c}{4} \\
		& \geq T_F(a+) - T_F(a) - \frac{c}{4} = \frac{3}{4}c.
	\end{align*} 
	This shows that $\sum_{i=2}^n |F(x_i) - F(x_{i-1})| \geq \frac{3}{4}c - \varepsilon$ since

	Consider $[a, x_1]$. $\exists a = t_0 < t_1 < \ldots < t_k = x_1 \st$ \[
		\sum_{i=1}^k |F(t_i) - F(t_{i-1})| \geq T_F(x_1) - T_F(a) - \frac{c}{4} \geq \frac{3}{4}c.
	\]
	So we can write $[a, a + \delta] = [a, x_1] \cup [x_1, a + \delta]$. So \begin{align*}
		\varepsilon + c & \geq T_F(a + \delta) - T_F(a+) + T_F(a+) - T_F(a) \\
						& = T_F(a + \delta) - T_F(a) \\
						& \geq \sum_{j=1}^k |F(t_j) - F(t_{j-1})| + \sum_{i=2}^n |F(x_i) - F(x_{i-1})| \geq \frac{3}{4}c - \varepsilon + \frac{3}{4}c = \frac{3}{2} - \varepsilon \\
		\implies c & \leq 4\varepsilon.
	\end{align*}
	Since $\varepsilon > 0$ is arbitrary, we conclude that $c = 0$, a contradiction.
\end{proof}

\begin{corollary}
	$F \in \operatorname{NBV} \iff F = F_1 - F_2, F_1, F_2 \in \operatorname{NBV}$ and $\nearrow$.
\end{corollary}
\begin{proof}
	Write $F = \dfrac{T_F + F}{2} - \dfrac{T_F - F}{2}$. $T_F(x_1) - T_F(x_2) \geq $ total variation of $F$ on $(x_1, x_2) \geq |F(x_1) - F(x_2)|$ so both functions are increasing.
\end{proof}

\begin{theorem}\fnl
	\begin{enumerate}
		\item $\mu$ is a finite \emph{signed} Borel measure on $\R \implies F(x) \defeq \mu((-\infty, x]) \in \operatorname{NBV}$.
		\item $F \in \NBV \implies \exists!$ finite \emph{signed} Borel measure $\mu_F$ on $\R$ satisfying $\mu((-\infty, x]) = F(x)$.
	\end{enumerate}
\end{theorem}

\begin{proof}
	\begin{enumerate}
		\item $\mu = \mu^+ - \mu^- \implies F = F^+ - F^-, F^\pm (x) = \mu^\pm ((-\infty, x])$ is increasing, bounded, right-continuous, and $F^\pm (-\infty) = 0$. 
		\item $F \in \NBV \implies F = F_1 - F_2, F_1, F_2 \in \NBV$ and are increasing. So $\exists \mu_{F_1}, \mu_{F_2}$ Lebesgue-Stieltjes measure. $\mu_F \defeq \mu_{F_1} - \mu_{F_2}$. Uniqueness is left for homework.
	\end{enumerate}
\end{proof}

\begin{proposition}
	Let $F \in \NBV$. Then \begin{enumerate}
		\item $F$ is differentiable a.e, $F \in L^1(\R, m)$.
		\item $\ndf \mu_{F} = \ndf \lambda + F' \ndf m, \lambda \perp m$.
		\item $\mu_F \perp m \iff F' = 0$ Lebesgue a.e.
		\item $\mu_F \ll m \iff \displaystyle \int_{-\infty}^x F'(t)\df t = F(x)$.
	\end{enumerate}
\end{proposition}
\begin{proof}
	Check (a), (b), (c).

	(d) $\mu_F \ll m \iff \lambda = 0 \iff \ndf \mu_F = F' \ndf m \iff \mu_F = \int_E F' \df m, \forall E$ Borel $\iff F(x) = \int_{-\infty}^x F'(t)\df t, \forall x \in \R$. (by uniqueness)
\end{proof}

\section{Absolutely Continuous Functions}
\def \AC {\operatorname{AC}}
\begin{definition}
	$F: \R \to \R$ is absolutely continuous ($F \in \AC$) means $\forall \varepsilon > 0, \exists \delta > 0 \st$ if $(a_1, b_1), \ldots, (a_N, b_N)$ are \emph{disjoint} open intervals satisfying $\sum_{n=1}^N(b_n - a_n) < \delta$, then $\displaystyle \sum_{n=1}^N |F(b_n) - F(a_n)| < \varepsilon$.
\end{definition}
\begin{lemma}
	\begin{enumerate}
		\item $F \in \AC \implies F$ is uniformly continuous.
		\item $F$ is Lipschitz $\implies F \in \AC$.
		\item $\displaystyle F(x) = \int_\infty^x f(t)\df t, f \in L^1 \implies F \in \AC$.
	\end{enumerate}
\end{lemma}
\begin{proof}
	\begin{align*}
		\sum_{n=1}^N |F(b_n) - F(a_n)| = \sum_{1}^N \left|\int_{a_n}^{b_n} f(t) \df t\right| \leq \sum_1^N \int_{a_n}^{b_n} |f(t)| \df t = \int_E |f| \df m
	\end{align*}
	where $E = \bigcup_{1}^N (a_n, b_n)$.
	By midterm Q1, If $f \in L^1(X, \mu)$ then $\forall \varepsilon > 0, \exists \delta > 0 \st \mu(E) < \delta \implies \int_E |f| < \varepsilon$.
\end{proof}
The inverse of (a) is not always true. The Cantor function $C(x)$ is uniformly continuous but $C \notin \AC$.

\begin{proposition}
	Suppose $F \in \NBV$. Then $F \in \AC \iff \mu_F \ll m$.
\end{proposition}
\begin{corollary}
	$F \in \NBV \cap \AC \iff \displaystyle F(x) = \int_\infty^x f(t)\df t$ for some $f \in L^1(\R, m)$. If this holds, $f = F'$ Lebesgue a.e.
\end{corollary}
\begin{lemma}
	$F \in \AC([a, b]) \implies F \in \NBV([a, b])$.
\end{lemma}
\begin{proof}
	Check. (read the textbook)
\end{proof}
\begin{theorem}[Fundamental theorem of Calculus]
	For $F: [a, b] \to \R$, TFAE: \begin{enumerate}
		\item $F \in \AC([a, b])$,
		\item $\displaystyle F(x) - F(a) = \int_a^x f(t) \df t$ for some $f \in L^1([a, b], m)$,
		\item $F$ is differentiable a.e on $[a, b]$ and $F(x) - F(a) = \int_a^x F'(t) \df t$.
	\end{enumerate}
\end{theorem}

\begin{proof}[Proof of Prop]
	$\impliedby$: Suppose $\mu_F \ll m$. Then $F(x) = \int_{-\infty}^x F'(t)\df t, F' \in L^1 \implies F \in \AC$.

	$\implies$: Suppose $F \in \AC$.
	
	Note: since $F$ is continuous, $\mu_F((a, b]) = \lim_{n \to \infty} \mu_F\left(\left(a, b-\frac{1}{n}\right]\right) = \lim_{h \to \infty} F\left(b - \frac{1}{n}\right) - F(a) = F(b) - F(a)$.
	
	Let $E$ be a Borel set with $m(E) = 0$. Fix $\varepsilon > 0$. Let $\delta > 0$ be the constant from $F \in \AC$. Since $m$ and $\mu_F$ are \emph{regular}, \begin{align*}
		& \exists \text{ open } U_1 \supset U_2 \supset \ldots \supset E \st \lim_{n \to \infty} m(U_n) = m(E) = 0, \\
		& \exists \text{ open } V_1 \supset V_2 \supset \ldots \supset E \st \lim_{n \to \infty} \mu_F(V_n) = \mu_F(E).
	\end{align*}
	Let $O_n = U_n \cap V_n$. $O_n$ is open and $O_1 \supset O_2 \supset \ldots \supset E$. Then \[
		\lim_{n \to \infty} m(O_n) = m(E) = 0, \quad \lim_{n \to \infty} \mu_F(O_n) = \mu_F(E)\ \text{(think about it)}.
	\]
	WLOG, we may assume $m(O_1) < \delta$. Each $O_n = \bigcup_{k=1}^\infty (a^n_k, b^n_K)$ disjoint, $\sum_{k=1}^N (b_k^n, a_k^n) \leq m(O_n) \leq m(O_1) \leq \delta \implies$ \[
		\mu_F \left(\bigcup_{k=1}^N (a^n_k, b^n_K)\right) = \sum_{k=1}^N \mu_F(a^n_k, b^n_K) = \sum_{k=1}^N F(b_k^n) - F(a_k^n).
	\]
	Take the absolute value we have \[
		\left|\mu_F \left(\bigcup_{k=1}^N (a^n_k, b^n_K)\right)\right| \leq \sum_{k=1}^N 
		|F(b_k^n) - F(a_k^n)| < \varepsilon.
	\] Hence \[
		|\mu_F(O_n)| = \lim_{n \to \infty} \left|\mu_F \left(\bigcup_{k=1}^N (a^n_k, b^n_K)\right)\right| \leq \varepsilon \implies |\mu_F(E)| = \lim_{n \to \infty} |
		\mu_F(O_n)| \leq \varepsilon.
	\] Since $\varepsilon > 0$ is arbitrary we conclude that $\mu_F(E) = 0$.
\end{proof}

\begin{definition}
	Suppose $\mu$ a finite signed Borel measure on $\R$. 
	\begin{itemize}
		\item $\mu$ is a \emph{discrete} measure means $\exists$countable set $\{x_n\}$ and $c_n \neq 0 \st \sum_{1}^\infty c_n < \infty$ and $\mu = \sum_n c_n \delta_{x_n}$.
		\item $\mu$ is a \emph{continuous} measure means $\mu(\{a\}) = 0, \forall a \in \R$.
	\end{itemize}
\end{definition}
\begin{lemma}
	\begin{enumerate}
		\item $\mu = \mu_d + \mu_c$ uniquely, where $\mu_d$ is a discrete measure and $\mu_c$ is a continuous measure.
		\item $\mu$ discrete $\implies \mu \perp m$.
		\item $\mu \ll m \implies \mu$ is continuous.
	\end{enumerate}
\end{lemma}
\begin{corollary}
	Suppose $\mu$ is finite signed Borel measure on $\R$. Then $\mu$ can be uniquely written as \[
		\mu = \mu_d + \mu_{ac} + \mu_{sc}	
	\] where $\mu_{ac} \in \AC$ and $\mu_{sc}$ is singularly continuous (continuous and $\perp m$).
\end{corollary}

\chapter{Hilbert Spaces}
\cite[5.5]{follandRealAnalysisModern1999}
\section{Inner Product Spaces}
\begin{definition}
	Suppose $V$ a (complex) vector space. An \emph{inner product} is $\gen{,}, V \times V \to \C$ such that \begin{enumerate}
		\item $\inner{\alpha x + \beta y}{z} = \alpha\inner{x}{z} + \beta\inner{y}{z}$,
		\item $\inner{x}{y} = \overline{\inner{y}{x}}$,
		\item $\inner{x}{x} \in [0, \infty)$,
		\item $\inner{x}{x} = 0 \iff x = 0$.
	\end{enumerate}
\end{definition}
Note that $\inner{x}{\alpha y + \beta z} = \bar{\alpha}\inner{x}{y} + \bar{\beta} \inner{x}{z}$.

\begin{example}
	\begin{itemize}
		\item $\R^d, \inner{x}{y} = x \cdot y = \sum_{1}^d x_iy_i$
		\item $\C^d, \inner{x}{y} = x \cdot y = \sum_{1}^d x_i\bar y_i$.
		\item $L^2(X, \mu), \inner{f}{g} = \int_X f\bar{g}\df \mu$. (Note: by Hölder, $\left|\int f\bar{g}\right| \leq \norm{f\bar g}_1 \leq \norm{f}_2 \norm{g}_2$)
		\item $\ell^2, \inner{x}{y} = \sum_{1}^\infty x_iy_i$.
	\end{itemize}
\end{example}
\begin{definition}
	$\norm{x} = \sqrt{\inner{x}{x}}$. Does it satisfy triangle inequality?
\end{definition}
$\norm{x + y}^2 = \inner{x}{x} + \inner{x}{y} + \inner{y}{x} + \inner{y}{y} = \norm{x}^2 + 2\re\inner{x}{y} + \norm{y}^2$.

\begin{theorem}[Cauchy-Schwarz Inequality]
	$|\inner{x}{y}| \leq \norm{x}\norm{y}$.
\end{theorem}
\begin{proof}
	Clearly if $\inner{x}{y} = 0$. Assume that $\inner{x}{y} \neq 0$.

	$\forall \alpha \in \C, 0 \leq \norm{\alpha x - y}^2 = |\alpha|^2\norm{x}^2 - 2\re \alpha\inner{x}{y} + \norm{y}^2$. Write $\inner{x}{y} = |\inner{x}{y}|e^{i\theta}$.

	Let $\alpha = e^{-i\theta}t, t \in \R$. Then $0 \leq \norm{x}^2t^2 - 2|\inner{x}{y}|t + \norm{y}^2, \forall t \in \R$. Hence $4|\inner{x}{y}|^2 - 4\norm{x}^2\norm{y}^2 \leq 0$.
\end{proof}
\begin{corollary}
	$\norm{x + y} \leq \norm{x} + \norm{y}$. As a consequence, $\norm{x} = \sqrt{\inner{x}{x}}$ is a norm.
\end{corollary}
\begin{proof}
	$\norm{x + y}^2 = \norm{x}^2 + 2\re\inner{x}{y} + \norm{y}^2 \leq \norm{x}^2 + 2\norm{x}\norm{y} + \norm{y}^2 = (\norm{x} + \norm{y})^2$.
\end{proof}

\begin{theorem}
	[Parallelogram law]
	Let $V$ be a normed space. Then, $\norm{\cdot}$ is induced by an inner product $\iff \norm{x + y}^2 + \norm{x - y}^2 = 2\norm{x}^2 + 2\norm{y}^2, \forall x, y \in V$.
\end{theorem}
\begin{proof}
	$\implies$: Follows from $\norm{x \pm y} = \norm{x}^2 \pm 2\re\inner{x}{y} + \norm{y}^2$.

	$\impliedby$: Let \[\inner{x}{y} = \frac{1}{4}\left(\norm{x + y}^2 - \norm{x - y}^2 +i\norm{x + iy}^2 - i\norm{x - iy}^2\right)\]
	and check that it is a inner product.
\end{proof}
\begin{example}
	$L^p(\R, m), f = 1_{(0, 1)}, g = 1_{(1, 2)}$. For $p \neq 2$, the parallelogram law fails.
\end{example}

\begin{lemma}
	Let $V$ be an inner product space. If $X_n \to X$ strongly (i.e. $\lim_{n \to \infty} \norm{x_n - x} = 0$.) Then $X_n \to X$ weakly (i.e. $\forall y \in V, \lim_{n \to \infty}\inner{x_n - x}{y} = 0$.)
\end{lemma}
\begin{proof}
	$|\inner{x_n - x}{y}| \leq \norm{x_n - x}\norm{y}$.
\end{proof}
\begin{example}
	$\ell^2$, $x_n = (0, \ldots, 0, 1 (\text{$n$-th}), 0, \ldots)$.
	Fix $y = \ell^2$. Then $\inner{x_n}{y} = \overline{y_n} \to 0$ as $n \to \infty$ since $\sum_{1}^\infty |y_n|^2 < \infty$. 
	
	Thus, $x_n \to 0$ weakly. But $\norm{x_n - 0} = \norm{x_n} = 0$ so $x_n \not\to 0$ strongly.
\end{example}

\section{Orthonormal Basis}
\begin{definition}
	$x, y$ are called orthogonal ($x \perp y$) if $\inner{x}{y} = 0$.
\end{definition}
\begin{lemma}[Pythagorean theorem]
	\[x_1, \ldots, x_n \in V, \inner{x_i}{x_j} = 0, \forall i \neq j \implies \norm{x_1 + \cdots + x_n}^2 = \norm{x_1}^2 + \cdots + \norm{x_n}^2.\]
\end{lemma}
\begin{definition}
	$\{e_k\}$ is an \emph{orthonormal set} is $\inner{e_m}{e_n} = \begin{cases}
		0 & m \neq n \\
		1 & m = n
	\end{cases}$.
\end{definition}
\begin{lemma}
	[Best approximation] Let $e_1, \ldots, e_n$ orthonormal vectors. For $x \in V$, let $\alpha_i = \inner{x}{e_i}, i = 1, \ldots, N$. Then \[
		\norm{x - \sum_{i=1}^N \alpha_i e_i} \leq \norm{x - \sum_{i=1}^N \beta_i e_i}, \quad \forall \beta_1, \ldots, \beta_N \in \C.
	\]
\end{lemma}
\begin{proof}
	Let $z = x - \sum_1^N \alpha_i e_i, w = \sum_1^N(\alpha_i - \beta_i)e_i$. $\forall n = 1, \ldots, N, \inner{z}{e_n} = \inner{x}{e_n} - \alpha_n = 0 \implies \inner{z}{w} = 0 \implies \norm{z + w}^2 = \norm{z}^2 + \norm{w}^2 \geq \norm{z}^2$.
\end{proof}

\begin{lemma}
	Suppose $\{e_i\}_1^\infty$ orthonormal set. For $x \in V$, let $\alpha_i = \inner{x}{e_i}$. Then \begin{enumerate}
		\item $\norm{x}^2 = \norm{x - \sum_{1}^N \alpha_i e_i}^2 + \sum_{1}^N |\alpha_i|^2, \forall N \in \N$.
		\item $\sum_1^\infty |\alpha_i|^2 \leq \norm{x}^2$. (Bassel's inequality)
	\end{enumerate}
\end{lemma}
\begin{proof}
	\begin{enumerate}
		\item We have \begin{align*}
			\norm{x - \sum_1^N\alpha_i e_i} & = \norm{x}^2 - 2\re\inner{x}{\sum_1^N \alpha_i e_i} + \norm{\sum_1^N \alpha_i e_i}^2 \\
			& = \norm{x}^2 - 2\sum_1^N \re\overline{\alpha_i}\inner{x}{e_i} + \norm{\sum_1^N \alpha_i e_i}^2 \\
			& = \norm{x}^2 - \sum_1^N |\alpha_i|^2.
		\end{align*}
		\item follows from (a). \qedhere
	\end{enumerate}
\end{proof}

\begin{definition}
	An orthonormal set $\{e_i\}$ is said to be an orthonormal basis of $V$ if $\overline{W} = V$ where $W = \left\{\sum_1^n \beta_i e_i \mid N \in \N, \beta_1, \ldots, \beta_N \in \C \right\} = \{\text{finite linear combinations of $\{e_i\}$}\}$ i.e. $\forall x \in V, \forall \varepsilon > 0, \exists w \in W \st \norm{x - w} < \varepsilon$.
\end{definition}
\begin{example}
	$\C^d, e_i = (0, \ldots, 0, 1, 0, \ldots, 0), i = 1, \cdots, d$ and $\ell^2, e_i = (0, \ldots, 0, 1, 0, \ldots), i = 1, 2, \cdots$. 
\end{example}

\begin{definition}
	A \emph{Hilbert space} is an inner product space that is complete.
\end{definition}
\begin{example}
	$\R^d, \C^d, L^2(X, \cA, \mu), \ell^2$.
	
	$C([0, 1]) \subset L^2([0, 1], m)$ is not closed, so it is not a Hilbert space.
\end{example}
\begin{theorem}
	Let $\mathcal{H}$ be a Hilbert space. Let $\{e_i\}_{i=1}^\infty$ be an orthonormal set. TAFE: \begin{enumerate}
		\item $\{e_i\}_{i=1}^\infty$ is an orthonormal basis.
		\item $x \in \cH$ and $\inner{x}{e_i} = 0, \forall i \implies x = 0$.
		\item $x \in \cH \implies S_N \defeq \sum_1^N \alpha_i e_i \to x$ strongly where $\alpha_i = \inner{x}{e_i}$.
		\item $x \in \cH \implies \norm{x}^2 = \sum_1^\infty |\alpha_i|^2$. (Plancherel identity)
	\end{enumerate}
\end{theorem}
\begin{proof}
	(c) $\implies$ (d): $\norm{x} = \norm{x - s_N}^2 + \sum_1^N \norm{\alpha_i}^2$. Since $S_N \to x$ strongly we have $\norm{x} = \lim_{N \to \infty} \sum_1^N \norm{\alpha_1}^2$.

	(d) $\implies$ (a): $\norm{x} = \norm{x - s_N}^2 + \sum_1^N \norm{\alpha_i}^2$ taking limit of both sides we have $0 = \lim_{N \to \infty}\norm{x - s_N}^2$.

	(a) $\implies$ (b): Fix $x \in \cH$. Fix $\varepsilon > 0$. Then, by (a), $\exists y \in \left\{\sum_1^N \beta_i e_i\right\} \st \norm{x - y} < \varepsilon$. By the best approximation lemma, $\norm{x - s_k} \leq \norm{x - y} < \varepsilon$. If $\inner{x}{e_i} = 0, \forall i$, then $s_k = 0$. Thus, $\norm{x} = \norm{x - S_k} < \varepsilon$. Since $\varepsilon > 0$ arbitrary, $\norm{x} = 0$.

	(b) $\implies$ (c): Bessel $\implies \sum_{1}^\infty |\alpha_i| \leq \norm{x} < \infty$. 
	\[\norm{S_N - S_M}^2 = \norm{\sum_{i = M + 1}^N \alpha_i e_i}^2 \sum_{i = M+1}^N \alpha_i|^2 \to 0 \text{ as } N > M \to \infty.\]
	So $\{S_N\}_{N=1}^\infty$ is a Cauchy sequence in $\cH$. Since $\cH$ is complete, $\exists y \in \cH$ such that\\ $\lim_{N \to \infty}\norm{S_n - y} = 0$ i.e. $S_n \to y$ strongly. Is $y = x$?

	Fix $i \in \N, \inner{y - x}{e_i} = \inner{y - S_n}{e_i} + \inner{S_n - x}{e_i} = \alpha_i - \inner{x}{e_i} = 0$ (if $N > i$). So for $N > i$, $\inner{y - x}{e_i} = \inner{y - S_n}{e_i} \implies \inner{y - x}{e_i}$ as $N \to 0$. (Since $S_n \to y$ strongly $\implies S_n \to y$ weakly)

	By (b) we have $y - x = 0 \iff y = x$.
\end{proof}

\begin{corollary}
	[Parseval] $\inner{x}{y} = \sum_{1}^\infty \alpha_n\overline{\beta_n}$.
\end{corollary}

\begin{definition}
	A metric space is called \emph{separable} if $\exists$ countable dense subset.
\end{definition}
\begin{definition}
	$\Q^d \subset \R^d$. $\ell^p, 1 \leq p < \infty$ not $p = \infty$. $L^p(\R, m), 1 \leq p < \infty$ not $p = \infty$.
\end{definition}
\begin{proposition}
	Every separable Hilbert space has a countable orthonormal basis.
\end{proposition}
\begin{proof}
	Gram-Schmidt process.
\end{proof}
Every vector space has a basis, but need to use Zorn's lemma.


\chapter{Intro to Fourier Analysis}
\section{Fourier Series}
\begin{lemma}
	$e_n(x) = \dfrac{1}{\sqrt{2\pi}}e^{inx} = \dfrac{1}{\sqrt{2\pi}}\left(\cos(nx) + i\sin(nx)\right)_{n \in \Z}$ is an orthonormal set in $\cH = L^2\left([-\pi, \pi]\right)$.
\end{lemma}
\begin{proof}
	Direct check. \[
		\frac{1}{2\pi}\int_{-\pi}^\pi e^{i(m - n)x} \df x = \begin{cases}
			1 & m = n \\
			0 & m \neq n
		\end{cases}. \qedhere
	\]
\end{proof}
Question: is $\{e_n\}$ an orthonormal basis?

In $L\left([-\pi, \pi]\right)$, we have \[
	\norm{f}_1 = \int_{-\pi}^\pi |1||f(x)| \leq \norm{1}_2\norm{f}_2 = \frac{1}{\sqrt{2\pi}}\norm{f}_2 \leq 2\pi\norm{f}_\infty.
\]

\begin{definition}
	For $F \in L^1([-\pi, \pi])$, its Fourier coefficients are \[
		\hat{f}_n = \inner{f}{e_n} = \frac{1}{\sqrt{2\pi}}\int_{-\pi}^\pi f(y)e^{-iny}\df y.	
	\]
\end{definition}

We want to have
\begin{align*}
	\sum_{n = -M}^N \hat{f}_n e_n(x) & = \frac{1}{2\pi}\sum_{n=-M}^N \left[\int_{-\pi}^\pi f(y) e^{-iny} \df y\right] \\
	& = \frac{1}{2\pi}\int_{-\pi}^\pi f(y)\left(\sum_{n=-M}^N e^{-in(x-y)}\right) \df y \xrightarrow[\text{in } L^2 ??]{M, N \to \infty} f(x).
\end{align*}

\begin{definition}
	[Poisson Kernel] For $0 \leq r < 1$, \[P_r(t) = \frac{1}{2\pi}\sum_{n = -\infty}^\infty e^{int}r^{|n|} = \frac{1- r^2}{2\pi(1 - 2r\cos t + r^2)}.\]
\end{definition}
\begin{lemma}\label{le:kerneluni1}
	For $f \in L^1([-\pi, \pi])$ and $0 \leq r < 1$, $\sum_{-\infty}^\infty \hat{f}_n e_n(x)r^{|n|}$ converges absolutely and uniformly for $x \in [-\pi, \pi]$, and is equal to \[
		\int_{-\pi}^\pi P_r(x - y)f(y) \df y.	
	\]
\end{lemma}
\begin{proof}
	\begin{align*}
		\sum_{-\infty}^\infty \left[\int_{-\pi}^\pi \left|f(y)e^{-int}\right| \df y\right]\left|e_n(x)\right|r^{|n|} & = \frac{\norm{f}_1}{2\pi} \sum_{-\infty}^\infty r^{|n|} < \infty.
	\end{align*} Thus, Fubini's theorem applies.
	Now \begin{align*}
		\sum_{-\infty}^\infty \left[\int_{-\pi}^\pi \left|f(y)e^{-int}\right| \df y\right]|e_n(x)|r^{|n|} & = \frac{1}{2\pi} \int_{-\pi}^\pi f(y)\left(\sum_{n=-M}^N e^{-in(x-y)}r^{|n|}\right) \df y \\
		& = \int_{-\pi}^\pi P_r(x - y)f(y) \df y.
	\end{align*}
	Need to check a bit more about uniform convergence.
\end{proof}
\fancyem{Note} $P_r(0) = \dfrac{1-r^2}{2\pi(1 - r)^2} = \dfrac{1 + r}{2\pi(1 - r)} \to \infty$ as $r \nearrow 1$.

\begin{lemma}
	$P_r(t)$ form a "family of good kernels" i.e. \begin{enumerate}
		\item $P_r(t) \geq 0$
		\item $\displaystyle \int_{-\pi}^\pi P_r(t) \df t = 1$
		\item $\displaystyle \forall \delta > 0, \lim_{r \nearrow 1} \int_{[-\pi, \pi] \setminus [-\delta, \delta]} P_r(t) \df t = 0$.
	\end{enumerate}
\end{lemma}
\begin{proof}
	(b) 1st formula; (a), (c) 2nd formula.
	\[
		\int_{[-\pi, \pi] \setminus [-\delta, \delta]} P_r(t) \df t \leq \frac{1 - r^2}{2\pi(1 - 2r\cos\delta + r^2)}2\pi \xrightarrow[]{r \nearrow 1} 0. \qedhere
	\]
\end{proof}

\begin{lemma}\label{le:kerneluni2}
	For $f \in C([-\pi, \pi])$ satisfying $f(-\pi) = f(\pi)$, then \[
		\lim_{r \nearrow 1} \int_{-\pi}^\pi P_r(x - y)f(y) \df y = f(x)
	\] uniformly for $x \in [-\pi, \pi]$.
\end{lemma}
\begin{proof}
	Extend $f$ to $f: \R \to \R$ where $f(x + 2\pi) = f(x)$. So $f$ is uniformly continuous and bounded.
	\begin{align*}
		\int_{-\pi}^\pi P_r(x - y)f(y) \df y - f(x)	& = \int_{-\pi}^\pi P_r(y)f(x - y)\df y - f(x)\int_{-\pi}^\pi P_r(y) \df y \\
		& = \int_{-\delta}^\delta P_r(y) (f(x - y) - f(x)) \df y \\
		& \quad + \int\limits_{[-\pi, \pi] \setminus [-\delta, \delta]} P_r(y)(f(x - y) - f(x)) \df y. \qedhere
	\end{align*}
\end{proof}

% Recall: An orthonormal set $\{e_i\}$ is an orthonormal basis if $\forall x \in V, \forall \varepsilon > 0, \exists w = \sum_{1}^k\beta_i e_i \st \norm{x - w} < \varepsilon$.

\begin{theorem}
	$\displaystyle \left\{e_n(x) = \frac{1}{\sqrt{2\pi}}e^{inx}\right\}$ is an orthonormal basis of $L^2([-\pi, -\pi])$.
\end{theorem}
\begin{proof}
	Let $f \in L^2([-\pi, \pi])$. Fix $\varepsilon > 0$.

	$\exists g \in C([-\pi, \pi])$ with $g(\pi) = g(-\pi) \st \norm{f - g}_2 < \frac{\varepsilon}{3}$ (why?)
	
	Let $g_r(x) = \int_{-\pi}^\pi P_r(x - y)g(y) \df y$. By \ref{le:kerneluni2}, $\exists r \in [0, 1) \st \norm{g_r - g}_\infty < \frac{\varepsilon}{3\sqrt{2\pi}}$. So $\norm{g_r - g} < \frac{\varepsilon}{3}$.

	Let $g_{r, N}(x) = \sum_{-N}^N \hat{g_n}e_n(x)r^{|n|}$. By \ref{le:kerneluni1}, $\exists N \in \N \st \norm{g_{r, N} - g_r}_\infty < \frac{\varepsilon}{3\sqrt{2\pi}}$. Thus $\norm{g_{r, N} - g_r}_2 < \frac{\varepsilon}{3}$.

	Hence, $\norm{f - g_{r, N}}_21 < \varepsilon$.
\end{proof}

\begin{example}
	[Plancherel identity] $\norm{f}^2 = \sum_{-\infty}^\infty |\hat{f_n}|^2$.
	
	\[f(x) = x, \hat{f}_n = \frac{1}{\sqrt{2\pi}}\int_{-\pi}^\pi xe^{-inx} \df x\begin{cases}
		0 & n = 0 \\
		\frac{(-1)^ni\sqrt{2\pi}}{n} & n \neq 0
	\end{cases}\]

	So the identity becomes \[
		\sum_{1}^\infty \frac{1}{n^2} = \frac{\pi^2}{6}.	
	\]
\end{example}
\begin{example}[Isoperimetric inequality]
	Suppose $(x(t), y(t)), t \in [-\pi, \pi]$ is a parametric curve in $\R^2$ that is \begin{enumerate}
		\item closed: $(x(-\pi), y(-\pi)) = (x(\pi), y(\pi))$,
		\item smooth: $x, y$ are $C^1$ functions,
		\item simple.
	\end{enumerate}
	Suppose \[
		L = \int_{-\pi}^\pi\sqrt{x'(t)^2 + y'(t)^2} \df t = 2\pi.	
	\]
	What is the largest area $A$ encloses?

	By Green's theorem ($\oint_C P \df x - Q \df y = \iint_D (Q_x - P_y) \df A$), \[
		A = \frac{1}{2} \oint (x\df y - y \df x) = \frac{1}{2}\oint (x(t)y'(t) - x(t)y'(t)) \df t.
	\] 
	Arc length parametrization so that $x'(t)^2 + y'(t)^2 = 1$ for all $t$. Then the condition $L = 2\pi$ can be written as \[
		L = \int_{-\pi}^\pi \left(x'(t)^2 + y'(t)^2\right) \df t = 2\pi
	\]

	Rewrite using $z(t) = x(t) + iy(t), t \in [-\pi, \pi]$ subject to \[
		\int_{-\pi}^\pi \norm{z'(t)}^2 \df t = 2\pi,
	\] find the max of \[
		A = \frac{1}{4i}\int_{-\pi}^\pi \left(\overline{z(t)}z'(t) - z(t)\overline{z'(t)} \right)\df t	
	\]
	Note that $z \in C^1$ and $z(-\pi) = z(\pi)$.

	Denote $\hat{z}_n = \alpha_n$. Now, $\widehat{(z')}_n = \frac{1}{\sqrt{2\pi}}\int_{-\pi}^\pi z'(t)e^{-int} \df t = in\alpha_n$ (integrate by parts).

	By Plancherel, the $L$ constraint becomes \[
		\sum_{-\infty}^\infty |in\alpha_n|^2 = 2\pi.
	\]
	By Parseval, the $A$ object becomes \[
		A = \frac{1}{4i}\sum_{-\infty}^\infty \overline{\alpha_n}(in\alpha_n) - \alpha_n\overline{(in\alpha_n)} = \frac{1}{2}\sum_{-\infty}^\infty n|\alpha_n|^2.
	\]
	The question now becomes the max of $\displaystyle \frac{1}{2}\sum_{-\infty}^\infty n|\alpha_n|^2$ subject to $\displaystyle \sum_{-\infty}^\infty n^2|\alpha_n|^2 = 2\pi$.

	Show that $2\pi - \displaystyle \sum_{-\infty}^\infty n|\alpha_n|^2$ is nonnegative $\iff \displaystyle \sum_{-\infty}^\infty (n^2 - n)|\alpha_n|^2$ is nonnegative, which is obvious. 
	
	$A = \pi \iff$ the equality holds $\iff \alpha_n = 0$ for $n \neq 0, 1 \iff z(t) = \alpha_0 + \alpha_1e_1(t) \iff z(t) = \alpha_0 + \alpha_1e^{-it} \iff |z(t) - \alpha_0| = |\alpha_1|$, which is a circle.

	This beautiful proof is by Hurwitz.
\end{example}

Books: Fourier Series \& Integrals, Dym \& McKean.


\bibliography{analysis}
\bibliographystyle{alpha}
\end{document}
