\documentclass{report}
\usepackage{amsmath,amssymb,amsthm,textcomp,gensymb}
\usepackage{mathtools}
\renewcommand{\qedsymbol}{$\blacksquare$}

\setlength{\topmargin}{0.5in}
\usepackage[margin=4cm]{geometry}
\usepackage{enumerate}

\usepackage{setspace}
\onehalfspacing
\usepackage{parskip}
\setlength{\parskip}{0.5em}
\usepackage[T1]{fontenc}
\usepackage{palatino}

% useful characters/operators
\newcommand{\R}{\mathbb{R}}
\newcommand{\C}{\mathbb{C}}
\newcommand{\Z}{\mathbb{Z}}
\newcommand{\Q}{\mathbb{Q}}
\newcommand{\N}{\mathbb{N}}
\newcommand{\matP}{\mathbb{P}}
\newcommand{\matS}{\mathbb{S}}
\newcommand{\matH}{\mathbb{H}}
\newcommand{\matT}{\mathbb{T}}
\newcommand{\st}{\ s.t.\ }
\newcommand{\ie}{\ i.e.\ }
\newcommand{\eg}{\ e.g.\ }
\def \diam {\operatorname{diam}}
\def \Hom {\operatorname{Hom}}
\def \id {\operatorname{id}}
\def \tr {\operatorname{tr}}
\def \dist {\operatorname{dist}}
\def \intr {\operatorname{int}}
\def \sgn {\operatorname{sgn}}
\def \im {\operatorname{Im}}
\def \re {\operatorname{Re}}
\def \curl {\operatorname{curl}}
\def \divg {\operatorname{div}}
\def \GL {\operatorname{GL}}
\newcommand{\pdr}[2]{\dfrac{\partial #1}{\partial #2}}
\newcommand{\dr}[2]{\dfrac{\text{d} #1}{\text{d} #2}}
\newcommand{\df}{\text{d}}
\newcommand{\inner}[2]{\left\langle #1, #2\right\rangle}

% arrows and :=, =:
\makeatletter
\providecommand*{\twoheadrightarrowfill@}{%
  \arrowfill@\relbar\relbar\twoheadrightarrow
}
\providecommand*{\twoheadleftarrowfill@}{%
  \arrowfill@\twoheadleftarrow\relbar\relbar
}
\providecommand*{\xtwoheadrightarrow}[2][]{%
  \ext@arrow 0579\twoheadrightarrowfill@{#1}{#2}%
}
\providecommand*{\xtwoheadleftarrow}[2][]{%
  \ext@arrow 5097\twoheadleftarrowfill@{#1}{#2}%
}
\makeatother

\newcommand{\defeq}{\vcentcolon=}
\newcommand{\eqdef}{=\mathrel{\mathop:}}

% integral for measure theory
\newcommand{\lowerint}{\underline{\int_{\R^d}}}
\newcommand{\upperint}{\overline{\int_{\R^d}}}
\newcommand{\lint}[1]{\underline{\int_{\R^d}} #1 (x)dx}
\newcommand{\uint}[1]{\overline{\int_{\R^d}} #1 (x)dx}
\newcommand{\sint}[1]{\simp{\int_{\R^d} #1 (x)dx}}
\newcommand{\lesint}[1]{\int_{\R^d} #1 (x)dx}

% note taking
\newcommand{\fancyem}[1]{\underline{\textsc{#1}}}

% theorem style
\newtheorem{theorem}{Theorem}[chapter]
\newtheorem{corollary}[theorem]{Corollary}
\newtheorem{lemma}[theorem]{Lemma}
\newtheorem{conjecture}[theorem]{Conjecture}
\newtheorem{proposition}[theorem]{Proposition}

\theoremstyle{definition}
\newtheorem{definition}[theorem]{Definition}
\newtheorem{example}[theorem]{Example}
\theoremstyle{remark}
\newtheorem*{remark}{Remark}

% for clearer reference
\usepackage{hyperref}
\newcommand{\corollaryautorefname}{Corollary}
\newcommand{\lemmaautorefname}{Lemma}
\newcommand{\definitionautorefname}{Definition}
\newcommand{\exampleautorefname}{Example}
\newcommand{\conjectureautorefname}{Conjecture}
\renewcommand{\subsectionautorefname}{Section}

% other styling
\usepackage{fancyvrb, fancyhdr}
\usepackage{tikz}
\usepackage{tcolorbox}

\pagestyle{fancy}
\fancyhead[LO,L]{\leftmark}
\fancyhead[RO,R]{Yiwei Fu}
\fancyfoot[CO,C]{\thepage}
\renewcommand{\sectionmark}[1]{\markboth{#1}{#1}}

\begin{document}
\pagenumbering{gobble}
\clearpage
\thispagestyle{empty}
% \renewcommand{\ref}[1]{\autoref{#1}}
\title{Notes for Math 597 -- Real Analysis}
\author{Yiwei Fu}
\date{WN 2022}
\maketitle

\tableofcontents
Office hour is Mon 12:30 - 1:30, Tue 12:30 - 1:30 in person EH 5838, Th 1 - 2 online.

\clearpage
\pagenumbering{arabic}
\chapter{Abstract Measure}

\section{$\sigma$-Algebra}
\begin{definition}
Let $X$ be a set. A collection $\mathcal{M}$ of subsets of $X$ is called a $\sigma$-algebra on $X$ if
\begin{itemize}
\item
	$\emptyset \in \mathcal{M}.$
\item
	$\mathcal{M}$ is closed under \underline{complements}: $E \in \mathcal{M} \implies E^c \in \mathcal{M}.$
\item
	$\mathcal{M}$ is closed under \underline{countable unions}: $E_1, E_2, \ldots \in \mathcal{M} \implies \bigcup_{i=1}^\infty E_i \in \mathcal{M}.$
\end{itemize}
\end{definition}
\fancyem{Simple properties:}
\begin{itemize}
\item
	$X = \emptyset^c \in \mathcal{M}.$
\item
	$\bigcap_{i=1}^\infty E_i = \left(\bigcup_{i=1}^n E_i^c\right)^c \in \mathcal{M}.$ It is closed under countable intersections.
\item
	$\bigcup_{i=1}^N E_i = E_i \cup \ldots \cup E_n \cup \emptyset \cup \ldots.$ It is closed under finite unions (similarly, intersections).
\item
	$E \setminus F = E \cap F^c \in \mathcal{M}, E \triangle F = (E \cap F^c) \cup (F \cap E^c) \in \mathcal{M}.$
\end{itemize}

\begin{example}
\begin{enumerate}[(a)]
\item $\mathcal{A} = \mathcal{P}(X)$ power algebra.
\item $\mathcal{A} = \{\emptyset, X\}$ trivial algebra.
\item Let $B \subset X, B \neq \emptyset, B \neq X. \mathcal{A} = \{\emptyset, B, B^c, X\}.$
\end{enumerate}
\end{example}

\begin{lemma}(An intersection of $\sigma$-algebras is a $\sigma$-algebra)
Let $\mathcal{A}_\alpha, \alpha \in I,$ be a family a $\sigma$-algebras of $X$. Then $\bigcap_{\alpha \in I} A_\alpha$ is a $\sigma$-algebra. ($I$ can be uncountable.)
\end{lemma}
\begin{proof}
DIY
\end{proof}

\begin{definition}
For $\mathcal{E} \subset \mathcal{P}(X)$ (not necessarily a $\sigma$-algebra), let $\langle \mathcal{E}\rangle$ be the intersection of all $\sigma$-algebras on $X$ that contains $\mathcal{E}.$ Call it the $\sigma$-algebra generated by $\mathcal{E}.$
\end{definition}
\begin{itemize}
\item $\langle \mathcal{E}\rangle$ is the \underline{smallest} $\sigma$-algebra containing $\mathcal{E}$ and is \underline{unique}.
\item $\{\emptyset, B, B^c, X\} = \langle \{B\} \rangle = \langle \{B^c\}\rangle = \langle \{\emptyset, B\}\rangle.$
\end{itemize}

The above definition gives us (potentially) lots of examples of $\sigma$-algebra on a set $X$

\begin{lemma}\label{le:1.5}
\begin{enumerate}[(a)]
\item
	Suppose $\mathcal{E} \subset \mathcal{P}(X), \mathcal{A}$ a $\sigma$-algebra on $X$. $\mathcal{E} \in \mathcal{A} \implies \langle\mathcal{E}\rangle \in \mathcal{A}.$
\item
	$E \subset F \subset \mathcal{P}(X) \implies \langle \mathcal{E} \rangle \subset \langle\mathcal{F}\rangle.$
\end{enumerate}
\end{lemma}
\begin{proof}

\end{proof}

\begin{definition}
For a topological space $X$, \underline{the Borel $\sigma$-algebra} $\mathcal{B}(X)$ is the $\sigma$-algebra generated by the collection of open sets.
\end{definition}

\begin{example}($X = \R$)
$\mathcal{B}(\R)$ contains the following collections
\[\mathcal{E}_1 = \{(a, b) \mid a < b\}, \quad \mathcal{E}_2 = \{[a, b] \mid a < b\},\]
\[\mathcal{E}_3 = \{(a, b] \mid a < b\}, \quad \mathcal{E}_4 = \{[a, b) \mid a < b\},\]
\[\mathcal{E}_5 = \{(a, \infty) \mid a \in \R\}, \quad \mathcal{E}_6 = \{[a, \infty) \mid a \in \R \},\]
\[\mathcal{E}_7 = \{(-\infty, a) \mid a \in \R\}, \quad \mathcal{E}_8 = \{(-\infty, a] \mid a < b\}\]
\end{example}

\begin{proposition}
$\mathcal{B}(\R) = \langle\mathcal{E}_i\rangle$ for each $i = 1, \ldots, 8.$ 
\end{proposition}
\begin{proof}
\ref{le:1.5}.
\end{proof}

\begin{definition}
$(X, \mathcal{A})$ is called a measurable space.
\end{definition}


\section{Measures}
\begin{definition}
A measure on $(X, \mathcal{A})$ is a function $\mu: \mathcal{A} \to [0, \infty] \st$
\begin{enumerate}
\item $\mu(\emptyset) = 0$
\item (countable additive) For $A_1, A_2, \ldots \in \mathcal{A}$ disjoint we have
\[
\mu\left(\bigcup_1^\infty A_i \right) = \sum_{i=1}^\infty \mu(A_i).
\]
\end{enumerate}
$(X,  \mathcal{A}, \mu)$ is then called a measure space.
\end{definition}

\begin{example}
\begin{enumerate}
\item For any $(X, \mathcal{A}), \mu(A) = \#A$ counting measure.
\item For any $(X, \mathcal{A}),$ let $x_0 \in X.$ The Dirac measure at $x_0$ is
\[
\mu(A) = \begin{cases}
1 & x_0 \in A, \\
0 & x_0 \notin A.
\end{cases}
\]
\item For $(\N, \mathcal{P}(\N)),$ let $a_1, a_2, \ldots \in [0, \infty).$ $\mu(A) = \sum_{i \in A} a_i$ is a measure.
\end{enumerate}
\end{example}

$(X,  \mathcal{A})$ measurable space

$(X,  \mathcal{A}, \mu)$ measure space

$\mu: \mathcal{A} \to [0, \infty] \st \mu(\emptyset) = 0, $ countable additivity.

\fancyem{Note}: $A, B \in \mathcal{A}, A \subset B, $ then $ \mu(B \setminus A) + \mu(A) = \mu(B) \implies \mu(B \setminus A) = \mu(B) - \mu(A)$ if $\mu(A) < \infty.$

\setcounter{theorem}{12}
\begin{theorem}
$(X,  \mathcal{A}, \mu)$ measure space
\begin{enumerate}
\item(monotonicity)
\[A, B \in \mathcal{A}, A \subset B \implies \mu(A) \leq \mu(B).\]
\item(countable subadditivity)
	\[A_1, A_2, \ldots, \in \mathcal{A}, \implies \mu\left(\bigcup_i^\infty A_i\right) \leq \sum_i^\infty \mu(A_i).\]
\item(continuity from below/(MCT) from sets)
	\[A_1, A_2, \ldots \in \mathcal{A}, A_1 \subset A_2 \subset A_3 \subset \ldots \implies \mu\left(\bigcup_i^\infty A_i\right) = \lim_{n \to \infty} \mu(A_n).\]
\item(continuity from above)
	\[A_1, A_2, \ldots \in \mathcal{A}, A_1 \supset A_2 \supset A_3 \supset \ldots, \mu(A_1) < \infty \implies \mu\left(\bigcap_i^\infty A_i\right) = \lim_{n \to \infty} \mu(A_n).\]
\end{enumerate}
\end{theorem}
\begin{proof}
1, 2, DIY.

For 3, let $B_1 = A_1, B_i = A_i \setminus A_{i-1}, i \geq 2. B_i \in \mathcal{A}$ and are disjoint.
\begin{align*}
& \bigcup_i^\infty A_i = \bigcup_i^\infty B_i \\
\implies\ & \mu\left(\bigcup_i^\infty A_i\right) = \mu\left(\bigcup_i^\infty B_i\right) = \sum_{i}^\infty \mu(B_i) = \lim_{n \to \infty} \sum_{i}^n \mu(B_i) = \lim_{n \to \infty} \mu(A_n).
\end{align*}


For 4, let $E_i = A_1 \setminus A_i.$ Hence $E_i \in \mathcal{A}, E_1 \subset E_2 \subset \ldots$
We have
\[
\bigcup_i^\infty E_i = \bigcup_i^\infty (A_1 \setminus A_i) = A_1 \setminus \left(\bigcap_1^\infty A_i\right) \implies \bigcap_1^\infty A_i = A_1 \setminus \left(\bigcup_1^\infty E_i\right).
\]
Hence
\[
\mu\left(\bigcap_1^\infty A_i \right) = \mu(A_1) - \mu\left(\bigcup_1^\infty E_i\right) = \mu(A_1) - \lim_{n\to \infty}\mu(E_n) = \mu(A_1) - \lim_{n\to \infty} \mu(A_1) - \mu(A_n). \qedhere
\]
\end{proof}

\fancyem{Note:} the condition that $\mu(A_1) < \infty$ cannot be dropped. \\
For example,
in $(\N, \mathcal{P}(N), \text{counting measure})$, let $A_n = \{n, n+1, n+2\}, A_1 \supset A_2 \supset A_3 \supset \ldots$ We have $\bigcap_1^\infty = \emptyset \implies \mu\left(\bigcap_1^\infty A_i \right) = 0.$

\begin{definition}
For $(X,  \mathcal{A}, \mu)$ measure space,
\begin{itemize}
\item
$A \subset X$ is a $\mu$-null set if $A \in \mathcal{A}, \mu(A)= 0.$
\item
$A \subset X$ is a $\mu$-subnull set if $\exists B, \mu$-null set $A \subset B.$
\item
$(X,  \mathcal{A}, \mu)$ is a complete measure space if every $\mu$-subnull set is $\mathcal{A}$-measurable.
\end{itemize}
\end{definition}

\begin{definition}$(X,  \mathcal{A}, \mu)$ measure space.
A statement $P(x), x \in X$ holds $\mu$-almost everywhere (a.e.) if the set $\{x \in X \mid P(x) \text{does not hold}\}$ is $\mu$-null.
\end{definition}

\begin{definition}$(X,  \mathcal{A}, \mu)$ measure space.
\begin{itemize}
\item
	$\mu$ is a \underline{finite measure} is $\mu(X) < \infty.$
\item
	$\mu$ is a \underline{$\sigma$-finite measure} if $X = \bigcup_1^\infty X_n, X_n \in \mathcal{A}, \mu(X_n) < \infty.$
\end{itemize}
\end{definition}
HW: every measure space can be "completed."

\section{Outer measures}
\begin{definition}
An \underline{outer measure} on $X$ is $\mu^*: \mathcal{P}(X) \to [0, \infty] \st$
\begin{itemize}
\item
	$\mu^*(\emptyset) = 0$
\item(monotonicity)
	$\mu^*(A) \leq \mu^*(B)$ if $A \subset B.$
\item(countable subadditivity)
	\[\forall A_1, A_2, \ldots \in X, \mu^*\left(\bigcup_i^\infty A_i\right) \leq \sum_i^\infty \mu^*(A_i).\]
\end{itemize}
\end{definition}
\begin{example}
	For $A \subset \R$,
	\[\mu^*(A) = \inf\left\{\sum_{i=1}^\infty(b_i - a_i) \ \biggr\rvert\ \bigcup_1^\infty (a_i, b_i) \supset A\right\}.\]
	is an outer measure due to the next proposition.
\end{example}
\begin{proposition}(1.19)
Let $\mathcal{E} \in \mathcal{P}(X) \st \emptyset, X \in \mathcal{E}.$
Let $\rho: \mathcal{E} \to [0, \infty] \st \rho(\emptyset) = 0.$
Then
\[
\mu^*(A) = \inf\left\{\sum_{i=1}^\infty \rho(E_i) \ \biggr\rvert\ E_i \in \mathcal{E}, \forall i \in N, \bigcup_1^\infty E_i \supset A\right\}\]
is an outer measure on $X$.
\end{proposition}
\begin{proof}
\begin{enumerate}
	\item
	$\mu^*$ is well-defined (inf is taken over non-empty set.)
	\item
	$\mu^*(\emptyset) = 0$
	\item
	$A \subset B \implies \mu^*(A) \leq \mu^*(B).$
\end{enumerate}
We check the countable subadditivity.

Let $A_1, A_2, \ldots \subset X.$ If one of $\mu^*(A_i) = \infty,$ then the result holds. Suppose $\mu^*(A_n) < \infty, \forall n \in \N.$

"Give your self a room of epsilon":\\
Fix $\varepsilon > 0.$ We will show \[\mu^*\left(\bigcup_1^\infty A_n\right) \leq \sum_1^\infty \mu^*(A_i) + \varepsilon.\]

For each $n \in \N, \exists E_{n,1}, E_{n,2}, \ldots \in \mathcal{E} \st$ \[\bigcup_{k=1}^\infty E_{n, k} \supset A_n \quad \text{and} \quad \mu^*(A_n) + \frac{\varepsilon}{2^n}\geq \sum_{k=1}^\infty \rho(E_{n, k}).\]

Then, \[
\bigcup_1^\infty A_n \subset \bigcup_{n=1}^\infty \bigcup_{k=1}^\infty E_{n, k} = \bigcup_{(n, k) \in \N^2} E_{n, k}.
\]

\fancyem{Recall:} Tonelli's thm for series.
If $a_{ij} \in [0, \infty], \forall i, j \in \N, $ then
\[
\sum_{(i, j) \in \N^2} a_{ij} = \sum_{i=1}^\infty\sum_{j=1^\infty} a_{ij} = \sum_{j=1}^\infty\sum_{i=1}^\infty a_ij.
\]
Hence 
\begin{align*}
\mu^*\left(\bigcup_{n=1}^\infty A_n\right) \leq \sum_{n=1}^\infty \rho(E_{k, n}) = \sum_{n=1}^\infty \sum_{k=1}^\infty \rho(E_{k, n}) \leq \sum_{n=1}^\infty\left(\mu^*(A_n) + \frac{\varepsilon}{2^n}\right) = \sum_{n=1}^\infty \mu^*(A_n) + \varepsilon.
\end{align*}
We have shown countable subadditivity.
\end{proof}

Outer measure is very close to a measure. Here the textbooks diverge.

Tao: introduce Lebesgue measure on $\R$ using topological qualities of subsets of $\R.$\\
Folland: introduce abstract method by Carathéodory and Kolomogorov.

\begin{definition}
Let $\mu^*$ be an outer measure on $X$. We say $A \subset X$ is Carathéodory measurable with respect to $\mu^*$ if $\forall E \subset X, \mu^*(E) = \mu^*(E \setminus A) + \mu^*(E \cap A).$ 
\end{definition}
\begin{lemma}
Let $\mu^*$ be an outer measure on $X$. Suppose $B_1, B_2, \ldots, B_N$ are disjoint $C$-measurable sets. Then,
\[
\forall E \subset X, \mu^*\left(E \cap \left(\bigcup_{1}^N B_i\right)\right) = \sum_{i=1}^n \mu^*(E \cap B_i)
\]
\end{lemma}
\begin{proof}
\begin{align*}
\mu^*\left(E \cap \left(\bigcup_1^N B_i\right)\right) = \mu^*(E \cap B_1) +  \mu^*\left(E \cap \left(\bigcup_2^N B_i\right)\right)
\end{align*}
because $B_1$ is $C$-measurable.
Then, iterate.
\end{proof}

Improved version: 

$B_1, B_2, \ldots$ $C$-measurable and \underline{disjoint} $\implies \mu^*\left(E \cap \bigcup_1^\infty B_n\right) = \sum_1^\infty \mu^*\left(E \cap B_n\right), \forall E \subset X.$
\begin{proof}
\begin{align*}
\sum_1^\infty \mu^*(E \cap B_n) & \geq \mu^*\left(E \cap \bigcup_1^\infty B_n\right) \\
& \geq \mu^*\left(E \cap \bigcup_1^N B_n \right) = \sum_1^N \mu^*(E \cap B_n.)
\end{align*}
Take $N \to \infty$ or note that $N \in \N$ is arbitrary we get the result.
\end{proof}


First big theorem:
\begin{theorem}[Carathéodory extension theorem]
Let $\mu^*$ be an outer measure on $X.$ Let $\mathcal{A}$ be the collection of $C$-measurable sets with respect to $\mu^*.$ Then
\begin{enumerate}
\item
	$\mathcal{A}$ us a $\sigma$-algebra on $X$.
\item
	$\mu = \mu^*|_{\mathcal{A}}$ is a measure on $(X, \mathcal{A}).$
\item
	$(X, \mathcal{A}, \mu)$ is a complete measure space.
\end{enumerate}
\end{theorem}
\begin{proof}
\begin{enumerate}
\item
	\begin{enumerate}
	\item
		$\emptyset \in \mathcal{A}.$
	\item
		$\mathcal{A}$ is closed under complements.
	\item
		To show $\mathcal{A}$ closed under countable unions.
		\begin{itemize}
		\item(finite union)\\
		\fancyem{Claim} $A, B \in \mathcal{A} \implies A \cup B \in \mathcal{A}.$
		
		\begin{figure}[h]
		\centering
		\begin{tikzpicture}
		\draw (0,2) circle (1.5) node[left] {$A$};
		\draw (2,2) circle (1.5) node[right] {$B$};
		\draw (-0.5,1.5) rectangle (2.5,0) node[above left] {$E$};
		\node[text width=0.5] at (0, 1.0) {$1$};
		\node[text width=0.5] at (0.925, 1.2) {$2$};
		\node[text width=0.5] at (2, 1.0) {$3$};
		\node[text width=0.5] at (0.925, 0.3) {$4$};
		\end{tikzpicture}
		\label{fig:AB}
		\caption{Venn diagram of $A, B, E$}
		\end{figure}
		
		Fix arbitrary $E \subset X$. We need to show 
		\[
			\mu^*(E) = \mu^*(E \cap (A \cup B)) + \mu^*(E \setminus (A \cup B)).		
		\]
		\ie
		\[
			\mu^*(1 \cup 2 \cup 3 \cup 4) = \mu^*(1 \cup 2 \cup 3) + \mu^*(4)
		\]
		Since $A$ is $C$-measurable, we have
		\[
			\mu^*(1 \cup 2 \cup 3 \cup 4) = \mu^*(1 \cup 2) + \mu^*(3 \cup 4)
		\]
		\[
			\mu^*(1 \cup 2 \cup 3) = \mu^*(1 \cup 2) + \mu^*(3)
		\]
		Similarly since $B$ is $C$-measurable, we have
		\[
			\mu^*(3 \cup 4) = \mu^*(3) + \mu^*(4)
		\]
		Hence
		\begin{align*}
		\mu^*(1 \cup 2 \cup 3 \cup 4)
		& = \mu^*(1 \cup 2) + \mu^*(3 \cup 4) \\
		& = \mu^*(1 \cup 2 \cup 3) - \mu^*(3) + \mu^*(3) + \mu^*(4) \\
		& = \mu^*(1 \cup 2 \cup 3) + \mu^*(4).
		\end{align*}
		\item(countable disjoint unions)\\
		Let $A_1, A_2, \ldots \in \mathcal{A}$ and \underline{disjoint}.
		
		Fix $E \subset X$ arbitrary.
		Since $\mu^*$ is countably subadditive, 
		\[
		\mu^*(E) \leq \mu^*\left(E \cap \bigcup_1^\infty\right) + \mu^*\left(E \setminus \bigcup_1^\infty A_n\right)
		\]
		Fix $n \in \N.$
		\begin{align*}
		& \implies \bigcup_1^N A_n \in \mathcal{A} \\
		& \implies \mu^*(E) = \mu^*\left(E \cap \bigcup_1^N\right) + \mu^*\left(E \setminus \bigcup_1^N A_n\right) \\
		& \geq \sum_{1}^N \mu^*(E \cap A_n) + \mu^*\left(E \setminus \bigcup_1^\infty A_n\right) \text{ by lemma}.
		\end{align*}
		Take $n \to \infty.$
		\item
		(countable unions)\\
		Let $A_1, A_2, \ldots \in \mathcal{A}.$ Take $E_1 = A_1, E_n = A_n \setminus \left(\bigcup_1^{n-1} A_i\right)$ for $n \geq 2.$ Then $\bigcup A_n = \bigcup E_n$ and $E_n$'s are disjoint.
		\end{itemize}
	\end{enumerate}
	\item
	Firstly we have $\mu(\emptyset) = \mu^*(\emptyset) = 0.$
	
	Countable additvity of $\mu^*$ on $\mathcal{A}$ follows from the improved lemma with $E = X$.	
	
	\item
	HW. \qedhere
\end{enumerate}
\end{proof}

\section{Hahn-Kolmogorov Theorem}
\fancyem{Recall} Prop 1.19 Let $\mathcal{E} \subset \mathcal{P}(X) \st \emptyset, X \in \mathcal{E}.$ Let $\rho: \mathcal{E} \to [0, \infty] \st \rho(\emptyset) = 0$

\[
(\mathcal{E}, \rho) \xrightarrow[1.19]{} (\mathcal{P}(X), \mu^*) \xrightarrow[\text{C-theorem}]{} (A, \mu)
\] 
\fancyem{Question} $\mathcal{E} \subset \mathcal{A}$ and $\mu|_\mathcal{E} = \rho$? No!
\begin{definition}
Let $\mathcal{A}_0$ be an algebra on $X$. We say $\mu_0: \mathcal{A}_0 \to [0, \infty]$ is a \underline{pre-measure} if 
\begin{enumerate}
\item $\mu_0(\emptyset) = 0.$
\item (finite additivity)
\[
\mu_0\left(\bigcup_1^N A_i1\right) = \sum_1^N \mu_0(A_i) \text{ if } A_1, \ldots, A_N \in \mathcal{A}_0 \text{ are disjoint.}
\]
\item (countable additivity within the algebra) If $A \in \mathcal{A}_0$ and 
\[
A = \bigcup_1^\infty A_n, A_n \in \mathcal{A}_0 \text{ and are disjoint, then } \mu_0(A) = \sum_1^\infty \mu_0(A_n)
\]
\end{enumerate}
\end{definition}

\fancyem{Notation:} Folland uses $\mathcal{M}$ for $\sigma$-algebra and $\mathcal{A}$ for algebra. (Jinho) uses $\mathcal{A}$ for $\sigma$-algebra and $\mathcal{A}_0$ for alegbra.

\begin{example}
$\mathcal{A}_0$ finite disjoint unions of $(a, b].$
\[
\mu_0\left( \bigcup_1^\infty (a_i, b_i]\right) = \sum_1^\infty (b_i - a_i) \text{ or $b_i^n - a_i^n, e^{b_i} - e^{a_i},$ etc.}
\]
\end{example}

\begin{lemma}
\begin{itemize}
\item 1 + 3 $\implies$ 2.
\item $\mu_0$ is monotone.
\end{itemize}
\end{lemma}

\begin{theorem}[Hahn-Kolmogorov Theorem]
Let $\mu_0$ be a pre-measure on algebra $\mathcal{A}_0$ on $X$. Let $\mu^*$ be the outer measure induced by $(\mathcal{A}_0, \mu_0)$ in Prop 1.19. Let $\mathcal{A}$ and $\mu$ be the Carathéodory $\sigma$-algebra and measure for $\mu^* \implies (\mathcal{A}, \mu)$ extends $(\mathcal{A}_0, \mu_0) \ie \mathcal{A} \supset \mathcal{A}_0, \mu|_{\mathcal{A}_0} = \mu_0.$ 
\end{theorem}
\begin{proof}
\begin{enumerate}
\item ($\mathcal{A} \supset \mathcal{A}_0$)
Let $A \in \mathcal{A}_0$. 

Question: $A \in \mathcal{A}$? i.e. is $A$ $C$-measurable? i.e. $\mu^*(E) = \mu^*(E \cap A) + \mu^*(E \cap A^c), \forall E \subset X.$

Fix $E \subset X.$
\begin{itemize}
\item (countable) subadditivity of $\mu^* \implies \mu^*(E) \leq \mu^*(E \cap A) + \mu^*(E \cap A^c).$
\item If $\mu^*(E) = \infty$ then $\mu^*(E) = \infty \geq \mu^*(E \cap A) + \mu^*(E \cap A^c).$
\item If $\mu^*(E) < \infty.$

Fix $\varepsilon > 0.$ By the definition of $\mu^*, \exists B_1, B_2, \ldots \in \mathcal{A}_0 \st \bigcup_1^\infty B_n \supset E$ and \[\mu^*(E) + \varepsilon \geq \sum_1^\infty \mu_0(B_n) = \sum_1^\infty \left(\mu_0(B_n \cap A) + \mu_0\left(B_n \cap A^c\right))\right).\]

Note that \[
\bigcup_1^\infty (B_n \cap A) \supset E \cap A, \quad \bigcup_1^\infty \left(B_n \cap A^c\right) \supset E \cap A^c \implies \geq
\]
\end{itemize}

\item
Let $A \in \mathcal{A}_0$. We want to show that $\mu(A) = \mu_0(A).$

By definition, $\mu(A) = \mu^*(A).$
\begin{itemize}
\item
Let $B_i = \begin{cases}
A & i = 1, \\
\emptyset & i = 2
\end{cases} \in \mathcal{A}_0$ and $\bigcup_{1}^\infty B_i \supset A.$

Hence $\mu^*(A) \leq \sum_1^\infty \mu_0(B_i) = \mu_0(A).$

\item
Let $B_i \in \mathcal{A}_0, \bigcup_{1}^\infty B_i \supset A$ an arbitrary collection of sets.

Let $C_1 = A \cap B_1, C_i = A \cap B_i \setminus \left(\bigcup_{j=1}^{i-1} B_j\right).$ Then $A = \bigcup_1^\infty$ is a disjoint countable union. By countable additivitiy we have
\[
\mu_0(A) = \sum_1^\infty \mu_0(C_i) \implies \mu_0(A) \leq \sum_1^\infty \mu_0(B_i).
\]
\end{itemize}
Hence we have $\mu_0(A) = \mu^*(A) = \mu(A)$. We have completed our proof. \qedhere
\end{enumerate}
\end{proof}

\begin{definition}
Such $(\mathcal{A}, \mu)$ is called the \underline{Hahn-Kolmogorov extension} of $(\mathcal{A}_0, \mu_0)$, and is also called the \underline{Carathéodory $\sigma$-algebra} for $(\mathcal{A}_0, \mu_0)$.
\end{definition}

\begin{theorem}[uniqueness of HK extension]\label{th:hkunique}
Let $\mathcal{A}_0$ be an algebra on $X$, $\mu_0$ be a pre-measure on $\mathcal{A}_0$, $(\mathcal{A}, \mu)$ be the Hahn-Kolmogorov extension of $(\mathcal{A}_0, \mu_0)$.
And let $(\mathcal{A}', \mu')$ be another extension of $(\mathcal{A}_0, \mu_0)$.

If $\mu_0$ is \underline{$\sigma$-finite}, then $\mu = \mu' = \mathcal{A} \cap \mathcal{A}'.$
\end{theorem}

\fancyem{Note} $\sigma$-finite means
\[\forall X, X = \bigcup_1^\infty X_n, X_n \in \mathcal{A}_0, \mu_0(X_n) < \infty.\]

\begin{corollary}
Let $\mu_0$ be a pre-measure on algebra $\mathcal{A}_0$ on $X$. Suppose $\mu_0$ is $\sigma$-finite, then $\exists!$ mreasure $\mu$ on $\langle \mathcal{A}_0 \rangle$ that extends $\mathcal{A}_0$. Furthermore, 
\begin{enumerate}
\item the completion of $(X, \langle \mathcal{A}_0 \rangle, \mu)$ is the HK extension of $(\mathcal{A}_0, \mu_0).$
\item 
\[
\mu(A) = \inf \left\{\sum_{i=1}^\infty \mu_0(B_i) \mid B_i \subset A_0, \forall i \in \N, \bigcup_1^\infty B_i \supset A\right\}, \forall A \in \overline{\langle \mathcal{A}_0\rangle}.
\]
\end{enumerate}
\end{corollary}

\begin{proof}[Proof of \ref{th:hkunique}]
Let $A \in \mathcal{A} \cap \mathcal{A}'$. We need to show $\mu(A) = \mu^*(A) = \mu'(A).$
\begin{itemize}
\item
$\mu^*(A) \geq \mu'(A)$ (HW)

\item
$\mu(A) \leq \mu'(A)$:
\begin{enumerate}[(i)]
\item Assume $\mu(A) < \infty.$ Fix $\varepsilon > 0$. Then $\exists B_i \in \mathcal{A}_0, \forall i \in \N, \bigcup_1^\infty B_i \supset A \st$
\[
\mu(A) + \varepsilon = \mu^*(A) + \varepsilon \geq \sum_1^\infty \mu_0(B_i) = \sum_1^\infty \mu(B_i) \geq \mu \left(\bigcup_1^\infty B_i\right) = \mu(B)
\]
Hence $\mu(B \setminus A) = \mu(B) - \mu(A) \leq \varepsilon.$

On the other hand, 
\[
\mu(B) = \lim_{N \to \infty} \mu\left(\bigcup_1^N B_i\right) = \lim_{N \to \infty} \mu'\left(\bigcup_1^N B_i\right) = \mu'(B)
\]
by continuity of measure from below.

\[
\mu(A) \leq \mu(B) = \mu'(B) = \mu'(A) + \mu'(B \setminus A) \leq \mu'(A) = \varepsilon.
\]
\item
Assume $\mu(A) = \infty.$

Since $\mu_0$ is $\sigma$-finite, $X = \bigcup_1^\infty X_n, X_n \in \mathcal{A}_0, \mu_0(X_0) < \infty.$
Replacing $X_n$ by $X_1 \cup \ldots \cup X_n$, we may assume $X_1 \subset X_2 \subset \ldots$.

\[\forall n \in N, \mu(A \cap X_n) < \infty \implies \mu(A \cap X_n) \leq \mu'(A \cap X)n).\]

Hence
\[
\mu(A) = \lim_{N \to \infty} \mu(A \cap X_n) \leq \lim_{N \to \infty} \mu'(A \cap X_n) = \mu'(A). \qedhere
\]
\end{enumerate}
\end{itemize}
\end{proof}

\section{Borel Measures on $\R$}
\begin{definition}
$F: \R \to \R$ is an \underline{increasing} function if $F(x) \leq F(y)$ for $x < y$.
$F: \R \to \R$ is increasing and right-continuous $\implies F$ is distribution function. 
\end{definition}

\begin{example}
\begin{itemize}
\item
\[
F(x) = \begin{cases}
1, & x \geq 0 \\
0, & x < 0.
\end{cases}
\]
\item
$\Q = \{r_1, r_2, \ldots\}, F_n(x) = \begin{cases}
1 & x \geq r_n \\
0 & x < r_n.
\end{cases}$
\[
F(x) = \sum_{n=1}^\infty \frac{F_n(x)}{2^n} \text{ is a distribution function.}
\]
\end{itemize}
\end{example}

\fancyem{Note} If $F$ is increasing, $F(\infty) \defeq \lim_{x \to \infty} F(x), F(-\infty) \defeq \lim_{x \to -\infty} F(x)$ exists in $[-\infty, \infty]$.

In probability theory, cumulative distribution function (CDF) is a distribution function with $F(\infty) = 0$ and $F(-\infty) = 0$.

There are distributions [Folland, Ch9], but these are different from \underline{distribution} functions.

\begin{definition}
Suppose $X$ a topological space. $\mu$ on $(X, \mathcal{B}(X))$ is called \underline{locally finite} is $\mu(K) < \infty$ for any compact set $K \subset X$.
\end{definition}

\begin{lemma}
Let $\mu$ be a locally finite Borel measure on $\R \implies$
\[
F_\mu(x) = \begin{cases}
\mu((0, x]), & x > 0 \\
0, & x = 0 \\
- \mu((x, 0]), & x < 0
\end{cases} \text{ is a distribution function.}
\]
\end{lemma}
\begin{proof}
DIY. Use continuity of measure.
\end{proof}

\begin{definition}
$h$-intervals are $\emptyset, (a, b], (a, \infty), (-\infty, b], (\infty, \infty)$.
\end{definition}

\begin{lemma}
Let $\mathcal{H}$ be the collections of finite disjoint unions of $h$-intervals. Then $\mathcal{H}$ is an algebra on $\R$.
\end{lemma}
\begin{proof}
DIY.
\end{proof}

\begin{proposition}[Distribution function defines a pre-measure]
Let $F: \R \to \R$ be a distribution function. For an $h$-interval $I$, define
\[
\ell(I) = \ell_F(I) = \begin{cases}
0,  & I = \emptyset \\
F(b) - F(a), & I = (a, b] \\
F(\infty) - F(a), & I = (a, \infty) \\
F(b) - F(\infty), & I = (-\infty, b] \\
F(\infty) - F(-\infty), & I = (-\infty, \infty). 
\end{cases}
\]
Define $\mu_0 = \mu_{0, F}: \mathcal{H} \to [0, \infty]$ by
\[
\mu_0(A) \defeq \sum_{k=1}^N \ell(I_k) \quad \text{if } A = \bigcup_{k=1}^N I_k, \text{ finite disjoint union of $h$-invervals.}
\]
Then $\mu_0$ is a pre-measure.
\end{proposition}
\begin{proof}
\begin{enumerate}
\item $\mu_0$ is well-defined.
\item $\mu_0$ is finite additive.
\item $\mu_0$ is countable additive within $\mathcal{H}$.

Suppose $A \in \mathcal{H}$ and $A = \bigcup_1^\infty A_i$ a disjoint union, $A_i \in \mathcal{H}$. It is enough to consider the case $A = I, A_k = I_k$ all $h$-intervals. (Why?)

Focus on the case $I = (a, b]$: (HW: check other cases)\\
We have
\[(a, b] = \bigcup_1^\infty(a_n, b_n], \text{ a disjoint union}.\]
Check
\[F(b) - F(a) ?= \sum_1^\infty(F(b_n) - F(a_n))\]
$(a, b] \supset \bigcup_1^N (a_n, b_n] \implies F(b) - F(a) \geq \sum_1^N F(b_n) - F(a_n), \forall N \in \N$. (Arranging them in decreasing order) Take $N \to \infty$ we have
\[F(b) - F(a) \geq \sum_1^\infty(F(b_n) - F(a_n)).\]

Since $F$ is right-continuous, $\exists a' > a \st F(a') - F(a) < \varepsilon$.
For each $n \in \N$, $\exists b'_n > b_n \st F(b'_n) - F(b_n) < \frac{\varepsilon}{2^n}$ 

$\implies [a', b] \subset \bigcup_1^\infty(a_n, b'_n)$
$\implies \exists N \in \N \st [a', b] \subset \bigcup_1^n (a_n, b'_n)$
$\implies F(b) - F(a') \leq \sum_1^N F(b'_n) - F(a_n)$
$\implies F(b) - F(a) \leq F(b) - F(a') + \varepsilon \leq \sum_1^\infty(F(b'_n)-F(a_n)) + \varepsilon \leq \sum_1^\infty(F(b_n)-F(a_n) + \frac{\varepsilon}{2^n}) + \varepsilon$

\end{enumerate}
\end{proof}


\end{document}
