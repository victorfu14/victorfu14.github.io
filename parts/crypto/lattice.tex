\documentclass{report}
\usepackage{amsmath,amssymb,amsthm,textcomp,gensymb,nccmath}
\usepackage{mathtools}
\renewcommand{\qedsymbol}{$\blacksquare$}

\setlength{\topmargin}{0.5in}
\usepackage[margin=4cm]{geometry}
\usepackage{enumerate}

\usepackage{setspace}
\onehalfspacing
\usepackage{parskip}
\setlength{\parskip}{0.5em}
\usepackage[T1]{fontenc}
\usepackage{palatino}

% useful characters/operators
\newcommand{\R}{\mathbb{R}}
\newcommand{\C}{\mathbb{C}}
\newcommand{\Z}{\mathbb{Z}}
\newcommand{\Q}{\mathbb{Q}}
\newcommand{\N}{\mathbb{N}}
\newcommand{\F}{\mathbb{F}}
\newcommand{\B}{\mathbb{B}}
\newcommand{\matP}{\mathbb{P}}
\newcommand{\matE}{\mathbb{E}}
\newcommand{\matS}{\mathbb{S}}
\newcommand{\matH}{\mathbb{H}}
\newcommand{\matT}{\mathbb{T}}
\newcommand{\st}{\ s.t.\ }
\newcommand{\ie}{\ i.e.\ }
\newcommand{\eg}{\ e.g.\ }
\def \diam {\operatorname{diam}}
\def \Hom {\operatorname{Hom}}
\def \id {\operatorname{id}}
\def \tr {\operatorname{tr}}
\def \rk {\operatorname{rk}}
\def \sp {\operatorname{span}}
\def \dist {\operatorname{dist}}
\def \intr {\operatorname{int}}
\def \sgn {\operatorname{sgn}}
\def \im {\operatorname{Im}}
\def \re {\operatorname{Re}}
\def \curl {\operatorname{curl}}
\def \divg {\operatorname{div}}
\def \GL {\operatorname{GL}}
\def \Aut {\operatorname{Aut}}
\def \per {\operatorname{per}}
\def \LE {\operatorname{LE}}
\def \indeg {\operatorname{indeg}}
\def \outdeg {\operatorname{outdeg}}
\def \Par {\operatorname{Par}}
\def \Gr {\operatorname{Gr}}
\def \del {\operatorname{del}}
\def \add {\operatorname{add}}
\newcommand{\pdr}[2]{\dfrac{\partial #1}{\partial #2}}
\newcommand{\df}{\mathrm{d}}
\newcommand{\dr}[2]{\dfrac{\df #1}{\df #2}}
\newcommand{\inner}[2]{\left\langle #1, #2\right\rangle}
\newcommand{\qbin}[2]{\begin{bmatrix}{#1}\\ {#2}\end{bmatrix}_q}


% arrows and :=, =:
\makeatletter
\providecommand*{\twoheadrightarrowfill@}{%
  \arrowfill@\relbar\relbar\twoheadrightarrow
}
\providecommand*{\twoheadleftarrowfill@}{%
  \arrowfill@\twoheadleftarrow\relbar\relbar
}
\providecommand*{\xtwoheadrightarrow}[2][]{%
  \ext@arrow 0579\twoheadrightarrowfill@{#1}{#2}%
}
\providecommand*{\xtwoheadleftarrow}[2][]{%
  \ext@arrow 5097\twoheadleftarrowfill@{#1}{#2}%
}
\makeatother

\newcommand{\defeq}{\vcentcolon=}
\newcommand{\eqdef}{=\mathrel{\mathop:}}

% integral for measure theory
\newcommand{\lowerint}{\underline{\int_{\R^d}}}
\newcommand{\upperint}{\overline{\int_{\R^d}}}
\newcommand{\lint}[1]{\underline{\int_{\R^d}} #1 (x)dx}
\newcommand{\uint}[1]{\overline{\int_{\R^d}} #1 (x)dx}
\newcommand{\sint}[1]{\simp{\int_{\R^d} #1 (x)dx}}
\newcommand{\lesint}[1]{\int_{\R^d} #1 (x)dx}

% note taking
\newcommand{\fancyem}[1]{\underline{\textsc{#1}}}

% theorem style
\newtheorem{theorem}{Theorem}[section]
\newtheorem{corollary}{Corollary}[section]
\newtheorem{lemma}{Lemma}[section]
\newtheorem{conjecture}{Conjecture}[section]
\newtheorem{proposition}{Proposition}[section]

\theoremstyle{definition}
\newtheorem{definition}{Definition}[section]
\newtheorem{example}{Example}[section]
\theoremstyle{remark}
\newtheorem*{remark}{Remark}

% pseudocode and algorithms
\usepackage{algorithm}
\usepackage{algpseudocode}
\usepackage{algorithmicx}
\counterwithin{algorithm}{section}
\renewcommand{\algorithmicrequire}{\textbf{Input:}}
\renewcommand{\algorithmicensure}{\textbf{Output:}}

% for clearer reference
\usepackage{hyperref}
\newcommand{\corollaryautorefname}{Corollary}
\newcommand{\lemmaautorefname}{Lemma}
\newcommand{\definitionautorefname}{Definition}
\newcommand{\exampleautorefname}{Example}
\newcommand{\conjectureautorefname}{Conjecture}
\renewcommand{\subsectionautorefname}{Section}
\newcommand{\algorithmautorefname}{Algorithm}

% other styling
\usepackage{fancyvrb, fancyhdr}
\usepackage{tikz-cd}
\usepackage{tikz}
\PassOptionsToPackage{usenames, x11names}{xcolor}
\usepackage{tcolorbox}
\selectcolormodel{cmy}

\newcommand{\edge}{
    \begin{tikzcd}[cramped, sep=small, labels={font=\everymath\expandafter{\the\everymath\textstyle}}]
        u \arrow[r, "e", no head] & v
    \end{tikzcd}
}


\pagestyle{fancy}
\fancyhead[LO,L]{\leftmark}
\fancyhead[RO,R]{Yiwei Fu}
% \fancyhead[C]{MATH 566}
\fancyfoot[CO,C]{\thepage}
\renewcommand{\sectionmark}[1]{\markboth{#1}{#1}}

\numberwithin{equation}{section}

\newcommand{\fnl}{\parbox[t]{0\linewidth}{}}
\newcommand*\ttlmath[2]{\texorpdfstring{$\boldsymbol{#1}$}{#2}}

\usepackage{epigraph}

% \epigraphsize{\small}% Default
\setlength\epigraphwidth{8cm}
\setlength\epigraphrule{0pt}

\usepackage{etoolbox}

\makeatletter
\patchcmd{\epigraph}{\@epitext{#1}}{\itshape\@epitext{#1}}{}{}
\makeatother

% combinatorics special
\usepackage{pgfopts}
\usepackage{ytableau}

\begin{document}
\title{Notes for}
\author{Yiwei Fu, Instructor: Chris Peikert}
\date{FA 2022}
\maketitle


\tableofcontents
Office hours: 

\clearpage
\pagenumbering{arabic}

\chapter{Introduction}
\section{Brief History of Lattices in Cryptography}
In late to mid 18th century the concept of lattice is start to be formally defined by Gauss, Laplace, etc. It was used in number theory to give proofs such as quadratic reciprocity and the four square theorem. Minkowski went all in ``Geometry of Numbers'', generalizing to higher dimensions.

In early 1980s, Lenstra-Lenstra-Lovasz: `LLL' algorithm for lattice basis reduction. Other applications include breaking many proposed cryptosystems.

In mid 1990s (1996), Ajtai proves ``worse-case to average-case'' reduction for lattices. This gives a secure crypto primitive, assuming certain lattice problems are hard. 

NTRU(Hoffstein, Pipher, Silverman) cryptosystem (no theoretical soundness found yet): seems secure!

From 2000s onward, a better, faster, stronger, smaller, cleaner, simpler, \ldots

2016-, NIST, Goggle, Cloudfare: `post-quantum' secure proposals, deployments of lattice based cryptography.

\section{Lattice}
\begin{definition}
    An $n$-dimentional lattice is a discrete additive subgroup of $\R^n$.
\end{definition}

Every lattice (except the degenerate case) is infinite. But it has a finite representation through basis.
\begin{definition}
    A basis $B$ is a set of vectors $\{b_1, \ldots, b_n\}$ of a lattice $\mathcal{L}$ is a set of linearly independent vectors whose integer linear combinations generates $\mathcal{L}$.

    Equivalently, view $B$ as a nonsingular matrix whose $i$-th column is $b_i$, then $\mathcal{L} = B \cdot \Z^n = \{Bz: z \in \Z^n\}$.
\end{definition}

\begin{theorem}
    Every lattice has a basis. (not unique)
\end{theorem}
\begin{lemma}\label{lem:uni}
    Suppose bases $B_1, B_2$ generate the same lattice $\mathcal{L}$ if and only if there exists an unimodular matrix $U \in \Z^{n \times n}, \det U = \pm 1$ such that $B_1 = B_2\cdot U$.
\end{lemma}

\begin{corollary}
    The bases of $Z^n$ are exactly the unimodular matrices $U \in \Z^{n \times n}$.
\end{corollary}
\begin{corollary}
    Given two sets $B_1, B_2$ we can efficiently test whether they generate the same lattice by check is $B_2^{-1}, B_1$ is unimodular.
\end{corollary}

\begin{proof}[Proof of \autoref{lem:uni}]
    Suppose $B_1, B_2$ generates the same lattice $\mathcal{L}$ i.e. $B_1\cdot \Z^n = B_2 \cdot \Z^n$. So each column of $B_1$ is an integer linear combination of $B_2$'s columns: $B_1 = B_2\cdot X$ for some $X \in \Z^{n \times n}$.

    Similarly, we have $B_2 = B_1 \cdot Y$. So $B_2 = B_2 (X \cdot Y) \implies X \cdot Y = \id \implies \det(X) \det(Y) = 1$.

    Suppose $B_1 = B_2 \cdot U$ for unimodular $U$. Then $B_1 \cdot \Z^n = B_2 \cdot U \cdot Z^n = B_2 \cdot \Z^n$. (for any $z \in Z^n$, we have $U(U^{-1}z) = z$. $U^{-1}z$ is an integral vector since $U^{-1}$ is an integer matrix.)
\end{proof}




\end{document}
